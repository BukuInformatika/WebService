\section{Backend}
Back-end atau server-side adalah merupakan bagaimana sebuah website berkerja, meng-update
dan berubah. back-end adalah sesuatu system yang dimana User tidak akan dapat melihatnya di dalam browser,
seperti Database dan Server. biasanya orang - orang yang bekerja di bagian back-end di panggil atau disebut sebagai
Programmers atau Developer. Back-end developers adalah orang yang paling khawatir tentang hal - hal yang menyangkut keamanan,
struktur sistem dan manajemen konten. sebenarnya para back-end develper juga mengetahui tentang front-end seperti HTML dan CSS.
namun itu bukanlah bidang mereka bekerja. 

\section{TugasBackend}
Back-end biasanya mengacu pada program dan skrip yang bekerja di dalam server, untuk membuat sebuah halaman web yang dinamis dan interatif. Back-end memiliki tugas-tugas yaitu seperti :

Desain Informasi pada web
Pemrosesan form
Pemrograman dalam database
Aplikasi Berbasis Web

Dari tugas - tugas tersebut Back-end memiliki tiga bagian diantaranya yaitu server,aplikasi, dan database.

\section{BahasaPemrogramanBackend}
Bahasa Pemrograman yang digunakan diBack-end untuk mejadi acuan antara lain :
PHP
 	adalah salah satu server side untuk dirancang khusus untuk aplikasi web dan PHP disisipkan diantara bahasa HTML sebab bahasa server side, maka dieksekusi diserver. sehingga yang kirimkan ke browser adalah hasil jadi dalam bentuk HTML. PHP termasuk Open Source Product dapat diubah disource code dan mendistribusikan secara bebas.

\section{JavaBackend}
Dalam backend juga menggunaka javascriptyaitu suatu format teks untuk mengserialisasi data yang terstruktur yang berasal dari objek literal JavaScripts.Pada JavaScript dapat mewakili dari empat tipe yaitu string, angka, boolean, dan nul, dan ada juga dua tipe terstruktur yaitu objek dan array. Objek kumpulan yang tanpa batas dari nol ata lebih dari nama atau nilai pasangannya, yang dimana nama ialah string sedangkan nilai ialah angka, boolean, objek, dan null. Array yaitu urutan dari nol atau melebihi banyak nilai.

\sectian{DataBaseBackend}
Pada sistem basis data yaitu telah lama dilanda masalah kinerja pada penigkatan dalam penggunaan mainframe atau di dalam aplikasi basis data. Dan solusi untuk masalah ini yaitu dengar membongkar sistem basis data dari komputer mainframe ke komputer backend. Pada komputer itu memiliki penyimpanan disk sendiri, digunakan juga untuk melakukan semuoa operasi data base, dan saling berinteraksi dengan mainframe.






		

