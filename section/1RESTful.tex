%\begin{itemize}
%\item Ahmad Syafrizal Huda (1164062) 
%\item Annisa Fathoroni (1164067) 
%\item Puad Hamdani (1164084) 
%\item Rahmi Roza (1164085) 
%\item Tasya Wiendhyra (1164086) 
%\end{itemize}

\section{Definisi RESTful}
RESTful adalah salah satu teknologi web service untuk membuat suatu sistem yang terdistribusi dimana cara kerjanya berdasarkan resource. RESTful sendiri merupakan software yang didesain untuk penekanan pada skalabilitas,kesederhanaan dan kegunaan. Metode dalam REST terdiri dari empat prinsip utama teknologi, yaitu \cite{aji2016penerapan}:
\begin{enumerate}
\item Resource identifier melalui Uniform Resource Identifier (URI), REST Web service mencari sekumpulan sumber daya yang mengidentifikasi interaksi antar klien. 
\item Uniform interface, sumber daya yang dimanipulasi CRUD (Create, Read, Update, Delete) menggunakan operasi PUT, GET, POST, dan DELETE.
\item Self-descriptive messages, sumberdaya informasi tidak terikat, sehingga dapat mengakses berbagai format konten (HTML, XML, PDF, JPEG, Plain Text dan lainnya). Metadata pun dapat digunakan. 
\item Stateful interactions melalui hyperlinks, setiap interaksi dengan suatu sumber daya bersifat stateless, yaitu request messages tergantung jenis kontennya.
\end{enumerate}


