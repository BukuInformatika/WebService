%KELOMPOK 4
%\begin{enumerate}
%\item Dadi Hasanudin
%\item Farhan Maulana
%\item Septi Nurhidayah
%\item Seta Permana
%\item Wulan Dwi Hartati
%\end{enumerate}

\section{Pengenalan Linux dan Windows}
Sistem Operasi

Sebenarnya untuk setiap mesin yang ada pasti mempunyai system untuk mengoperasikannya. System operasi berperan sebagai penghubung antara pengguna mesin dengan perangkat keras mesin tersebut. System operasi bisa diartikan sebagai system yang terdiri atas komponen-komponen kerja yang berisi metode kerja yang digunakan untuk memanfaatkan mesin tersebut sehingga mesin dapat bekerja sesuai dengan yang diinginkan.

Sistem operasi  adalah program yang mengontrol jalannya aplikasi dan juga sebagai interface antara pengguna dan perangkat keras. Sistem operasi memiliki 3 tujuan, yang pertama adalah komputer akan lebih mudah dan nyaman untuk dipakai, yang kedua adalah mempunyai kemungkinan sumber daya sistem komputer digunakan dengan efisien, dan yang ketiga adalah demi tercapainya pengembangan yang baik, pengujian, dan tidak menggaggu layanan yang ada maka sistem operasi harus dikontrol.

Sistem operasi di desain untuk menyediakan berbagai layanan untuk penggunanya. Linux merupakan salah satu sistem operasi yang bersifat multi user dan multi tasking. Sistem operasi ini mencakup ratusan program. Microsoft windows memiliki kemiripan dengan linux, yaitu terletak pada file sistem yang bersifat hirarki yang tidak mendukung multi user dan multi tasking.

Dan ada yang menjadi perbedaan Linux dengan sistem operasi lainnya yaitu mengenai source code. Source code pada Linux tersedia untuk semua orang sehingga mudah dalam pengembangan. Karena Linux tersedia bebas, berbagai vendor telah menyediakan paket distribusi yang bisa dianggap versi kemasan Linux. Paket ini tersedia lingkungan Linux lengkap, perangkat lunak untuk instalasi dan termasuk perangkat lunak khusus dan dukungan khusus.

Linux disusun dengan standar sistem operasi POSIX yang diturunkan berdasarkan fungsi kerja UNIX. Secara umum Linux lebih cepat dibanding UNIX lain pada hardware yang sama dan UNIX tidak bersifat free.
MS-DOS mirip dengan Linux yaitu file sistem bersifat hirarkis. MS-DOS hanya dapat dijalankan pada prosesor x86 dan tidak adanya multi-user dan multi-tasking, yang juga tidak bersifat free. Kemudian ada MS Windows yang mempunyai kemampuan grafis yang ada di Linux dan juga kemampuan networking serta mempunyai kekurangan yang sama pada MS-DOS.

\section{Kelebihan Linux}
\begin{enumeratae}
\item Adanya proses khusus yaitu terminal, printer dan device hardware lainnya bisa diakses seperti mengakses file yang tersimpan dalam harddisk.
\item Sistem operasi yang menyediakan multi-tasking dan multi-user.
\item Saat ada program yang berjalan dimulai dari harddisk ke dalam RAM dan setelah dijalankan disebut proses.
\item Memberikan servis untuk membuat, memodifikasi program, proses dan file.
\item Mendukung struktur file yang mempunyai sifat hirarki.
\end{enumerate}

Linux juga tentu memiliki kekurangan, dari sekian banyak kekurangannya akan disebutkan 3 saja disini. Yang pertama adalah linux tidak memiliki sifat “friendly user”, kemudian yang kedua adalah masalah dibagian instalasi nya yang kebanyakan orang masih bingung bagaimana cara intal linux disbanding dengan windows, dan yang terakhir adalah masih kurang terkenal atau mungkin bisa dibilang belum karena setiap orang lebih mengenal dan familiar dengan windows dibandingkan linux.

Subsystem

Tentang subsystem, beberapa aplikasi  seperti bash script atau FLOSS yang ada di linux tidak akan memanggil layanan sistem operasi contohnya adalah Windows NT yang asli secara langsung, tapi justru sebaliknya, aplikasi ini akan melalui satu atau lebih subsistem dynamic-link libraries (DLL). Yang dilakukan DLL subsistem disini adalah untuk menerjemahkan fungsi yang tidak terdokumentasi menjadi terdokumentasi ke dalam panggilan layanan sistem Windows NT.

Windows NT memiliki desain yang berlapis dengan bagian low-level dari sistem yang di mana arsitektur atau platform yang spesifik diisolasi ke dalam modul yang terpisah. Sehingga lapisan atas dari sistem dapat terlindungi dari adanya perbedaan-perbedaan antara platform dari hardware. Dua komponen penting yang menyediakan portabilitas system operasi adalah HAL dan kernel.


\section{Windows Subsystem for Linux}
Windows Subsystem for Linux (WSL) atau yang biasa disebut dengan Bash on Ubuntu on Windows yang dimana WSL ini diluncurkan dan dikenalkan pada Windows 10. WSL menggunakan GNU/Linux dari Ubuntu. Hal inilah yang memungkinkan pengguna untuk dapat memakai manajer paket dari Ubuntu yaitu apt untuk install, update, dan uninstall aplikasi yang berasal dari repositori Ubuntu tersebut.

GNU/Linux adalah sistem yang paling cepat berkembang yang biasa disebut “Linux” setelah kernel. Tetapi dalam faktanya sistem ini lebih dari sekedar kernel. Komponen dari sistem ini merupakan campuran dari dua jenis software, yaitu free software dan open source software. Kedua jenis ini mengharuskan end user diizinkan untuk mengakses source code yang digunakan untuk membuat software. Sistem GNU/Linux yang lengkap terdiri dari UNIX, GNU, dan Linux.

Distribusi Linux Ubuntu yang nantinya akan terinstall di Windows 10 akan memilih keterbatasan, seperti terletak pada antarmuka grafis. Dikarenakan hal ini hanya subsystem maka nantinya akan ada juga beberapa aplikasi yang tidak bias dijalankan oleh WSL ini tidak bias langsung aktif sendiri setelah kita selesai install Windows 10, namun kita harus terlerbih dahulu mengaktifkan fitur tersebut jika ingin menggunakannya.

WSL ini ditujukan untuk para developer (pengembang) atau programmer Windows 10 yang ingin menggunakan sistem operasi Linux Ubuntu tanpa harus melakukan install ulang sistem operasi Windows menjadi Linux Ubuntu ataupun melakukan dualbooting antara Linux Ubuntu dengan Windows. WSL ini juga merupakan sebuah subsistem yang pastinya jika digunakan tidak akan menghabiskan banyak memori dan tentunya juga lebih ringan. Sehingga developer ataupun programmer bisa mengembangkan aplikasinya tanpa terkendala dengan adanya WSL ini.

WSL adalah kumpulan komponen yang memungkinkan biner ELF64 Linux asli untuk berjalan di Windows. Ini berisi mode pengguna dan komponen mode kernel. Ini terutama terdiri dari:
Layanan pengelola sesi mode pengguna yang menangani siklus hidup instance Linux
Pengandar penyedia Pico (lxss.sys, lxcore.sys) yang meniru kernel Linux dengan menerjemahkan syscalls Linux
Proses Pico yang menghosting mode pengguna Linux yang tidak dimodifikasi (mis. / Bin / bash)
Ini adalah ruang antara mode pengguna Linux binari dan komponen kernel Windows di mana keajaiban terjadi.

WSL mengeksekusi biner ELF64 Linux yang tidak dimodifikasi dengan memvirtualisasi antarmuka kernel Linux di atas kernel Windows NT. Salah satu antarmuka kernel yang dihadapkan adalah panggilan sistem. Syscall adalah layanan yang disediakan oleh kernel yang dapat dipanggil dari mode pengguna. Baik kernel Linux dan kernel Windows NT memaparkan beberapa ratus syscall ke mode pengguna, tetapi mereka memiliki semantik yang berbeda dan umumnya tidak secara langsung kompatibel.

Subsistem Windows untuk Linux termasuk driver mode kernel (lxss.sys dan lxcore.sys) yang bertanggung jawab untuk menangani permintaan panggilan sistem Linux dalam koordinasi dengan kernel Windows NT. Driver tidak mengandung kode dari kernel Linux tetapi sebaliknya merupakan implementasi ruang bersih dari antarmuka kernel yang kompatibel dengan Linux. Pada Linux asli, ketika syscall dibuat dari mode pengguna yang dapat dieksekusi, ia ditangani oleh kernel Linux. 

\section{Integrating Flexible Support for Security Policies into the Linux Operating System}
Mekanisme perlindungan sistem operasi utama saat ini tidak memadai untuk mendukung persyaratan kerahasiaan dan integritas untuk sistem akhir. Kontrol akses wajib (MAC) diperlukan untuk mengatasi persyaratan tersebut, tetapi keterbatasan MAC tradisional telah menghambat pengadopsiannya ke dalam sistem operasi utama. National Security Agency (NSA) bekerja dengan Secure Computing Corporation (SCC) untuk mengembangkan arsitektur MAC fleksibel yang disebut Flask untuk mengatasi keterbatasan MAC tradisional. NSA telah mengimplementasikan arsitektur ini dalam sistem operasi Linux, menghasilkan prototipe Protected-Enhanced Linux (SELinux), untuk membuat teknologi tersedia untuk komunitas yang lebih luas dan untuk memungkinkan penelitian lebih lanjut ke dalam sistem operasi yang aman. NAI Labs telah mengembangkan contoh konfigurasi kebijakan keamanan untuk menunjukkan manfaat arsitektur dan menyediakan landasan bagi orang lain untuk digunakan. Makalah ini menjelaskan arsitektur keamanan, mekanisme keamanan, antarmuka pemrograman aplikasi, konfigurasi kebijakan keamanan, dan kinerja SELinux.

NSA menciptakan Linux Security-Enhanced, atau SELinux, dengan mengintegrasikan arsitektur yang disempurnakan ini ke dalam sistem operasi Linux. Ini telah diterapkan pada subsistem utama dari Linux kernel, termasuk integrasi kontrol akses wajib untuk operasi pada proses, file, dan soket. NAI Laboratorium telah menyatukan upaya dan menerapkan beberapa kontrol akses wajib kernel tambahan, termasuk kontrol untuk sistem file procf dan devpts.

Arsitektur Keamanan
Bagian ini memberikan gambaran implementasi SELinux dari bagian arsitektur. Dalam Implementasi SELinux, server keamanan hanya subsistem kernel dan kernel lainnya subsistem (mis. manajemen proses, sistem file, socket IPC, System V IPC) adalah manajer objek. Komponen dalam sistem yang menegakkan keamanan kebijakan disebut sebagai manajer objek dalam Flask Arsitektur.


Mekanisme Keamanan
Kontrol Proses
Prosesnya mengeksekusi izin digunakan untuk mengontrol kemampuan suatu proses untuk dieksekusi dari gambar yang dapat dieksekusi. Izin ini dicentang antara label proses transformasi dan label yang dapat dieksekusi pada setiap eksekusi program.
Kontrol File
Karena deskripsi file terbuka dapat diwariskan di seberang atau di alihkan melalui UNIX socket IPC, SELinux label dan kontrol buka deskripsi file. Deskripsi file terbuka diberi label dengan SID dari proses pembuatannya, karena negara bagiannya biasanya diperlakukan sebagai bagian dari proses privat.
Kontrol Soket
SELinux menyediakan kendali atas soket IPC melalui a mengatur kontrol berlapis atas soket, pesan, simpul, dan antarmuka jaringan. Saat ini, prototipe SELinux hanya menyediakan label dan kontrol untuk INET dan Soket domain UNIX. Pada lapisan soket, SELinux mengontrol kemampuan proses untuk melakukan operasi di soket. Pada lapisan transport, SELinux mengontrol kemampuan soket untuk berkomunikasi dengan soket lain. Di lapisan jaringan, SELinux mengontrol kemampuan untuk mengirim dan menerima pesan di antarmuka jaringan, dan itu mengontrol kemampuan untuk mengirim pesan ke node dan menerima pesan dari node. SELinux juga mengendalikan kemampuannya proses untuk mengkonfigurasi antarmuka jaringan dan memanipulasi tabel routing kernel.

\section{Install Windows Subsystem for Linux}

\subsection{Persiapan}
Sebelum menginstall Windows Subsystem for Linux (WSL), yang harus kita lakukan adalah menyiapkan terlebih dahulu environment windows yang harus sesuai dengan kebutuhan, agar WSL tersebut bisa teinstall dengan baik. Environment windows yang dibutuhkan adalah dimana perangkat yang bisa digunakan untuk menginstall WSL harus memiliki prosesor 64 bit dan setidaknya minimal Windows 10 Anniversary Update (build 14393.0.).



