\documentclass{Install the Windows Subsystem for Linux}
\usepackages[indonesian]{babel}
\title{Install the Windows Subsystem for Linux \LaTex}
\author{Kelompok 4
1. Farhan Maulana       1164012
2. Seta Permana         1164027
3. Dadi Hasanudin       1164006
4. Septi Nurhidayah     1164027
5. Wulan Dwi Hartati    1164030}
\date{4 April 2017}
\begin{document}
\maketitle
\section{Pengenalan Linux dan Windows}
Sistem operasi di desain untuk menyediakan berbagai layanan untuk penggunanya. Linux merupakan salah satu sistem operasi yang bersifat multi user dan multi tasking. Sistem operasi ini mencakup ratusan program. Microsoft windows memiliki kemiripan dengan linux, yaitu terletak pada file sistem yang bersifat hirarki yang tidak mendukung multi user dan multi tasking.
Dan ada yang menjadi perbedaan Linux dengan sistem operasi lainnya yaitu mengenai source code. Source code pada Linux tersedia untuk semua orang sehingga mudah dalam pengembangan. Karena Linux tersedia bebas, berbagai vendor telah menyediakan paket distribusi yang bisa dianggap versi kemasan Linux. Paket ini tersedia lingkungan Linux lengkap, perangkat lunak untuk instalasi dan termasuk perangkat lunak khusus dan dukungan khusus.
\end{document} 