\documentclass{Install the Windows Subsystem for Linux}
\usepackages[indonesian]{babel}
\title{Install the Windows Subsystem for Linux \LaTex}
\author{Kelompok 4
1. Farhan Maulana       1164012
2. Seta Permana         1164027
3. Dadi Hasanudin       1164006
4. Septi Nurhidayah     1164027
5. Wulan Dwi Hartati    1164030}
\date{4 April 2017}
\begin{document}
\maketitle
\section{Pengenalan Linux dan Windows}
Sistem operasi di desain untuk menyediakan berbagai layanan untuk penggunanya. Linux merupakan salah satu sistem operasi yang bersifat multi user dan multi tasking. Sistem operasi ini mencakup ratusan program. Microsoft windows memiliki kemiripan dengan linux, yaitu terletak pada file sistem yang bersifat hirarki yang tidak mendukung multi user dan multi tasking.

Dan ada yang menjadi perbedaan Linux dengan sistem operasi lainnya yaitu mengenai source code. Source code pada Linux tersedia untuk semua orang sehingga mudah dalam pengembangan. Karena Linux tersedia bebas, berbagai vendor telah menyediakan paket distribusi yang bisa dianggap versi kemasan Linux. Paket ini tersedia lingkungan Linux lengkap, perangkat lunak untuk instalasi dan termasuk perangkat lunak khusus dan dukungan khusus.

Tentang subsystem, beberapa aplikasi  seperti bash script atau FLOSS yang ada di linux tidak akan memanggil layanan sistem operasi contohnya adalah Windows NT yang asli secara langsung, tapi justru sebaliknya, aplikasi ini akan melalui satu atau lebih subsistem dynamic-link libraries (DLL). Yang dilakukan DLL subsistem disini adalah untuk menerjemahkan fungsi yang tidak terdokumentasi menjadi terdokumentasi ke dalam panggilan layanan sistem Windows NT.

\section{Integrating Flexible Support for Security Policies into the Linux Operating System}
Mekanisme perlindungan sistem operasi utama saat ini tidak memadai untuk mendukung persyaratan kerahasiaan dan integritas untuk sistem akhir. Kontrol akses wajib (MAC) diperlukan untuk mengatasi persyaratan tersebut, tetapi keterbatasan MAC tradisional telah menghambat pengadopsiannya ke dalam sistem operasi utama. National Security Agency (NSA) bekerja dengan Secure Computing Corporation (SCC) untuk mengembangkan arsitektur MAC fleksibel yang disebut Flask untuk mengatasi keterbatasan MAC tradisional. NSA telah mengimplementasikan arsitektur ini dalam sistem operasi Linux, menghasilkan prototipe Protected-Enhanced Linux (SELinux), untuk membuat teknologi tersedia untuk komunitas yang lebih luas dan untuk memungkinkan penelitian lebih lanjut ke dalam sistem operasi yang aman. NAI Labs telah mengembangkan contoh konfigurasi kebijakan keamanan untuk menunjukkan manfaat arsitektur dan menyediakan landasan bagi orang lain untuk digunakan. Makalah ini menjelaskan arsitektur keamanan, mekanisme keamanan, antarmuka pemrograman aplikasi, konfigurasi kebijakan keamanan, dan kinerja SELinux.

\end{document} 