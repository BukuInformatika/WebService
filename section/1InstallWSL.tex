%KELOMPOK 4
%\begin{enumerate}
%\item Dadi Hasanudin
%\item Farhan Maulana
%\item Septi Nurhidayah
%\item Seta Permana
%\item Wulan Dwi Hartati
%\end{enumerate}

\section{Pengenalan Linux dan Windows}
Sistem Operasi

Sebenarnya untuk setiap mesin yang ada pasti mempunyai system untuk mengoperasikannya. System operasi berperan sebagai penghubung antara pengguna mesin dengan perangkat keras mesin tersebut. System operasi bisa diartikan sebagai system yang terdiri atas komponen-komponen kerja yang berisi metode kerja yang digunakan untuk memanfaatkan mesin tersebut sehingga mesin dapat bekerja sesuai dengan yang diinginkan.

Sistem operasi di desain untuk menyediakan berbagai layanan untuk penggunanya. Linux merupakan salah satu sistem operasi yang bersifat multi user dan multi tasking. Sistem operasi ini mencakup ratusan program. Microsoft windows memiliki kemiripan dengan linux, yaitu terletak pada file sistem yang bersifat hirarki yang tidak mendukung multi user dan multi tasking.

Dan ada yang menjadi perbedaan Linux dengan sistem operasi lainnya yaitu mengenai source code. Source code pada Linux tersedia untuk semua orang sehingga mudah dalam pengembangan. Karena Linux tersedia bebas, berbagai vendor telah menyediakan paket distribusi yang bisa dianggap versi kemasan Linux. Paket ini tersedia lingkungan Linux lengkap, perangkat lunak untuk instalasi dan termasuk perangkat lunak khusus dan dukungan khusus.

Tentang subsystem, beberapa aplikasi  seperti bash script atau FLOSS yang ada di linux tidak akan memanggil layanan sistem operasi contohnya adalah Windows NT yang asli secara langsung, tapi justru sebaliknya, aplikasi ini akan melalui satu atau lebih subsistem dynamic-link libraries (DLL). Yang dilakukan DLL subsistem disini adalah untuk menerjemahkan fungsi yang tidak terdokumentasi menjadi terdokumentasi ke dalam panggilan layanan sistem Windows NT.

Windows NT memiliki desain yang berlapis dengan bagian low-level dari sistem yang di mana arsitektur atau platform yang spesifik diisolasi ke dalam modul yang terpisah. Sehingga lapisan atas dari sistem dapat terlindungi dari adanya perbedaan-perbedaan antara platform dari hardware. Dua komponen penting yang menyediakan portabilitas system operasi adalah HAL dan kernel.

\section{Windows Subsystem for Linux}
Windows Subsystem for Linux (WSL) atau yang biasa disebut dengan Bash on Ubuntu on Windows yang dimana WSL ini diluncurkan dan dikenalkan pada Windows 10. WSL menggunakan GNU/Linux dari Ubuntu. Hal inilah yang memungkinkan pengguna untuk dapat memakai manajer paket dari Ubuntu yaitu apt untuk install, update, dan uninstall aplikasi yang berasal dari repositori Ubuntu tersebut.

GNU/Linux adalah sistem yang paling cepat berkembang yang biasa disebut “Linux” setelah kernel. Tetapi dalam faktanya sistem ini lebih dari sekedar kernel. Komponen dari sistem ini merupakan campuran dari dua jenis software, yaitu free software dan open source software. Kedua jenis ini mengharuskan end user diizinkan untuk mengakses source code yang digunakan untuk membuat software. Sistem GNU/Linux yang lengkap terdiri dari UNIX, GNU, dan Linux.

\section{Integrating Flexible Support for Security Policies into the Linux Operating System}
Mekanisme perlindungan sistem operasi utama saat ini tidak memadai untuk mendukung persyaratan kerahasiaan dan integritas untuk sistem akhir. Kontrol akses wajib (MAC) diperlukan untuk mengatasi persyaratan tersebut, tetapi keterbatasan MAC tradisional telah menghambat pengadopsiannya ke dalam sistem operasi utama. National Security Agency (NSA) bekerja dengan Secure Computing Corporation (SCC) untuk mengembangkan arsitektur MAC fleksibel yang disebut Flask untuk mengatasi keterbatasan MAC tradisional. NSA telah mengimplementasikan arsitektur ini dalam sistem operasi Linux, menghasilkan prototipe Protected-Enhanced Linux (SELinux), untuk membuat teknologi tersedia untuk komunitas yang lebih luas dan untuk memungkinkan penelitian lebih lanjut ke dalam sistem operasi yang aman. NAI Labs telah mengembangkan contoh konfigurasi kebijakan keamanan untuk menunjukkan manfaat arsitektur dan menyediakan landasan bagi orang lain untuk digunakan. Makalah ini menjelaskan arsitektur keamanan, mekanisme keamanan, antarmuka pemrograman aplikasi, konfigurasi kebijakan keamanan, dan kinerja SELinux.

NSA menciptakan Linux Security-Enhanced, atau SELinux, dengan mengintegrasikan arsitektur yang disempurnakan ini ke dalam sistem operasi Linux. Ini telah diterapkan pada subsistem utama dari Linux kernel, termasuk integrasi kontrol akses wajib untuk operasi pada proses, file, dan soket. NAI Laboratorium telah menyatukan upaya dan menerapkan beberapa kontrol akses wajib kernel tambahan, termasuk kontrol untuk sistem file procf dan devpts.


