
%Resume Tentang IDE Python D4TI2A Kelompok 3
%\begin{enumerate}
%\item Aldi Maulana Yusuf                		1164001
%\item Dieni Hanifah               			1164008
%\item Ikrima Ningrumsari Mulyana    		1164013
%\item Muhammad Farid Alwan Assyifa         	1164017
%\item Rojasqi Fadilla              			1164026
%\end{enumerate}


\section {Pengertian IDE}
IDE (Integrated Development Environment) merupakan sebuah program komputer yang mempunyai fasilitas - fasilitas yang 
diperlukan untuk pembangunan dan pengembangan software atau perangkat lunak. Sedangkan tujuan dari IDE ini sendiri 
yaitu sebagai penyedia utilitas - utilitas yang diperlukan untuk pembangunan maupun pengembangan software atau perangkat lunak.
Setidaknya IDE mempunyai fasilitas - fasilitas seperti, Editor (untuk menuliskan kode), Compiler (Untuk mengecek kode, 
dan mengubahnya kedalam bentuk binary), Linker (Untuk menyatukan data binary yang dihasilkan compiler), Debuger (Untuk mengetes jalanya program \cite{aripurnamayana2012rancangan}.

\section{Definisi Python}
Python adalah bahasa pemrograman multi-paradigma : pemrograman berorientasi obyek dan pemrograman terstruktursepenuhnya didukung, dan ada sejumlah fitur bahasa yang mendukung pemrograman fungsional dan pemrograman berorientasi aspek (termasuk oleh metaprogramming dan dengan metode sihir ).  Banyak paradigma lain didukung menggunakan ekstensi, termasuk desain dengan kontrak  dan pemrograman logika .
Python menggunakan pengetikan dinamis dan kombinasi penghitungan referensi dan pengumpul sampah pendeteksi siklus untuk manajemen memori . Fitur penting dari Python adalah resolusi nama dinamis (late binding), yang mengikat metode dan nama variabel selama eksekusi program.
Desain Python hanya menawarkan dukungan terbatas untuk pemrograman fungsional dalam tradisi Lisp . Bahasa memiliki fungsi map (), mengurangi () dan filter (), pemahaman untuk daftar, kamus, dan set, serta ekspresi generator. Perpustakaan standar memiliki dua modul (itertools dan functools) yang mengimplementasikan alat-alat fungsional yang dipinjam dari Haskell dan Standard ML \cite{van2007python}.

\subsection{Definisi Python}
	Pyton sendiri merupakan salah satu Language programming yang biasanya digunakan oleh para pengembang. Kebanyakan dari mereka mengembangkan aplikasi python hanya menggunakan editor yang ada maupun yang biasa. Namun tak sedikitpun para pengembang menggunakan Integrated Development Environment (IDE) untuk membantu mereka bekerja dengan proyek Python. Selain standar desktop Python IDE (integrated development environments), ada juga IDE berbasis IDEs, Sage, (dimaksudkan untuk mengembangkan ilmu pengetahuan dan program Python yang berhubungan dengan matematika), dan IDE yang di-hostingya.
	Sebagian besar implementasi Python (termasuk CPython) dapat berfungsi sebagai interpreter baris perintah , di mana pengguna memasukkan laporan secara berurutan dan menerima hasilnya dengan segera. Singkatnya, Python bertindak sebagai shell . Kerang lain menambah kemampuan di luar mereka di interpreter dasar, termasuk IDLE dan IPython . Meskipun umumnya mengikuti gaya visual shell Python, mereka menerapkan fitur seperti penyelesaian otomatis, retensi status sesi, dan penyorotan sintaks.
	Selain standar desktop Python IDE (lingkungan pengembangan terintegrasi), ada juga IDE berbasis browser , Sage , (dimaksudkan untuk mengembangkan sains dan program Python yang berhubungan dengan matematika), dan sebuah IDE yang di-hosting, pythonAnywhere. \cite{van2007python}

 Python Pada dasarnya Python merupakan perangkat lunak yang secara default termasuk dalam suatu paket distribusi GNU/Linux. Untuk GNU/Linux distribusi Slackware menggunakan Python versi 2.4, biasanya terdapat pada CD I direktori /slackware/d. Toolkit yang digunakan untuk melakukan instalasi paket di Slackware adalah installpkg, berikut langkah instalasinya:
  # mount /mnt/cdrom 
  # cd /mnt/cdrom/slackware/d 
  # installpkg python-2.4.1-i486-1.tgz \cite{utamipemrograman}.

\subsection{Definisi Python}
Python IDE merupakan program standar kemampuan konstruksi dan pelaksanaan, Python secara umum banyak pengembang profesional yang menggunakan nya 
lingkungan pengembangan terintegrasi (IDE) mendukung
Python dan beberapa pemrograman microworld berbasis Python lingkungan, seperti RUR-PLE17, Guido van Robot18, dan Turtlet19). ViLLE juga menyediakan beberapa dukungan untuk memvisualisasikan eksekusi Python dengan fungsi tersebut untuk menerjemahkan animasi kode Java ke yang setara animasi dengan Python, ViLLE  juga dapat menyediakan beberapa dukungan untuk visualisasi eksekusi Python dengan menggunakan fungsi tersebut untuk menerjemahkan animasi kode Java ke yang setara animasi dengan Python. aplikasi untuk memvisualisasikan kode seperti edukasi IDE (integrated development environments) secara terpisah dapat mengirimkan dan mendapatkan umpan balik jawaban mereka atas sistem penilaian berbasis web \cite{helminen2010jype}.

\subsection{Definisi Python}
Python adalah bahasa pemrograman interpretatif multiguna dengan filosofi perancangan yang berfokus pada tingkat keterbacaan kode. Python diklaim sebagai bahasa yang menggabungkan kapabilitas, kemampuan, dengan sintaksis kode yang sangat jelas dan dilengkapi dengan fungsionalitas pustaka standar yang besar serta komprehensif. Python mendukung multi paradigma pemrograman, namun tidak dibatasi pada pemrograman berorientasi objek, pemrograman imperatif, dan pemrograman fungsional. Salah satu fitur yang tersedia pada python adalah sebagai bahasa  
pemrograman dinamis yang dilengkapi dengan manajemen memori otomatis. Seperti halnya pada bahasa pemrograman dinamis lainnya, python umumnya digunakan sebagai bahasa skrip meski pada praktiknya penggunaan bahasa ini lebih luas mencakup konteks pemanfaatan yang umumnya tidak dilakukan dengan menggunakan bahasa skrip. Python dapat digunakan untuk berbagai keperluan pengembangan perangkat lunak dan dapat berjalan di berbagai platform sistem operasi. \cite{rosmalasarana}.

\subsection{Definisi Python}
Python adalah lingkungan pemrograman berbasis web untuk Python yang menghilangkan hambatan yang berhubungan dengan perangkat lunak untuk masuk ke programmer pemula, seperti menginstal IDE atau runtime Python. Hanya menggunakan browser web, dalam beberapa menit siswa dapat mulai menulis kode, menontonnya berjalan, dan mengakses materi dan tutorial dukungan. Meskipun ada sejumlah alat pengajaran Python berbasis web, Pythy berbeda dalam beberapa hal: ia mengatur tugas penugasan siswa, termasuk tenggat waktu, turn-in, dan grading; ini mendukung contoh kode interaktif dan hidup yang dapat ditulis oleh instruktur dan siswa dapat menjelajah; ini menyediakan penyimpanan otomatis kerja siswa di cloud, dengan kontrol versi penuh transparan; dan mendukung proyek-proyek gaya-komputasi yang memanipulasi gambar dan suara. Pythy menyediakan ekosistem lengkap untuk belajar siswa, dengan antarmuka pengguna yang mengikuti model penelusuran web yang lebih akrab, daripada antarmuka IDE yang berfokus pada pengembang. Evaluasi membandingkan persepsi siswa tentang Pythy dalam kaitannya dengan JES, lingkungan Python pemula yang ramah siswa. Pengalaman kelas menunjukkan bahwa Pythy memang mengurangi rintangan pemula yang ingin diatasi \cite{edwards2014pythy}.

\section{IDE Python}
Thonny adalah IDE Python baru untuk belajar dan mengajar pemrograman yang dapat membuat visualisasi program menjadi bagian alami dari alur kerja pemula.
Di antara fitur-fiturnya yang menonjol adalah cara yang berbeda untuk melakukan langkah-langkah melalui kode, langkah demi langkah expression evaluation, visualisasi rinci dari tumpukan panggilan dan mode untuk menjelaskan konsep referensi dan heap. Ini mendukung penelitian pendidikan dengan mencatat tindakan pengguna untuk mengulang atau menganalisis proses pemrograman. Ini gratis untuk digunakan dan terbuka untuk ekstensi \cite{annamaa2015introducing}.

\subsection{IDE python}
PyCharm adalah Python IDE dengan seperangkat set tools yang lengkap untuk pengembangan yang produktif dengan bahasa pemrograman Python. Selain itu, IDE ini menyediakan kemampuan kelas tinggi untuk pengembangan Web profesional dengan framework Django. Seperti IDE lainnya, PyCharm menawarkan editor kode cerdas yang memahami kekhususan Python dan menawarkan peningkatan produktivitas yang luar biasa seperti, pemformatan kode otomatis, penyelesaian kode, refactoring, impor otomatis, navigasi kode dengan sekali klik, dan banyak lagi. Didukung oleh rutinitas analisis kode tingkat lanjut, fitur-fitur ini menjadikan PyCharm alat yang ampuh di tangan pengembang Python profesional dan mereka yang baru memulai dengan teknologi tersebut \cite{fifli2016pirhoogammarhoalphamumualpha}.

\section{INSTALASI}
Berikut ini adalah cara instalasi bahasa pemrograman Python yang berjalan pada sistem operasi Windows. Langkah-langkahnya adalah: 
a. Pilih file “python-2.7.amd64”, tampil jendela “Python for Windows”, pilih “next” 
b. Pada jendela “Select Destination Directory”, pilih “next”
 c. Muncul jendela “Customize Python 2.7”, pilih “next
 d. Muncul jendela progress bar untuk proses instalasinya, setelah itu akan keluar jendela “Completing the Python 2.7 Installer”, dan pilih “finish” \citep{wicaksana2013user} 

\section{IDE Python}
Nama Python berasal dari serial televisi Monty Python Flying Circus dan umum digunakan referensi Monty Python dalam kode contoh. Sebagai contoh, variabel metasyntactic sering digunakan dalam literatur Python adalah spam dan telur , bukan foo dan bar tradisional .
Awalan Py- digunakan untuk menunjukkan bahwa ada sesuatu yang terkait dengan Python. Contoh penggunaan awalan ini di nama aplikasi Python atau perpustakaan termasuk Pygame, sebuah mengikat dari SDL untuk Python (yang biasa digunakan untuk membuat game); PyS60, implementasi untuk sistem operasi Symbian S60 ; PyQt dan PyGTK , yang mengikat Qt dan GTK , masing-masing, ke Python; dan PyPy , implementasi Python yang ditulis dengan Python.

