
%Resume Tentang IDE Python D4TI2A Kelompok 3
%\begin{enumerate}
%\item Aldi Maulana Yusuf                		1164001
%\item Dieni Hanifah               			1164008
%\item Ikrima Ningrumsari Mulyana    		1164013
%\item Muhammad Farid Alwan Assyifa         	1164017
%\item Rojasqi Fadilla              			1164026
%\end{enumerate}

\section{IDE Python}
Thonny adalah IDE Python baru untuk belajar dan mengajar pemrograman
yang dapat membuat visualisasi program menjadi bagian alami dari alur kerja pemula.
Di antara fitur-fiturnya yang menonjol adalah cara yang berbeda untuk melakukan langkah-langkah melalui kode, langkah demi langkah expression evaluation, visualisasi rinci dari tumpukan panggilan dan mode untuk menjelaskan konsep referensi dan heap. Ini mendukung penelitian pendidikan dengan mencatat tindakan pengguna untuk mengulang atau menganalisis proses pemrograman. Ini gratis untuk digunakan dan terbuka untuk ekstensi \cite{annamaa2015introducing}.

\subsection{IDE Pyton}
Python IDE merupakan program standar kemampuan konstruksi dan pelaksanaan, Python secara umum banyak pengembang profesional yang menggunakan nya 
lingkungan pengembangan terintegrasi (IDE) mendukung
Python dan beberapa pemrograman microworld berbasis Python lingkungan, seperti RUR-PLE17, Guido van Robot18, dan Turtlet19). ViLLE [22] juga menyediakan beberapa dukungan untuk memvisualisasikan eksekusi Python dengan fungsi tersebut untuk menerjemahkan animasi kode Java ke yang setara animasi dengan Python \cite{helminen2010jype}.

\subsection{IDE Pyton}
Pyton sendiri merupakan salah satu Language programming yang biasanya digunakan oleh para pengembang. Kebanyakan dari mereka mengembangkan aplikasi python hanya menggunakan editor yang ada maupun yang biasa. Namun tak sedikitpun para pengembang menggunakan Integrated Development Environment (IDE) untuk membantu mereka bekerja dengan proyek Python. Selain standar desktop Python IDE (integrated development environments), ada juga IDE berbasis IDEs, Sage, (dimaksudkan untuk mengembangkan ilmu pengetahuan dan program Python yang berhubungan dengan matematika), dan IDE yang di-hostingya \cite{van2007pythonss}.