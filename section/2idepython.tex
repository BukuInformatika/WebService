
%Resume Tentang IDE Python D4TI2A Kelompok 3
%\begin{enumerate}
%\item Aldi Maulana Yusuf                		1164001
%\item Dieni Hanifah               			1164008
%\item Ikrima Ningrumsari Mulyana    		1164013
%\item Muhammad Farid Alwan Assyifa         	1164017
%\item Rojasqi Fadilla              			1164026
%\end{enumerate}

\section{IDE Python}
Thonny adalah IDE Python baru untuk belajar dan mengajar pemrograman
yang dapat membuat visualisasi program menjadi bagian alami dari alur kerja pemula.
Di antara fitur-fiturnya yang menonjol adalah cara yang berbeda untuk melakukan langkah-langkah melalui kode, langkah demi langkah expression evaluation, visualisasi rinci dari tumpukan panggilan dan mode untuk menjelaskan konsep referensi dan heap. Ini mendukung penelitian pendidikan dengan mencatat tindakan pengguna untuk mengulang atau menganalisis proses pemrograman. Ini gratis untuk digunakan dan terbuka untuk ekstensi \cite{annamaa2015introducing}.

\subsection{IDE Pyton}
Python IDE merupakan program standar kemampuan konstruksi dan pelaksanaan, Python secara umum banyak pengembang profesional yang menggunakan nya 
lingkungan pengembangan terintegrasi (IDE) mendukung
Python dan beberapa pemrograman microworld berbasis Python lingkungan, seperti RUR-PLE17, Guido van Robot18, dan Turtlet19). ViLLE [22] juga menyediakan beberapa dukungan untuk memvisualisasikan eksekusi Python dengan fungsi tersebut untuk menerjemahkan animasi kode Java ke yang setara animasi dengan Python \cite{helminen2010jype}.

\subsection{IDE Pyton}
Pyton sendiri merupakan salah satu Language programming yang biasanya digunakan oleh para pengembang. Kebanyakan dari mereka mengembangkan aplikasi python hanya menggunakan editor yang ada maupun yang biasa. Namun tak sedikitpun para pengembang menggunakan Integrated Development Environment (IDE) untuk membantu mereka bekerja dengan proyek Python. Selain standar desktop Python IDE (integrated development environments), ada juga IDE berbasis IDEs, Sage, (dimaksudkan untuk mengembangkan ilmu pengetahuan dan program Python yang berhubungan dengan matematika), dan IDE yang di-hostingya \cite{van2007pythonss}.

 Python Pada dasarnya Python merupakan perangkat lunak yang secara default termasuk dalam suatu paket distribusi GNU/Linux. Untuk GNU/Linux distribusi Slackware menggunakan Python versi 2.4, biasanya terdapat pada CD I direktori /slackware/d. Toolkit yang digunakan untuk melakukan instalasi paket di Slackware adalah installpkg, berikut langkah instalasinya:
  # mount /mnt/cdrom 
  # cd /mnt/cdrom/slackware/d 
  # installpkg python-2.4.1-i486-1.tgz \cite{utamipemrograman}

\subsection{IDE Pyton}
Python adalah bahasa pemrograman multi-paradigma : pemrograman berorientasi obyek dan pemrograman terstruktursepenuhnya didukung, dan ada sejumlah fitur bahasa yang mendukung pemrograman fungsional dan pemrograman berorientasi aspek (termasuk oleh metaprogramming dan dengan metode sihir ).  Banyak paradigma lain didukung menggunakan ekstensi, termasuk desain dengan kontrak  dan pemrograman logika .
Python menggunakan pengetikan dinamis dan kombinasi penghitungan referensi dan pengumpul sampah pendeteksi siklus untuk manajemen memori . Fitur penting dari Python adalah resolusi nama dinamis (late binding), yang mengikat metode dan nama variabel selama eksekusi program.
Desain Python hanya menawarkan dukungan terbatas untuk pemrograman fungsional dalam tradisi Lisp . Bahasa memiliki fungsi map (), mengurangi () dan filter (), pemahaman untuk daftar, kamus, dan set, serta ekspresi generator. Perpustakaan standar memiliki dua modul (itertools dan functools) yang mengimplementasikan alat-alat fungsional yang dipinjam dari Haskell dan Standard ML . \cite{van2007python}

\subsection{IDE Pyton}
Python adalah bahasa pemrograman interpretatif multiguna dengan filosofi perancangan yang berfokus pada tingkat keterbacaan kode. Python diklaim sebagai bahasa yang menggabungkan kapabilitas, kemampuan, dengan sintaksis kode yang sangat jelas dan dilengkapi dengan fungsionalitas pustaka standar yang besar serta komprehensif. Python mendukung multi paradigma pemrograman, namun tidak dibatasi pada pemrograman berorientasi objek, pemrograman imperatif, dan pemrograman fungsional. Salah satu fitur yang tersedia pada python adalah sebagai bahasa  
pemrograman dinamis yang dilengkapi dengan manajemen memori otomatis. Seperti halnya pada bahasa pemrograman dinamis lainnya, python umumnya digunakan sebagai bahasa skrip meski pada praktiknya penggunaan bahasa ini lebih luas mencakup konteks pemanfaatan yang umumnya tidak dilakukan dengan menggunakan bahasa skrip. Python dapat digunakan untuk berbagai keperluan pengembangan perangkat lunak dan dapat berjalan di berbagai platform sistem operasi. \cite{rosmalasarana}.


