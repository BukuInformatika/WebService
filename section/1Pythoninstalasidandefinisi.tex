\documentclass[12pt, times new roman, a4paper]{article}
\begin{document}
\title{Python instalasi dan definisi dan contoh kode awal}
\maketitle

%\Kelompok 1
%\begin{enumerate}
%\Asep Setiawan                  	1164005
%\Ema Ainun Novia	              	1164010
%\Raymana Aprilyan                	1164023
%\Riandaka Rizal H.R.             	1164025
%\Sri Kurnia Ningsih				1164094
%\end{enumerate}
\section{Definisi Python}

Python merupakan salah satu bahasa pemrograman mutiguna dan juga model skrip (scripting language) yang berorientasi obyek. Bahasa pemrograman Python merupakan Bahasa pemrograman yang mudah untuk dipelajari. Python bagus bagi pemula yang belum atau baru memulai belajar coding. Python dapat digunakan untuk mempermudah berbagai keperluan pengembangan perangkat lunak dan dapat berjalan di berbagai platform sistem operasi.


\section{Definisi Python}

Python merupakan bahasa pemrograman yang freeware atau perangkat bebas, yang dimaksud perangkat bebas adalah tidak ada batasan dalam penyalinan maupun pendistribusian.  Lengkap dengan source codenya, debugger dan juga profiler, antarmuka yang terkandung di dalamnya untuk pelayanan antarmuka, fungsi sistem, GUI (antarmuka pengguna grafis), dan basis datanya.
Beberapa fitur yang dimiliki Bahasa pemrograman Python sebagai berikut :
\begin{enumerate}
  \item memiliki kepustakaan yang luas dalam arti distribusi Python telah disediakan pada modul-modul.
  \item memiliki tata bahasa yang jernih dan juga mudah untuk dipelajari bagi pengguna.
  \item memiliki aturan layout kode sumber yang memudahkan pengecekan, pembacaan kembali dan penulisan ulang kode sumber.
  \item dapat juga dibangun dengan bahasa pemrograman Python maupun C/C++.
\end{enumerate}

\section{Langkah-Langkah Menginstall (Bagian 1)}

Dalam instalasi kebutuhan library Python yang berguna untuk development aplikasi beserta pip. Serta akan dilakukan instalasi gcc dan openssh- server yang digunakan untuk SSH.
Fungsi-Fungsi yang dapat dijalankan pada melakukan instalasi library Python sebagai berikut:

\begin{enumerate}
\item Python-dev, merupakan instalasi Python development yang berfungsi untuk mengintegrasi modul eksternal Python.
  \item Python-setuptools, merupakan utilitas tambahan yang dapat mempermudah instalasi paket.
  \item Python-pip, Pip merupakan paket library tambahan pengganti untuk easy install.
  \item Mongodb, merupakan database yang akan digunakan untuk DBsistem.
  \item Gcc, sebagai compiler.
  \item Openssh-server, memiliki fungsi sebagai  protokol jaringan yang memungkinkan pertukaran data melalui jalur SSH.
\end{enumerate}

\section{langkah-Langkah Mengistall (Bagian 2)}

Tahapan selanjutnya adalah melakukan instalasi Library Python. Tapi sebelum Instalasi Library Python terlebih dahulu mengInstalas kebutuhan library Python yang berguna untuk development aplikasi.
Berikut ini adalah cara menginstal bahasa pemrograman Python yang berjalan pada sistem operasi Windows. Sebelum menginstal terlebih dahulu kita mendownload Python, setelah di download  . Pada folder download  disediakan 2 jenis file installer Python 3.6.5, untuk computer 32bit dan 64bit. Kita dapat memilih salah satu yang cocok buat sistem operasi kita.

\section{Langkah-Langkah Menginstall (Bagian 3)}
Maka setelah kita mendownload Pythonnya, berikut langkah langkah instalasinya :
\begin{enumerate}
  \item Pilih file “python-2.7.amd64”, tampil jendela “Python for Windows”, pilih “next”
  \item Pada jendela “Select Destination Directory”, pilih “next”
  \item Muncul jendela “Customize Python 2.7”, pilih “next”
  \item Muncul jendela progress bar untuk proses instalasinya, setelah itu akan keluar jendela “Completing the Python 2.7 Installer”, dan pilih “finish”
  \item Maka instalasi bahasa pemrograman Python sudah selesai dilakukan
\end{enumerate}

\section{Langkah-langkah Menginstall (Bagian 4)}
\begin{enumerate}
  \item Setelah proses instalasi selesai dilakukan, langkah berikutnya yang
dilakukan adalah melakukan “setting path”.
  \item “Setting path” dilakukan dengan membuka jendela “Properties” dengan klik
kanan pada “Computer”, dan pilih “Advanced System Setting”.
  \item Pada jendela “System Properties” dan tab “Advanced”, pilih “Environment
Variables”
  \item Di Jendela “Environment Variables” , klik 2x variable “Path” dan
jika belum ada Python27 maka tambahkan ketikan berikut ini: Python27, kemudian “ok”
\end{enumerate}


\section{Langkah-langkah Menginstall (bagian 5)}
Menjalankan python pertama kali pada tahap terakhir ini kita akan menjalankan python. Caranya sangat mudah, pada menu search windows, ketika “python” untuk mencari python. tetapi yang akan teman-teman klik bukan python melainkan Idle. yaitu tools untuk menjalankan python. kemudian akan terbuka ide dari python yang kita dapat gunakan untuk mempelajari python.

\section{Python Sebagai Alat Integrasi}
Salah satu kunci dari semua bahasa yang diterjemahkan modern adalah kemampuannya yang diperpanjang. Saya ingin meninjau secara singkat berbagai cara di mana kerangka kerja skrip berbasis Python dapat berintegrasi kode warisan dan diperluas dengan fungsionalitas baru. Ada tiga cara untuk menambahkan fungsi baru kerangka kerja dengan meng implementasikannya pada Python, caranya dengan "membungkus" atau dengan antarmuka kode yang ada yang memiliki beberapa kemampuan komunikasi.

\section{Menerapkan fungsionalitas dengan Python}
Metode ini adalah pilihan untuk semua perkembangan baru yang kita lakukan. Memang untuk kode yang belum ditulis sifat tingkat tinggi Python dan ekstensi yang sudah tersedia membuatnya jauh lebih mudah untuk menerapkan dengan Python daripada di C atau C ++. Ketika beberapa kode sudah ada, ada sejumlah kasus di mana kami memutuskan untuk menerapkan kembali dengan Python. Ini terjadi ketika coding ulang dengan Python membutuhkan relatif kecil jumlah kode (yang tidak biasa) dan kinerja kode Python diharapkan dapat diterima. Pendekatan ini memberikan keuntungan utama dari independensi platform.

\section{Membungkus kode C, C ++, dan Fortran yang sudah ada}
Ada banyak kasus di mana tidak masuk akal untuk mengimplementasikan kembali beberapa kode warisan dengan Python. Namun demikian
diinginkan untuk memiliki akses ke kode ini dari Python. Ini membutuhkan "membungkus" kode warisan untuk Python. Biarkan kami
perhatikan contoh sederhana berikut: Kami memiliki fungsi "foo" yang mengambil bilangan bulat sebagai argumen dan
mengembalikan pelampung.

\section{Dunia python}
Di dunia Python, bilangan bulat dan pelampung adalah obyek. Agar bisa memanggil fungsi foo
dari Python kita harus menulis fungsi dalam C yang akan: mengambil objek Python sebagai argumen, ekstrak bilangan bulat nilai dari objek ini, panggil fungsi foo untuk nilai integer ini, paket float yang dikembalikan ke Python objek yang mewakili float, dan akhirnya mengembalikan objek itu ke interpreter. Kode ini disebut "pembungkus kode". Setelah kita memiliki kode pembungkus kita dapat mengkompilasi bersama dengan fungsi foo dan membuat yang dibagikan objek (dll di dunia Windows) yang kemudian dapat diimpor ke interpreter Python. Tentu saja, ekstensi ini sekarang bergantung pada platform. Proses pembungkusan kode ini dapat diotomatisasi hingga tingkat yang wajar menggunakan SWI.

\section{Kode Interfacing}
Untuk kode pewarisan dirancang dengan beberapa kemampuan komunikasi, seperti melalui soket atau
menggunakan protokol komunikasi standar (HTTP, NNTP, SMTP, dll.) dukungan yang diberikan oleh standar
Modul Python untuk membuat protokol ini umumnya cukup mudah untuk menulis dengan Python.
Sehingga akhirnya, interpreter Python dapat disematkan dalam aplikasi sebagai bahasa tambahan. Selain menambahkan kemampuan scripting ke program di mana Interpreter Python tertanam, ini juga membuat semua alat alatnya ter-porting ke dalam bahasa Python di dalam programnya.

\end{document}

\section{Implementasi Python}
Python saat ini memiliki tiga implementasi kualitas produksi, yang dikenal sebagai Cpython, Jython dan Ironpython dan beberapa implementasi eksperimental lainnya, seperti PyPy. Didalam laporan ini terutama membahas Cpython, implementasi yang paling banyak digunakan, yang saya sebut hanya sebagai python untuk kesederhanaan. Namun perbedaan antara bahasa dan implementasinya adalah yang penting.

\section{Cpython}
Python klasik adalah implementasi python tercepat dan paling mutakhir(akurat). Oleh karena itu, dapat dianggap sebagai "implementasi referensi" dari bahasa. Cpython adalah compiler, interpreter, dan set modul ekstensi built-in dan opsional, semua dikodekan dalam standar C. Cpython dapat digunakan pada setiap platform di mana compiler C sesuai dengan standar ISO / IEC 9899: 1990.

\section{Jython}
Jython adalah implementasi python untuk Java Virtual Machine (JVM) yang kompatibel dengan java 1.2 atau lebih baik. Seperti JVM tersedia untuk semua platform modem yang populer. Dengan jython kita dapat menggunakan semua pustaka dan kerangka java. Untuk penggunaan jython yang optimal, kita membutuhkan kemewahan dengan kelas dasar java. Kita tidak harus kode di java, tetapi dokumentasi dan contoh untuk kelas java yang ada ditulis dalam istilah java, jadi Anda perlu kenalan dengan java untuk membaca dan memahaminya. Selain itu kita juga perlu menggunakan alat pendukung java untuk tugas-tugas seperti memanipulasi file .jar dan menandatangani applet.
\end{document}
