\documentclass[12pt, times new roman, a4paper]{article}
\begin{document}
\title{Python instalasi dan definisi dan contoh kode awal}
\maketitle

%\Kelompok 1
%\begin{enumerate}
%\Asep Saktiawan                  	1164001
%\Ema Ainun Novia	              	1164010
%\Raymana Aprilyan                	1164023
%\Riandaka Rizal H.R.             	1164025
%\Sri Kurnia Ningsih				1164094
%\end{enumerate}
\section{Definisi Python}

Python merupakan salah satu bahasa pemrograman mutiguna dan juga model skrip (scripting language) yang berorientasi obyek. Bahasa pemrograman Python merupakan Bahasa pemrograman yang mudah untuk dipelajari. Python bagus bagi pemula yang belum atau baru memulai belajar coding. Python dapat digunakan untuk mempermudah berbagai keperluan pengembangan perangkat lunak dan dapat berjalan di berbagai platform sistem operasi.
\end{document}

\section(Definisi Python)

Python merupakan bahasa pemrograman yang freeware atau perangkat bebas, yang dimaksud perangkat bebas adalah tidak ada batasan dalam penyalinan maupun pendistribusian.  Lengkap dengan source codenya, debugger dan juga profiler, antarmuka yang terkandung di dalamnya untuk pelayanan antarmuka, fungsi sistem, GUI (antarmuka pengguna grafis), dan basis datanya.
Beberapa fitur yang dimiliki Bahasa pemrograman Python sebagai berikut :
\begin{enumerate}
  \item memiliki kepustakaan yang luas dalam arti distribusi Python telah disediakan pada modul-modul.
  \item memiliki tata bahasa yang jernih dan juga mudah untuk dipelajari bagi pengguna.
  \item memiliki aturan layout kode sumber yang memudahkan pengecekan, pembacaan kembali dan penulisan ulang kode sumber.
  \item dapat juga dibangun dengan bahasa pemrograman Python maupun C/C++.
\end{enumerate}
\end(document)

