\documentclass[12pt]{journal}

\begin{document}

\title{Pengertian Web Service}
\maketitle

\section{Definisi}

\subsection{Hartati Deviana}
	Web service adalah konsep baru dalam sistem terdistribusi melalui Web yang menggunakan teknologi XML, dengan standar protokol  HTTP dan SOAP. Konsep teknologi Web service muncul untuk mendukung sistem terdistribusi yang memiliki infrastruktur yang berbeda. Karena Web service menggunakan XML, maka teknologi ini dapat mendukung integrasi berbagai platform sistem dan aplikasi, baik infrastruktur intranet dan ekstranet. Dalam penelitian ini akan disusun oleh sebuah sistem informasi dengan menggunakan teknologi Web service menggunakan PHP dan NuSOAP yang diimplementasikan pada sistem pengelolaan distribusi barang di sebuah apotek yang memiliki beberapa cabang. Penelitian ini menghasilkan sistem informasi yang mampu mengintegrasikan aplikasi dan platform dari seluruh cabang\cite{deviana2013penerapan}.

\subsection{Richards Robert}

Web service merupakan salah satu implementasi dari teknologi XML (Extensible Markup Language) pada proses pertukaran antara (data exchange) platform yang berbeda sercara berbeda.

"A Web service is a software system designed to support interoperable machine-to-machine interaction over a network. It has an interface described in a machine-processable format(specifically WSDL).Other systems interact with the Web service in a manner prescribed by its description using SOAP messages, typically conveyed using HTTP with an XML seriali zation in conjunction with other Web-related standards"\cite{ihya2011pembuatan}.

\end{document}

