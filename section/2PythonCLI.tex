\documentclass[12pt, times new roman, a4paper]{article}
\begin{document}
\title{Python instalasi dan definisi dan contoh kode awal}
\maketitle

%\Kelompok 1
%\begin{enumerate}
%\Asep Setiawan                  	1164005
%\Ema Ainun Novia	              	1164010
%\Raymana Aprilyan                	1164023
%\Riandaka Rizal H.R.             	1164025
%\Sri Kurnia Ningsih				1164094
%\end{enumerate}

\section{Definisi Command Line Interface (CLI)}
Command Line Interface (CLI) diatur oleh file konfigurasi, yang menentukan login pengguna, proyek awal, dataserver, Distributed Processing Unit (DPU) dan banyak opsi konfigurasi lainnya. CLI juga digunakan sebagai lingkungan pemrograman yang memungkinkan pengiriman program Python ke Astro-WISE, pengguna dapat menulis programnya sendiri melibatkan kelas dan perpustakaan Astro-WISE dan melaksanakannya melalui CLI.

\section{Definisi Command Line Interface (CLI)}
Astro-WISE CLI memberikan  akses ke semua Astro- Kelas dan juga  pustaka WISE yang memungkinkan untuk membangun programan  Bahasa Python. Yang kemudian CLI diatur oleh beberapa file konfigurasi, yang akan menentukan login pengguna, proyek awal, data server, Distributed Processing Unit (DPU) dan masih banyak opsi konfigurasi lainnya yang masih  sering digunakan.

\section{Definisi Command Line Interface (CLI)}
CLI juga digunakan sebagai lingkungan pemrograman yang memungkinkan pengiriman program Python ke Astro-WISE, pengguna dapat menulis program sendiri yang melibatkan kelas dan perpustakaan Astro-WISE dan mengeksekusinya melalui CLI. Ada versi berbasis web dari CLI, 8 meskipun tidak diberikan kinerja dan fungsionalitas penuh dari CLI lokal, itu dapat digunakan ketika pengguna tidak memiliki CLI lokal yang diinstal.

\section{Command Line Interface (CLI)}
Data yang dikelola oleh Lapisan Abstraksi Data dapat diakses melalui Command Line Interface (CLI) berdasarkan sintaks perintah sederhana yang mirip dengan gdb atau totalview. Interpreter untuk CLI didasarkan pada interpreter Python diperpanjang, yang memungkinkan pengguna untuk mengintegrasikan perintah aliran kontrol Python ke dalam skrip CLI. Pengguna dapat menggunakan CLI untuk mengakses Open | SpeedShop baik secara interaktif atau dalam mode batch.

\section{Command Line Interface}
Pengguna dapat menggunakan CLI untuk mengakses Open| SpeedShop baik secara interaktif atau dalam mode batch. Selain menggunakan CLI secara langsung, pengguna dapat mengakses Open|SpeedShop menggunakan modul Python atau GUI. Kedua antarmuka alternatif menggunakan CLI itu sendiri, yaitu, mereka menerjemahkan setiap input pengguna ke dalam perintah CLI yang sesuai dan menjalankannya melalui Lapisan CLI. Ini memiliki keunggulan tersendiri yang dimiliki semua pengguna antarmuka setara serta dapat dioperasikan.

\section{Python CLI}
Alat baris perintah Convertextract adalah perpustakaan Python berlisensi MIT yang dibangun dari garpu perpustakaan Textract Dean Malmgren. Textract merupakan perpustakaan yang mengekstrak teks dari berbagai format file yang berbeda. Convertextract melakukan daftar khusus untuk menemukan / mengganti transformasi pada teks sumber apa pun, dan menyimpan file konversi baru tanpa mengubah pemformatan gaya dokumen asli (ukuran font, garis bawah, keberanian dll).

\section{Python CLI}
Karena Convertextract mengharapkan untuk mengkonversi dari font 'diretas', itu akan memberikan teks yang dikonversi di Times New Roman secara default, meskipun font lain mungkin ditentukan. Di luar kotak, Convertextract saat ini mendukung tiga konversi: dari Heiltsuk Duolos, Heiltsuk Times, dan Tsilhqot’in Duolos ke Unicode. Convertextract juga mendukung konversi yang ditentukan pengguna yang dapat dijelaskan dalam dokumen Excel yang dilewatkan sebagai argumen ke Convertextract.

\section{Python CLI}
Urutan yang benar dari setiap substitusi sangat penting untuk menghasilkan output yang benar. Untuk mencegah urutan penggantian yang salah, mereka diperintahkan menurut panjangnya dari yang terpendek hingga yang terpendek. Perpustakaan Python dapat disebut baik dalam skrip Python atau langsung melalui baris perintah. Dokumentasi tentang cara menginstal dan menggunakan alat baris perintah tersedia di repositori publik3.

\section{Command Line Interface}
The Command Line Interface (CLI) merupakan komponen pusat fof Cyberaide Shell. Semua perintah yang masuk, baik langsung melalui CLI atau melalui layanan Web Shell Cyberaide ditafsirkan melalui CLI. Cyberaide Shell dibangun menggunakan Apache CLI 1.1. Kami telah mengimplementasikan peningkatan ke Apache CLI yang memudahkan untuk mengembangkan dan memanipulasi baris perintah dari dalam kode kami. Ini termasuk definisi kerangka kerja yang membutuhkan pembuatan argumen baris perintah pendek dan panjang serta dokumentasi wajib untuk semua perintah. Apache CLI didasarkan pada tiga tahap pemrosesan baris perintah: (a) definisi, (b) parsing, dan (c) interogasi.

\end{document} 