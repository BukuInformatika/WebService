\documentclass[12pt, times new roman, a4paper]{article}
\begin{document}
\title{Python instalasi dan definisi dan contoh kode awal}
\maketitle

%\Kelompok 1
%\begin{enumerate}
%\Asep Setiawan                  	1164005
%\Ema Ainun Novia	              	1164010
%\Raymana Aprilyan                	1164023
%\Riandaka Rizal H.R.             	1164025
%\Sri Kurnia Ningsih				1164094
%\end{enumerate}

\section{Definisi Command Line Interface (CLI)}
Command Line Interface (CLI) diatur oleh file konfigurasi, yang menentukan login pengguna, proyek awal, dataserver, Distributed Processing Unit (DPU) dan banyak opsi konfigurasi lainnya. CLI juga digunakan sebagai lingkungan pemrograman yang memungkinkan pengiriman program Python ke Astro-WISE, pengguna dapat menulis programnya sendiri melibatkan kelas dan perpustakaan Astro-WISE dan melaksanakannya melalui CLI.

\section{Definisi Command Line Interface (CLI)}
Astro-WISE CLI memberikan  akses ke semua Astro- Kelas dan juga  pustaka WISE yang memungkinkan untuk membangun programan  Bahasa Python. Yang kemudian CLI diatur oleh beberapa file konfigurasi, yang akan menentukan login pengguna, proyek awal, data server, Distributed Processing Unit (DPU) dan masih banyak opsi konfigurasi lainnya yang masih  sering digunakan.

\section{Definisi Command Line Interface (CLI)}
CLI juga digunakan sebagai lingkungan pemrograman yang memungkinkan pengiriman program Python ke Astro-WISE, pengguna dapat menulis program sendiri yang melibatkan kelas dan perpustakaan Astro-WISE dan mengeksekusinya melalui CLI. Ada versi berbasis web dari CLI, 8 meskipun tidak diberikan kinerja dan fungsionalitas penuh dari CLI lokal, itu dapat digunakan ketika pengguna tidak memiliki CLI lokal yang diinstal.

\end{document} 