\prologue{The sheer volumne of answers can often stifle insight...The purpose
of computing\index{computing!the purpose} is insight, not numbers.}
{Hamming}

\section{Definisi Web Service}
Web service adalah suatu sistem perangkat lunak yang dirancang untuk mendukung interoperabilitas dan interaksi antar sistem pada suatu jaringan. Web service digunakan sebagai suatu fasilitas yang disediakan oleh suatu web site untuk menyediakan layanan (dalam bentuk informasi) kepada sistem lain, sehingga sistem lain dapat berinteraksi dengan sistem tersebut melalui layanan-layanan (service) yang disediakan oleh suatu sistem yang menyediakan web service. Web service menyimpan data informasi dalam format XML, sehingga data ini dapat diakses oleh sistem lain walaupun berbeda platform, sistem operasi, maupun bahasa compiler.

Web service bertujuan untuk meningkatkan kolaborasi antar pemrogram dan perusahaan, yang memungkinkan sebuah fungsi di dalam Web Service dapat dipinjam oleh aplikasi lain tanpa perlu mengetahui detil pemrograman yang terdapat di dalamnya.

Beberapa alasan mengapa digunakannya web service  adalah sebagai berikut:

-Web service dapat digunakan untuk mentransformasikan satu atau beberapa bisnis logic atau class dan objek yang terpisah dalam satu ruang lingkup yang menjadi satu, sehingga tingkat keamanan dapat ditangani dengan baik. 

-Web service memiliki kemudahan dalam proses deployment-nya, karena tidak memerlukan registrasi khusus ke dalam suatu sistem operasi. Web service cukup di-upload ke web server dan siap diakses oleh pihak-pihak yang telah diberikan otorisasi.

-Web service berjalan di port  80 yang merupakan protokol standar HTTP, dengan demikian web service tidak memerlukan konfigurasi khusus di sisi firewall.\cite{awangga2017colenak}.

\section{Sejarah Web Service}
Penemu website adalah Sir Timothy John ¨Tim¨ Berners-Lee, sedangkan website yang tersambung dengan jaringan, pertamakali muncul pada tahun 1991. Maksud dari Tim ketika membuat website adalah untuk mempermudah tukar menukar dan memperbarui informasi kepada sesama peneliti di tempat dia bekerja. Pada tanggal 30 April 1993, CERN (tempat dimana Tim bekerja) menginformasikan bahwa WWW dapat digunakan secara gratis oleh semua orang. Sebuah website bisa berupa hasil kerja dari individu, atau menunjukkan kepemilikan dari sebuah organisasi, perusahaan, dan biasanya website itu menujukkan beberapa topik khusus, atau kepentingan tertentu. Sebuah website bisa berisi hyperlink yang menghubungkan ke website lain, jadi, kadang sulit membedakan antara website yang dibuat oleh individu perseorangan dengan website yang dibuat oleh organisasi bisnis bisa saja tidak kentara.

Website yang ditulis di konversi menjadi HTML oleh komputer dan diakses melalui web browser, yang dikenal juga dengan HTTP Client. Halaman web dapat dilihat atau diakses melalui jaringan komputer dan internet, perangkatnya bisa saja berupa komputer pribadi, laptop, PDA ataupun telepon selular.

Sebuah website dibuat didalam sebuah sistem komputer yang dikenal dengan server web, juga disebut HTTP Server, dan pengertian ini juga bisa menunjuk pada software yang dipakai untuk menjalankan sistem ini, yang kemudian menerima lalu mengirimkan halaman-halaman yang diperlukan untuk merespon permintaan dari pengguna. Apache adalah piranti lunak yang biasa digunakan dalam sebuah webserver, kemudian setelah itu adalah Microsoft Internet Information Services.



\subsection{Sir Timothy John ¨Tim¨ Berners-Lee}
Pada tanggal 30 April 1993, CERN (tempat dimana Tim bekerja) menginformasikan bahwa WWW dapat digunakan secara gratis oleh semua orang.

\begin{figure}[ht]

\centerline{\includegraphics[width=1\textwidth]{figures/petaduniaalid.JPG}}
\caption{Gambaran pengantar peta dunia karya al-Idrisi tahun 1154.}
\end{figure}

\begin{figure}[ht]
	\centerline{\includegraphics[width=1\textwidth]{figures/TabulaRogeriana.jpg}}
\vskip2pt
\caption{Tabula Rogeriana digambar oleh Al-Idrisi pada tahun 1154 untuk Raja Normandia Roger II dari Sisilia, setelah delapan menetap di istananya, di mana dia bekerja untuk penjelasan dan ilustrasi peta.}
\end{figure}


