%membuat dokumen
\documentclass{article}
\usepackage[indonesian]{babel} %untuk pemenggalan kata bahasa indonesia
\title{Definisi Dan Contoh CGI}
\author{Kelompok 2}
\date{4 April 2018}

\begin{document}
\maketitle
\section{Abstraksi}
Common Gateway Interface atau CGI adalah interface sederhana untuk menjalankan program external,perangkat lunak/software  atau gateway dibawah information server di dalam platform independen.saat ini information server telah didukung oleh HTTP Server
Antar muka ini telah digunakan oleh world wide web  atau www sejak 1993. Spesifikasinya didefinisikan sebagai parameter praktek terkini dari antarmuka CGI/1.1 yang dikembangkan dan didokumentasikan di Pusat Aplikasi supercomputer nasional AS.
\section{Tujuan}
Common Gateway Interface (CGI) mengijinkan HTTP server dan script CGI untuk berbagi tanggung jawab untuk menanggapi permintaan client. Permintaan client terdiri dari Uniform Resource Identifier (URI),  metode permintaan, dan berbagai informasi tambahan tentang permintaan yang disediakan oleh transport protocol.
CGI mendefinisikan parameter abstrak, dikenal sebagai meta-variabel, yang menggambarkan permintaan client. Bersama  dengan programmer interface konkrit, ini menentukan platform-independent interface antara script dan HTTP server.
Server bertanggung jawab untuk mengelola koneksi, transfer data, transportasi, dan masalah jaringan yang terkait dengan permintaan klien, sedangkan skrip CGI menangani masalah aplikasi, seperti pemrosesan pand dokumen akses data.
\section{Persyaratan}
Suatu implementasi tidak sesuai jika gagal memenuhi satu atau lebih persyaratan untuk protokol yang diimplementasikannya. Suatu implementasi yang memenuhi semua keharusan dan semua persyaratan untuk fitur-fiturnya dikatakan tanpa syarat sesuai. Salah satu yang memenuhi semua persyaratan tetapi tidak semua persyaratan untuk fitur-fiturnya dikatakan bersyarat secara kondisional.
\section{Spesifikasi}
Tidak semua fungsi dan fitur CGI didefinisikan di bagian utama dari spesifikasi ini. Frase berikut digunakan untuk menggambarkan fitur yang tidak ditentukan.
Fitur ini mungkin berbeda antara sistem, tetapi harus sama untuk implementasi yang berbeda menggunakan sistem yang sama. Suatu sistem biasanya akan mengidentifikasi kelas sistem operasi. Beberapa sistem didefinisikan di bagian 7 dokumen ini. Sistem baru dapat ditentukan oleh spesifikasi baru tanpa revisi  dokumen ini.
\section{Invoking the Script}
Tanggung Jawab Server
Server bertindak sebagai gateway aplikasi. Ia menerima permintaan dari klien, memilih skrip CGI untuk menangani permintaan, mengubah permintaan klien menjadi permintaan CGI, mengeksekusi skrip dan mengubah respon CGI menjadi respons untuk klien. Saat memproses permintaan klien, itu bertanggung jawab untuk mengimplementasikan protokol atau otentikasi dan keamanan tingkat transportasi. Server juga dapat berfungsi dengan cara yang tidak transparan, memodifikasi permintaan atau tanggapan untuk menyediakan beberapa layanan tambahan, seperti transformasi jenis media atau pengurangan protocol.
Server harus menerjemahkan dan mengkonversikan protocol di request data client yang dibutuhkan pada spesifikasi ini. Selanjutnya, server mempertahankan responsibilitynya kepada client untuk menyesuaikan dengan protocol network yang relevanbahkan ketika skrip CGI (Common Gateway Intrface) gagal menyesuaikan spesefikasi ini.
Jika server menerapkan authentication untuk permintaan, maka jangan jalankan skrip kecuali permintaan melewati  yang telah ditentukan oleh access controls.
\section{Script Selection}
Server menentukan CGI mana yang akan dieksekusi berdasarkan URI bentuk generik yang disediakan oleh klien. URI ini mencakup jalur hierarkis dengan komponen yang dipisahkan oleh "/". Untuk setiap permintaan tertentu, server akan mengidentifikasi semua atau bagian utama dari jalur ini dengan skrip individu, sehingga menempatkan skrip pada titik tertentu dalam hierarki jalur. Sisa jalur, jika ada, pengidentifikasi sumber daya atau sub-sumber untuk ditafsirkan oleh skrip.
Informasi mengenai bagian ini teresedia untuk skrip dalam variabel-variabel meta,di deskripsikan di bawah.Support atau dukungan untuk non hirarki skema URI diluar ruang lingkup spesifikasi ini
\section{Script-URI}
Pemetaan dari URI yg di minta klien ke skrip yang dipilih, ditentukan oleh
implementasi server tertentu dan konfigurasinya.Server mungkin untuk mengijinkan script untuk di identifikasi dengan beberapa bagian URI hirarki yang berbeda,dan karna itu pula maka diizinkan untuk mengganti URI oleh anggota lain dalam satu set atau bagian selama pemrosesan variabel variabel meta.Server :
\begin{enumerate}
\item Mungkin untuk mempertahankan URI dalam permintaan atau request klien tertentu,atau
\item Memungkinkan untuk memilih URI Kanonis dari kumpulan kemungkinan nilai untuk setiap skrip,atau
\item Dapat mengimplementasikan pilihan URI lain dari set.
\end{enumerate}
Dari meta-variabel yang dihasilkan, sebuah URI, 'Script-URI', dapat
dibangun. Ini HARUS memiliki properti yang jika klien dapat
mengakses URI ini, maka skrip akan dieksekusi
dengan nilai yang sama untuk SCRIPT_NAME, PATH_INFO, dan QUERY_STRING
meta-variabel. Script-URI memiliki struktur generic URI sebagai yang
didefinisikan dalam bagian 3 dari RFC 2396 [2], dengan pengecualian objek itu
parameter dan pengidentifikasi fragmen tidak diizinkan. Beragam
komponen dari Script-URI didefinisikan oleh beberapa
meta-variabel (lihat di bawah);

	script-URI = <scheme> "://" <server-name> ":" <server-port>
             	 <script-path> <extra-path> "?" <query-string>

dimana <scheme> didapatkan dari SERVER_PROTOCOL, <server-name>,
<server-port> dan <query-string> adalah nilai dari masing-masing meta-variabel.  The SCRIPT_NAME dan nilai-nilai PATH_INFO, URL-dikodekan dengan ";", "=" dan "?"  reserved, diberikan  <script-path> dan <extra-path>.

Skema dan protocol yang ada tidak identik karena skema mengidentifikasi metode akses selain protocol aplikasi.Sebagai contoh, sumber daya yang diakses menggunakan Transport Layer Security (TLS) akan memiliki permintaan URI dengan skema https ketika menggunakan HTTP protocol. CGI tidak menyediakan sarana generic kepada script untuk merekonstruksi ini, oleh karena itu Script-URI seperti yang didefinisikan meliputi protokol dasar yang digunakan. Tetpi, sebuah script memungkinkan dapat menggunakan skema khusus meta-variabel untuk lebih baik menyimpulkan skema URI.

Perhatikan bahwa definisi ini juga memungkinkan URI yang akan dikonstruksilah yang akan memanggil skrip dengan nilai yang diizinkan untuk info jalur/bagian atau query-string, dengan memodifikasi komponen yang sesuai
\section{Eksekusi}
Skrip ini dijalankan dengan cara yang ditentukan system.
Kecuali jika tidak ditentukan , file yang berisi skrip akan dipanggil sebagai program yang dapat dijalankan
\end{document}  