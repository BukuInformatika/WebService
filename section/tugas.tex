\section{Definisi Frontend}
frontend bisa disebut tampilan utama dari sebuah website pada frontend biasanya ditampilkan beberapa konten-konten yang bisa diakses oleh pengguna atau user yang menggunakan website tersebut. frontend juga berfungsi untuk user interace dari setiap web site. Biasanya frontend hanya menampilkan fungsi fungsi dari kontent sebuah web site seperti fungsi sebuah tombol untuk mengirim berkas atau untuk menampilkan konten konten yang lainnya dalam website tersebut.

\section{Fungsi Front-end}
Fungsi ini berhubungan langsung dengan pengguna dan berperan penting dalam keseluruhan proses bisnis dalam hal menghubungkan 
back-end dengan pengguna. layanan depan (front-end) bertugas mempresentasikan apa yang sudah dikerjakan oleh back-end
dan menjadi sarana bagi pengguna untuk mendapatkan segala sesuatu yang disediakan dibagian fungsi back-end. Peningkatan fungsi layanan depan yang baik akan mampu meningkatkan kepuasan pengguna.

\section{The front-end embodiment is a GUI ( Graphical User Interface )}
Frontend provides the user interface (user interface) that is easy to use. The existence of the frontend there is two kinds of CLI (Command Line Interface) and GUI (Graphical User Interface). The application of GUI on mobile devices can be exemplified in client-server applications, where users access the data contained on the web server. The data will be stored in the database web server, so if the search data, then the requested or desired data will be searched on the database server (database server) which will then be sent to the client who has requested data.