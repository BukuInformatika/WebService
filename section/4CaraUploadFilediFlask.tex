\documentclass[12pt,a4paper]{article}
\usepackage[left=3.00cm, right=2.00cm, bottom=2.00cm, top=3.00cm]{geometry}
\linespread{1.5}
\begin{document}
\title{definisi Dekorator, Contoh Kode dan Fungsi}
\maketitle

\begin{itemize}

\item
NAMA KELOMPOK 4\\
Ajis Trigunawan			1164031\\
Alimu Dzul Ikroom		1164032\\
Muhammad Hanafi			1164092\\
Riki Karnovi			1164052\\
Yoga Sakti Hadi P		1164059\\

\end{itemize}

\section{Cara Upload File di Flask}

\subsection{Pengantar Flask}

Flask framework didasarkan pada Werkzeug, dan Jinja 2, dan intensitas yang baik. Kerangka ini tidak memiliki dependensi terpisah dari Perpustakaan Standar Python. Labu tidak termasuk komponen yang membutuhkan pihak ketiga dukungan seperti memvalidasi formulir atau menyediakan sarana komunikasi antara aplikasi dan database. Namun, fitur tersebut dapat ditambahkan menggunakan ekstensi. Layanan yang ditawarkan oleh kerangka ini termasuk server HTTP built-in, dukungan untuk pengujian unit, dan Layanan web RESTful. Aplikasi dibangun menggunakan ini kerangka kerja adalah minitwit, flaskr, flask.pocoo.org dll.

\end{document}