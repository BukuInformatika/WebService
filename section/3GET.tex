%Resume GET (parameter GET, cara penggunaan dan kode) Kelompok 3 D4TI2B
%\begin{enumerate}
%\Fikri aldi nugraha                  1164038
%\Nur Arkhamia Batubara               1164049 
%\Miftahul Hasanah                    1164046 
%\Si Made Angga Dwitya P              1164053 
%\Widary Anggraini Mindo V Siahaan    1164057
%\end{enumerate}

\section{Pengenalan Method GET Pada HTTP}
Http adalah protokol permintaan-jawaban (request-reply). Client akan mengawali koneksi dengan server dengan mengirimkan permintaan dengan nama dokumen yang client inginkan dan kemudian server mengirimkannya kembali dengan normal, termasuk dokumen yang diminta. http juga mengizinkan client untuk mengirimkan data yang diminta pengguna ke server. Http juga mengizinkan client untuk mengirimkan data yang diminta pengguna ke server. 

HTTP mendefinisikan seperangkat metode permintaan untuk menunjukkan tindakan yang diinginkan yang akan dilakukan untuk sumber daya 
tertentu.
Meskipun mereka juga bisa menjadi kata benda, metode permintaan ini kadang-kadang disebut sebagai verba HTTP. Masing-masing menerapkan 
semantik yang berbeda, namun beberapa fitur umum digunakan bersama oleh mereka: mis. Metode permintaan dapat berupa safe, idempotent, atau cacheable. 
Salah satu metode permintaan yang digunakan dalam Http adalah GET, dimana GET ini digunakan untuk meminta representasi sumber atau 
menampilkan data/nilai pada url yang nantinya akan ditampung oleh action.

REST menggunakan protokol HTTP yang bersifat stateless. Perintah HTTP yang bisa
digunakan adalah fungsi GET, POST, PUT atau DELETE. Hasil yang dikirimkan dari server
biasanya dalam bentuk format XML atau JSON sederhana tanpa ada protokol pemaketan data,
sehingga informasi yang diterima dapat jauh lebih mudah dibaca dan diparsing disisi client.

\section{Pengertian Method Get}
GET adalah operasi read-only yang sering digunakan untuk meminta informasi yang spesifik pada sebuah server dalam bentuk query. 
Karakteristik dari operasi GET adalah idempotent dan safe. Idempontent dapat diartikan sebanyak-banyaknya apapun operasi ini dilakukan 
dan 
hasilnya akan tgetap sama sedangkan, safe berarti ketika operasi ini diinvokasi tetap tidak mengubah state di server.
Biasanya GET sering ditemukan di HTML,PHP dah diterapkan juga pada Web Service.

Dalam Bahasa PHP melakukan HTTP request ke halaman kemudian secara default adalah get request, data yang didapatkan dari web service 
dikirimkan dalam bentuk format standar misalnya XML atau JSON atau Java Object Notation. Get memiliki fungsi yang sama seperti POST 
digunakan untuk mengirimkan nilai atau value variabel ke file yang telah diatur. Perbedaan method GET dan method POST sangat kecil 
tetapi 
sangat terlihat dengan jelas.

Metode umum yang dapat digunakan oleh HTTP untuk dapat membaca data kedalam sebuha browser adalah dengan menggunakan metode get. 
Metodeget ini dirancang hanya dapat kita gunakan untuk membaca sebuah data namun dapat disayangkan secara praktek dapat digunakan juga untuk melewatkan data dengan menambahkan informasi pada sebuah URL. Maka dengan cara tersebut makametode get akan melewatkan data yang dapat dilihat oleh pengguna melalui alamat URL yang juga dapat kita simpan pada bookmark.

Metode GET pada HTTP/1.0 mempunyai batas maksimum parameter data sepanjang batas URL dengan batas ukuran maksimum 2 kb untuk browser 
saat in. HTTP/1.1 tidak memberikan batas maksimum untuk Uniform Resource Identifier (URI). Penyebutan yang lebih umum untuk URL. Metode 
lain yang dapat kita gunakan agar kita dapat  mengirimkan data yaitu kita dapat menggunakan  metode POST. Dimana metode  POST ini 
memiliki banyak kelebihan dibandingkan dengan  metode GET ,kelebihan yang dimaksud tersebut yaitu metode POST  memliki panjang parameter 
dan tidak  terbatas,  metode POST tidak dapat terlihat oleh pengguna  seperti manusia dan  kita juga tidak bias menyimpannya ke dalam 
bookmark.

\subsection{Karateristik dari Method GET}
Metode get juga dapat diartikan sebagai metode pengiriman data dengan menggunakan query berupa string, sehingga seluruh nilai dari form 
diitampilkan pada baris url/Address bar. Get juga dapat difungsikan sebagai penamaan (link) dalam sebuah website. Adapun beberapa 
karakteristiknya yang dimiliki oleh Get sehingga method ini bisa memudahkan user untuk melakukan pencarian data ataupun penamaan pada 
url yaitu sebagai berikut :
\begin{enumerate}
\item Variabel dapat terlihat pada url.
\item Dibatasi pada panjang string yaitu 2047 karakter.
\item Dapat memungkinkan pengunjung dapat langsung memasukkan nilai pada form variabel proses.
\end{enumerate}


