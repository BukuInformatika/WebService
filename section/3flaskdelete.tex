\section{Pengenalan Flask Delete}
Flask adalah sebuah framework web yang ada pada Python, dimana walaupun flask relatif kecil namun tangguh, flask sendiri dibuat dengan ide menyederhanakan inti framework seminimal mungkin.  Flask sendiri lebih mudah dipelajari dan digunakan, membuat para penggunanya bisa membangun dan mengembangkan aplikasi web dalam rentan waktu yang relatif singkat dan cepat. Flask bisa digunakan  untuk membuat website rumit yang berbasis database, memulai dengan membuat halaman statik berguna untuk memperkenalkan alur kerja, yang nantinya digunakan untuk membuat halaman lain yang lebih rumit.

Flask delete merupakan method dari Database pada Flask dimana berfungsi untuk menghapus elemen database itu sendiri apabila diperlukan. Flask Delete juga merupakan
sebuah method dari service pada rest.

contoh flask delete :

@app.route('/delete_post/<int:post_id>')
def delete_post(post_id):
    posts = models.Post.delete_instance().where(models.Post.id == post_id)
    if posts.count() == 0:
        abort(404)
    return redirect(url_for('index'))


<h2>
    <a href="{{ url_for('delete_post', post_id=post.id) }}">Delete</a>
</h2>

\section{Deleting Values}
Aplikasi dapat memaksa penggusuran nilai dengan menghapus kuncinya. penghapusan langsung dan atom.
Di python, Anda meneruskan kunci ke fungsi delete () atau metode. Ini mengembalikan satu dari tiga nilai: memchace.DELETE_SUCCESSFUL jika kunci ada dan berhasil dihapus, memcache.DELETE_ITEM_MISSING jika tidak ada nilai dengan kunci yang diberikan, atau memcache.DELETE_NETWORK_FAILURE jika penghapusan tidak dapat diselesaikan karena kegagalan layanan. Konstanta ini didefinisikan sedemikian rupa sehingga jika Anda tidak peduli tentang perbedaan antara penghapusan yang berhasil dan kunci yang hilang, Anda dapat menggunakan hasilnya sebagai ekspresi bersyarat. (DELETE_NETWORK_FAILURE adalah 0)\cite{sanderson2015programming}.

		succes = memcache.delete(key)
		if not succes;
			#ada masalah saat mengakses memcache

DELETE adalah metode yang digunakan untuk menghapus sumber daya yang diidentifikasi oleh URI yang diberikan. Ini melakukan operasi DELETE\cite{alemu2014rest}.
SQLAlchemy dalam Flask dan Section di Flask. Anda dapat menggunakan keduanya. Database, atau SQLAlchemy, akan menyimpan kartu tanpa batas hingga Anda menghapusnya. Sesi akan menyimpan kartu di sisi klien sampai mereka menghapusnya, baik browser ditutup atau klien secara manual menghapus sesi.berikut merupakan contoh codenya.

from flask import session
import random

session['user cards'] = []
# Default cards = empty list. Do the above to reset.

def deal_card():
  card_drawn = random.choice(CARDS)
  session['user cards'].append(card_drawn)
  # The "CARDS" variable is not here but it's just a list of cards.
  CARDS.pop(CARDS.index(card_drawn))

def print_cards():
  for card in session['user cards']:
    print(card)

Didalam flask juga terdapat method yang berfungsi untuk mghapus yaitu method delete(). Kebanyakan method delete() ini sering digunakan untuk menghapus session (sesi) yang terdapat di database atau sebagai perbandingan nilai yang ada di database.contoh berikut menghapus peran "moderator" dari database
>>> db.section.delete(mode_role)
>>> db.section.commit()
perhatikan bahwa penghapusan. seperti sisipan dan pembaruan, dijalankan hanya ketika sesi basis data dilakukan\cite{grinberg2018flask}.

