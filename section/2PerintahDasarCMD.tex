%membuat dokumen
\documentclass{article}
\usepackage[indonesian]{babel} %untuk pemenggalan kata bahasa indonesia
\title{Perintah Dasar CMD}
\author{Kelompok 2}
\date{23 Mei 2018}

\begin{document}
\maketitle
\section{Abstraksi}
Command prompt atau (CMD) adalah sebuah perintah DOS (Disk Operating System)  yang tersedia dan ter install di windows yang dapat di akses oleh kita secara offline atau online.CMD merupakan sebuah fitur yang sangat berfungsi dan berguna, menggunakan CMD kita bisa mengakses, me rename, menghapus atau bahkan memindahkan sebuah file dengan sangat mudah.
Dan mungkin untuk sebagian orang masih buta akan fitur CMD ini. Yang mungkin biasa kita kenal dengan Command Prompt atau DOS prompt merupakan sebuah command line (baris perintah) yang digunakan pada sebuah OS (OperatingSystem) berbasis GUI untuk mengeksekusi sebuah file dengan cara menuliskan perintahnya pada jendela cmd atau singkatnya Command Prompt ini merupakan sistem operasi berbasis beberapa baris perintah.
DOS Prompt atau yang lebih dikenal dengan Command Prompt ini pada awalnya digunakan sebagai sistem operasi. Tetapi setelah DOS mulai banyak ditinggalkan dan tidak digunakan lagi, maka sekarang DOS tetap diintegrasikan dan digunakan oleh Microsoft pada Windows yang lebih kita kenal dengan kata MS-DOS atau Command Prompt. Dan yang akan ditulis di sini akan lebih ke pengenalan pada perintah-perintah dasar dari Command Prompt.
\section{Perintah Dasar CMD}
Dibawah ini merupakan perintah dasar yang harus difahami, karena perintah dasar inilah yang nantinya akan mengantarkan kita pada kemudahan dalam mempelajari DOS prompt lebih lanjut.
• ADDUSERS : Perintah untuk menambah  daftar pengguna untuk atau dari CSV file
• ARP : Address Resolution Protocol
• Assoc : Ubah ekstensi file  asosiasi
• ASSOCIAT : Perintah yang digunakan untuk menjalankan asosiasi file
• Attrib : Perintah yang digunakan untuk mengubah atribut file
• Bootcfg : Edit Windows boot settings
• BROWSTAT : untuk menapatkan domain, info browser dan juga PDC
• CACLS : Perintah untuk mengubah file permissions
• CALL : untuk memanggil satu program batch yang lain
• CD : perintah untuk mengganti Directory – pindah ke Folder tertentu
• Change : untuk menganti Terminal Server Session properties
• CHKDSK : Check Disk yaitu untuk memeriksa dan memperbaiki masalah disk
• CHKNTFS : untuk memeriksa sistem file NTFS
• CHOICE : Menerima input dari keyboard ke file batch
• CIPHER : Encrypt atau Decrypt file atau folder
• CleanMgr : Dengan ototmatis membersihkan Temperatur file, recycle bin
• CLEARMEM : Menghapus kebocoran memori
• CLIP : Menyalin Standard Input ke Windows clipboard
• CLS : Menghapus layar (Clear The Screen)
• CLUSTER : Windows Clustering
• CMD : Memulai shell CMD baru
• COLOR : Mengubah warna dari jendela CMD
• COMP : Membandingkan isi dari dua file atau set file
• COMPACT : Mengcompress file atau folder pada partisi NTFS(New Technology File System)

