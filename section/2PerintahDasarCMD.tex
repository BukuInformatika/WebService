\documentclass{article}
\usepackage[indonesian]{babel}
\title{Perintah Dasar CMD}
\author{Kelompok 2}
\date{23 Mei 2018}
\begin{document}
\maketitle

\section{Abstraksi}
Command prompt atau (CMD) adalah sebuah perintah DOS (Disk Operating System)  yang tersedia dan ter install di windows yang dapat di akses oleh kita secara offline atau online.CMD merupakan sebuah fitur yang sangat berfungsi dan berguna, menggunakan CMD kita bisa mengakses, me rename, menghapus atau bahkan memindahkan sebuah file dengan sangat mudah.
Dan mungkin untuk sebagian orang masih buta akan fitur CMD ini. Yang mungkin biasa kita kenal dengan Command Prompt atau DOS prompt merupakan sebuah command line (baris perintah) yang digunakan pada sebuah OS (OperatingSystem) berbasis GUI untuk mengeksekusi sebuah file dengan cara menuliskan perintahnya pada jendela cmd atau singkatnya Command Prompt ini merupakan sistem operasi berbasis beberapa baris perintah.
DOS Prompt atau yang lebih dikenal dengan Command Prompt ini pada awalnya digunakan sebagai sistem operasi. Tetapi setelah DOS mulai banyak ditinggalkan dan tidak digunakan lagi, maka sekarang DOS tetap diintegrasikan dan digunakan oleh Microsoft pada Windows yang lebih kita kenal dengan kata MS-DOS atau Command Prompt. Dan yang akan ditulis di sini akan lebih ke pengenalan pada perintah-perintah dasar dari Command Prompt.
\section{Perintah Dasar CMD}
Dibawah ini merupakan perintah dasar yang harus difahami, karena perintah dasar inilah yang nantinya akan mengantarkan kita pada kemudahan dalam mempelajari DOS prompt lebih lanjut.
• ADDUSERS : Perintah untuk menambah  daftar pengguna untuk atau dari CSV file
• ARP : Address Resolution Protocol
• Assoc : Ubah ekstensi file  asosiasi
• ASSOCIAT : Perintah yang digunakan untuk menjalankan asosiasi file
• Attrib : Perintah yang digunakan untuk mengubah atribut file
• Bootcfg : Edit Windows boot settings
• BROWSTAT : untuk menapatkan domain, info browser dan juga PDC
• CACLS : Perintah untuk mengubah file permissions
• CALL : untuk memanggil satu program batch yang lain
• CD : perintah untuk mengganti Directory – pindah ke Folder tertentu
• Change : untuk menganti Terminal Server Session properties
• CHKDSK : Check Disk yaitu untuk memeriksa dan memperbaiki masalah disk
• CHKNTFS : untuk memeriksa sistem file NTFS
• CHOICE : Menerima input dari keyboard ke file batch
• CIPHER : Encrypt atau Decrypt file atau folder
• CleanMgr : Dengan ototmatis membersihkan Temperatur file, recycle bin
• CLEARMEM : Menghapus kebocoran memori
• CLIP : Menyalin Standard Input ke Windows clipboard
• CLS : Menghapus layar (Clear The Screen)
• CLUSTER : Windows Clustering
• CMD : Memulai shell CMD baru
• COLOR : Mengubah warna dari jendela CMD
• COMP : Membandingkan isi dari dua file atau set file
• COMPACT : Mengcompress file atau folder pada partisi NTFS(New Technology File System)
• Compress : Pada partisi NTFS dilakukan compress tunggal file
• CON2PRT : Untuk menghubungkan atau memutuskan sambungan computer dengan Printer
• CONVERT : Mengkonversikan FAT ke dalam drive NTFS
• COPY : Menyalin atau mengcopy salah satu file atau lebih ke lokasi lain
• CSCcmd : Clien-side cachin atau file offline
• CSVDE : Mengimpor dan Ekspor Active Directory data yang ada
• DATE : Berfungsi untuk pengaturan pada tanggal atau display
• Defrag : Perintah untuk mendefragmen hard drive
• DEL : Perintah untuk menghapus satu atau lebih file
• DELPROF : Untuk menghapus  profil user NT
• DELTREE : Menghapus folder dan  subfolder
• DevCon : Device Manager Command Line Utility
• DIR : Menampilkan daftar file dan folder
• DIRUSE : Menampilkan daftar disk yang terpakai
• DISKCOMP : Bandingkan  isi dua buah floppy disk
• Diskcopy : Menyalin isi dari satu disket atau direktori ke disket atau direktori yang lain
• DISKPART : Menghapus partisi pada harddisk
• DNSSTAT : Mengecek Statistik DNS
• DOSKEY : Mengedit baris perintah dan membuat macro
• DSADD : Menambah User (pada komputer) ke direktori aktif
• DSQUERY : Melihat daftar item dalam direktori aktif
• DSMOD : Mengubah user (pada komputer) di direktori aktif
• DSRM : Hapus item dari Active Directory
• ECHO : Untuk menampilkan pesan di layar
• ENDLOCAL : Akhir localisation  perubahan lingkungan dalam file batch
• ERASE : Untuk menghapus satu atau lebih file
• EVENTCREATE : Tambahkan pesan ke Windows event log
• EXIT : Keluar dari skrip arus atau rutin dan menetapkan error level
• EXPAND : Untuk uncompress file
• FC : Untuk membandingkan dua file
• FIND : Untuk mencari string teks dalam sebuah file
• FINDSTR : Untuk mencari file berdasarkan potongan kata
• FOR / F : Untuk pengulangan perintah terhadap satu set file
• FOR / F : Untuk pengulangan perintah terhadap hasil perintah lain
• FOR : Untuk pengulangan perintah terhadap semua options Files, Directory, List
• FORFILES : Memproses sekumpulan beberapa file
• FORMAT : Untuk memformat disk
• FREEDISK : Memeriksa free disk space/disk yang tersisa (dalam bytes)
• FSUTIL :  Melakukan pekerjaan yang berhubungan dengan sistem fi le FAT dan NTFS, seperti meng-query atau mengubah atribut fi le dan disk, atau memperbesar  volume utilitas
• FTP : Mentransfer file  protocol
• FTYPE : Menampilkan atau memodifikasi jenis-jenis file yang digunakan untuk  asosiasi ekstensi file
• GLOBAL : Mendisplay/menata keanggotaan kelompok global
• GOTO : Direct a batch program untuk melompat ke baris berlabel
• GPUPDATE : Update pengaturan Kebijakan Grup
• HELP : Online Help
• ICACLS : Ubah file dan folder permissions
• IF : kondisional melakukan perintah
• IFMEMBER : Apakah pengguna saat ini dalam sebuah NT Workgroup
• IPCONFIG : Configure IP
• KILL : Remove program dari memori
• LABEL : Mengedit label disk
• LOCAL : Menampilkan keanggotaan kelompok-kelompok lokal
• LOGEVENT : Menulis teks tulisan ke NT event viewer
• Logoff : Melakukan user log off
• LOGTIME : Log waktu dan tanggal di dalam file
• MAPISEND : Mengirim email dari baris perintah
• MBSAcli : Baseline Security Analyze
• MEM : Menampilkan penggunaan memori
• MD : Untuk membuat folder baru
• MKLINK : Membuat link simbolik (linkd)
• MODE : Untuk mengkonfigurasi perangkat system
• MORE : Display output, satu layar pada satu waktu
• MOUNTVOL : Untuk mengelola volume mount point
• MOVE : Memindahkan file dari satu folder ke yang lain
• MOVEUSER : Memindahkan pengguna dari satu domain ke domain lainnya
• MSG : mengirim pesan atau message
• MSIEXEC : Untuk memulai Microsoft Windows Installer
• MSINFO : Membuka informasi atau Windows NT diagnostics
• MSTSC : Terminal Server Connection (Remote Desktop Protocol)
• MUNGE : Mencari dan mengganti teks dalam file (s)
• MV : Mengcopy in-menggunakan file
• NET : Mengelola sumber daya jaringan
• NETDOM : Domain Manager
• Netsh : Configure Network Interfaces, Windows Firewall & Remote akses
• NETSVC : Command-line Service Controller
• NBTSTAT : Menampilkan statistik jaringan (NetBIOS over TCP / IP)
• NETSTAT : Mendisplay networking statistics (TCP / IP)
• NOW : Menampilkan  Tanggal dan Waktu saat ini
• NSLOOKUP : Untuk mengetahui IP dari sebuah domain.
• NTBACKUP : Untuk membackup  folder ke tape
• NTRIGHTS : Mengedit hak user account (wilayah akses yg diizinkan oleh admin)
• PATH : Menampilkan atau menetapkan path pencarian untuk file executable files
• PATHPING : Memeriksa jaringan dan Internet Data Path
• PAUSE : Men-suspend pengolahan file batch dan menampilkan pesan
• PERMS : Menampilkan izin untuk pengguna
• PERFMON : Menampilkan kinerja Monitor
• PING : Melakukan tes koneksi jaringan
• POPD : Mengembalikan nilai sebelumnya dari direktori sekarang yang di save oleh PUSHD
• PORTQRY : Menampilkan status ports dan services
• Powercfg : Untuk mengkonfigurasi pengaturan daya
• PRINT : Untuk mencetak file teks
• PRNCNFG : Display, untuk mengkonfigurasi atau mengubah nama printer
• PRNMNGR : Tambah, menghapus, daftar printer dan menetapkan printer default
• PROMPT : Untuk mengubah command prompt
• PsExec : Untu melakukan Proses Execute jarak jauh
• PsFile : Untuk menampilkan file dibuka dari jarak jauh (remote)
• PsGetSid : Untuk menampilkan SID sebuah komputer atau pengguna
• PsInfo : Untuk menampilkan daftar informasi tentang system
• PsKill : Untuk mematikan proses berdasarkan nama atau ID proses
• PsList : Untuk menampilkan daftar informasi rinci tentang proses-proses
• PsLoggedOn : Untuk melihat siapa saja yang log on (lokal atau melalui resource sharing)
• PsLogList : Menampilkan catatan kejadian log
• PsPasswd : Untuk mengubah sandi account
• PsService : Untuk melihat dan mengatur layanan
• PsShutdown : Shutdown atau reboot computer
• PsService : Melihat dan mengatur layanan
• PsShutdown : Shutdown atau reboot komputer
• PsSuspend : proses Suspend
• PUSHD : Simpan dan kemudian mengubah direktori sekarang
• QGREP : Cari file(s) untuk baris yang cocok dengan pola tertentu
• RASDIAL : Mengelola koneksi RAS
• RASPHONE : Mengelola koneksi RAS
• Recover : perbaikan file yang rusak dari disk yang rusak
• REG : Registry = Read, Set, Export, Hapus kunci dan nilai-nilai
• REGEDIT : Impor atau ekspor  pengaturan registry
• Regsvr32 : Register atau unregister dari sebuah DLL
• REGINI : Mengubah Registry Permissions
• REM : Record comments di sebuah file batch
• REN : Mengubah nama sebuah file
• REPLACE : Mengganti atau memperbarui satu file dengan yang lain
• RD : Menghapus sebuah folder (s)
• RMTSHARE : Share sebuah folder atau printer
• Robocopy : Mengcopy File dan Folder secara sempurna
• RUTE : Memanipulasi tabel routing jaringan
• RUNAS : menjalankan program dengan account pengguna yang berbeda
• RUNDLL32 : Menjalankan perintah DLL Command (add / remove print connections)
• SC : Sebagai servis control Layanan
• SCHTASKS : Membuat atau mengedit perintah untuk dijalankan pada waktu tertentu
• SCLIST : Menampilkan Layanan display NT service
• SET : Mendisplay, mengeset, atau menghapus variabel environment
• SETLOCAL : Mengendalikan environment visibilitas variable
• SETX : Mengeset variabel environment secara permanen
• SFC : Memeriksa Berkas Sistem
• SHARE :  Mendaftar atau mengedit file share atau share print
• SHIFT : Mengubah posisi parameter dalam sebuah file batch
• SHORTCUT : Membuat shortcut
• SHOWGRPS : Menampilkan daftar NT Workgroups dari  pengguna telah bergabung
• SHOWMBRS : Menampilkan daftar user yang menjadi anggota dari sebuah Workgroup
• SHUTDOWN : Mematikan komputer
• SLEEP : Men Sleep komputer
• SLMGR : Menampilkan software Licensing Management (Vista/2008)
• SOON : Menjadwal kan perintah untuk menjalankan dalam waktu dekat
• SORT : Penyortian input
• START : Memulai sebuah program atau perintah pada jendela terpisah
• SU : Switching User
• SUBINACL : Mengedit file dan folder Permissions, Kepemilikan dan Domain 
\end{document} 