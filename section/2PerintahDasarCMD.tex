%membuat dokumen
\documentclass{article}
\usepackage[indonesian]{babel} %untuk pemenggalan kata bahasa indonesia
\title{Perintah Dasar CMD}
\author{Kelompok 2}
\date{23 Mei 2018}

\begin{document}
\maketitle
\section{Abstraksi}
Command prompt atau (CMD) adalah sebuah perintah DOS (Disk Operating System)  yang tersedia dan ter install di windows yang dapat di akses oleh kita secara offline atau online.CMD merupakan sebuah fitur yang sangat berfungsi dan berguna, menggunakan CMD kita bisa mengakses, me rename, menghapus atau bahkan memindahkan sebuah file dengan sangat mudah. 
Dan mungkin untuk sebagian orang masih buta akan fitur CMD ini. Yang mungkin biasa kita kenal dengan Command Prompt atau DOS prompt merupakan sebuah command line (baris perintah) yang digunakan pada sebuah OS (OperatingSystem) berbasis GUI untuk mengeksekusi sebuah file dengan cara menuliskan perintahnya pada jendela cmd atau singkatnya Command Prompt ini merupakan sistem operasi berbasis beberapa baris perintah. 