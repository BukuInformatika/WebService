%Resume uwsgi D4TI2A Kelompok 3
%\begin{enumerate}
%\Aldi Maulana Yusuf                		1164001
%\Dieni Hanifah               			1164008 
%\Ikrima Ningrumsari Mulyana    		1164013
%\Muhammad Farid Alwan Assyifa         1164017 
%\Rojasqi Fadilla              			1164026 
%\end{enumerate}

\section{definisi uwsgi}
UWSGI merupakan antarmuka Gateway Server Web yang ditulis dalam C, Nginx merupakan sebuah web server untuk
menjalankan skrip python dan Uwsgi merupakan antarmuka yang didukung oleh web server nginx yang biasanya di implementasikan dengan kerangka kerja flask. Nginx terhubung ke protokol uWSGI untuk mengaktifkan frontend web berbasis kinerja tinggi WSGI. uWSGI melengkapi NGiNX dengan menangani dinamika konten \cite{balaji2013sentinel}.

\subsection{definisi uwsgi}
uWSGI merupakan protokol komunikasi untuk jaringan, maka uWSGI yang akan menanggapi Nginx melalui protokol WSGI. Lalu data dinamis dari interaksi antara program komputasi pada Python dan uWSGI yang akan behubungan dengan Nginx, maka Nginx dapat berkomunikasi dengan kode Python backend menggunakan uWSGI lalu uWSGI akan menjalankan kode Python pada program tersebut\cite{hirschbergreal}. 

\subsection{definisi uswgi}
server aplikasi uWSGI dapat menangani permintaan dinamis. testbed dapat dikontrol dengan menentukan sinyal Simulink dan parameter (misalnya variabel referensi, gain factor, parameter fungsi transfer) atau fungsi (misalnya start / stop, reset, dan perpindahan aliran sinyal). Karena penggunaan UDP blok Simulink, para RCL dapat disalurkan ke testbeds Simulink dan dikontrol dengan mudah \cite{zumsandedesign}.

