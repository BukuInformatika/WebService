%Resume uWSGI D4TI2A Kelompok 3
%\begin{enumerate}
%\item Aldi Maulana Yusuf                		1164001
%\item Dieni Hanifah               			1164008
%\item Ikrima Ningrumsari Mulyana    		1164013
%\item Muhammad Farid Alwan Assyifa         	1164017
%\item Rojasqi Fadilla              			1164026
%\end{enumerate}

\section{definisi uwsgi}
UWSGI merupakan antarmuka Gateway Server Web yang ditulis dalam C, Nginx merupakan sebuah web server untuk
menjalankan skrip python dan Uwsgi merupakan antarmuka yang didukung oleh web server nginx yang biasanya di implementasikan dengan kerangka kerja flask. Nginx terhubung ke protokol uWSGI untuk mengaktifkan frontend web berbasis kinerja tinggi WSGI. uWSGI melengkapi NGiNX dengan menangani dinamika konten \cite{balaji2013sentinel}.

\subsection{definisi uwsgi}
uWSGI merupakan protokol komunikasi untuk jaringan, maka uWSGI yang akan menanggapi Nginx melalui protokol WSGI. Lalu data dinamis dari interaksi antara program komputasi pada Python dan uWSGI yang akan behubungan dengan Nginx, maka Nginx dapat berkomunikasi dengan kode Python backend menggunakan uWSGI lalu uWSGI akan menjalankan kode Python pada program tersebut\cite{hirschbergreal}.

\subsection{definisi uswgi}
server aplikasi uWSGI dapat menangani permintaan dinamis. testbed dapat dikontrol dengan menentukan sinyal Simulink dan parameter (misalnya variabel referensi, gain factor, parameter fungsi transfer) atau fungsi (misalnya start / stop, reset, dan perpindahan aliran sinyal). Karena penggunaan UDP blok Simulink, para RCL dapat disalurkan ke testbeds Simulink dan dikontrol dengan mudah \cite{zumsandedesign}.

\subsection{definisi uwsgi}
Meskipun yang sangat konvensi penamaan membingungkan, uWSGI itu sendiri adalah sebuah proyek besar dengan banyak komponen, yang bertujuan untuk memberikan full [software] stackuntuk building hosting services. Salah satu komponen ini, server uWSGI, menjalankan aplikasi Python WSGI. Ini mampu menggunakan berbagai protokol, termasuk protokol kawat uwsgi sendiri , yang quasi-identik dengan SCGI. Dalam rangka memenuhi (dimengerti) permintaan untuk menggunakan server HTTP yang berdiri sendiri di depan server aplikasi, nginx dan Cherokee server web Modular untuk mendukung uWSGI (berperforma terbaik) 's uwsgi protokol untuk memiliki kontrol langsung atas proses nya\cite{dong2015chemdes}.

\subsection{definisi uwsgi}
UWSGI adalah aplikasi yang memproses kode Python dan menyediakan WSGI (Layanan Web
Gateway Interface) yang memungkinkan server web untuk berkomunikasi
Dengan Django, yaitu, dalam prakteknya Nginx menawarkan traffic dan aplikasi Djangos ke uWSGI
menangani lalu lintas yang diterima. WSGI adalah antarmuka standar untuk server
dan komunikasi aplikasi.
Untuk setiap aplikasi Django, konfigurasikan pengaturan uWSGI yang dapat digunakan
Tentukan berbagai pengaturan terkait server. Gambar 16 menunjukkan
definisi yang dibuat dalam file konfigurasi uWSGI dalam aplikasi perizinan\cite{wood2018ccbuilder}.

\section{Cara Instalasi UWSGI}
Cara install uWSGI
Resep ini menjelaskan cara menjalankan web2py di belakang server web Cherokee menggunakan uWSGI.
Cherokee adalah webserver yang ditulis dalam bahasa C, serupa dengan maksud untuk Lighttpd: cepat, kompak, dan modular.
Cherokee hadir dengan antarmuka administratif yang memungkinkan seseorang untuk mengelola konfigurasinya,
yang sulit dibaca dan dimodifikasi sebaliknya. uWSGI digambarkan di situs webnya sebagai cepat
(murni C), penyembuhan diri sendiri, server aplikasi kontainer pengembang / sysadmin-friendly. Cherokee punya
modul yang disertakan untuk berbicara dengan server uWSGI.

Instal paket atau unduh, kompilasi, dan instal komponen yang diperlukan.
Buat file berikut di root instalasi web2py, dan beri nama uwsgi.xml:
<uwsgi>
<pythonpath> / home / web2py </ python>
<modul> wsgihandler </ modul>
<socket> 127.0.0.1:37719 </ socket>
<master />
<processes> 8 </ processes>
<memory-report />
</ uwsgi>
Konfigurasi ini memunculkan delapan proses untuk mengelola beberapa permintaan dari
Server HTTP. Ubah sesuai kebutuhan, dan konfigurasikan <pythonpath> ke instalasi
akar web2py.\cite{mulone2012web2py}.

\subsection{cara instalasi USWGI}
Pada prosedur pemasangan berikut ini menguraikan tindakan yang diperlukan untuk menginstal Teaplot pada  FreeBSD berbasis sistem. Perintah yang diinputkan pertama kali  dengan menggunakan karakter  (#) hastag  untuk menunjukkan bahwa eksekusi harus dilakukan di dalam shell dengan root untuk hak aksesnya , sedangkan $ menunjukkan hak-hak atas tingkatan penggunanya dalam mengakses

1.	Jika TEACUP belum terintall maka bias didownload maelalui laman http://downloads.sourceforge.net/
	project/teacup/teacup-1.0.tar.gz. Informasi tambahan dapat diperoleh dari laporan CAIA nya.
2.	kemudian kita extract arsip instalasi Teaplotnya kedalam  TEACUP  dengan direktorinya ($TEACUP_DIR)dengan cara : 
	$ cd $TEACUP_DIR
	$ tar xvf teaplot-0.1.txz
3.	Setelah itu untuk menerapkan patch Teaplot kedalam file fabfile.py didalam direktori yang sama menggunakan peintah :
	$ patch -p1 < teaplot.patch
\cite{true150828teaplot}.

\subsection{instalasi uWSGI}
Untuk uWSGI kita perlu menginstallnya seperti berikut :
Ketik, “ pip install uwsgi “, (tanpa tanda “ ”) untuk awal menginstall uWSGI,
Sekarang, di dalam folder bin virtualenv kamu, akan ada perintah uWSGI. cari
di mana letak perintah uWSGI itu karena kita akan membutuhkannya. Setelah itu
Buat file wsgi.py di dalam folder proyek Anda dengan konten berikut:
# coding:utf-8
from main import app_factory
app = app_factory(name="myproject")
Sebuah uWSGI menggunakan instance aplikasi dari file yang telah kita buat di atas untuk terhubung ke aplikasi kita. Sebuah app_factory adalah fungsi dari bawaan pabrik yang menciptakan aplikasi kita. Pastikan saja aplikasi yang dibuat sudah dikonfigurasi dengan benar. Selanjutnya, kita melanjutkan untuk menghubungkan uWSGI ke aplikasi kita.

kita dapat memanggil binary uWSGI kita dengan semua parameter yang diperlukan untuk memuat wsgi kita. file ‘py’ langsung dari baris perintah atau kita dapat membuat file ‘ini’ , dengan semua konfigurasi yang diperlukan, dan hanya memberikannya ke binary. Seperti yang Anda lihat, pendekatan kedua biasanya lebih baik, jadi buatlah file ‘ini’ seperti berikut:

[uwsgi] 
user-home = /home/your-system-username 
project-name = myproject
project-path = %(user-home)/%(myproject) 
# make sure paths exist 
socket = %(user-home)/%(project-name).sock 
pidfile = %(user-home)/%(project-name).pid 
logto = /var/tmp/uwsgi.%(prj).log 
touch-reload = /tmp/reload 
chdir = %(project-path) 
wsgi-file = %(project-path)/wsgi.py 
callable = app 
chmod-socket = 664 
master = true 
processes = 5 
vacuum = true 
die-on-term = true 
optimize = 2

The user-home, project-name, dan project-path adalah alias yang kita gunakan untuk membuat pekerjaan kita lebih mudah. Opsi socket menunjuk ke file socket yang akan digunakan server HTTP kita untuk berkomunikasi dengan aplikasi kita. \cite{maia2015building}.

\subsection{cara instalasi uWSGI}
untuk menginstall uWSGI pertama yang di perlukan adalah Python dan kompiler C (gcc dan clang). Bergantung pada bahasa yang ingin anda dukung. Anda akan membutuhkan header pengembanganya. Pada sistem Debian/ Ubuntu Anda dapat mengistalnya(dan sisanya dari infrastruktur yang di perlukan untuk membangun perangkat lunak). Dan juga dapat menggunakan pip untuk membangun binary dengadukungan phyton\cite{berger2017usage}.

\subsection {cara instalasi uWSGI}
Setelah sertifikat SSL diperoleh, maka sertifikat perlu direferensikan pada konfigurasi Nginx, yang merupakan server web yang Tethys, lalu file konfigurasi Nginx atau uWSGI akan ditampilkan. Lalu buat salinan blok server yang tidak aman dan tempelkan di bawah yang asli. Lalu masuk ke Proxy Base URL terakhir restart Nginx untuk melihat hasil install\cite{swain2018tethys}.

\section{Contoh uWSGI}
Flask micro-framework merupakan pustaka Python yang dapat membantu untuk pembuatan Web. uWSGI merupakan implementasi protokol WSGI, uWSGI dapat menangani transmisi data
antara kode Python . Perbedaan diantara uWSGI dan Flask adalah bahwa uWSGI berfungsi sebagai aplikasi namun tidak hanya aplikasi sekarang UWSGI dapat dijalankan dimana saja sedangkan Flask berfungsi untuk membantu pembuatan web \cite{mulerolinked}.

\section{contoh uWSGi}
uWSGI mendukung berbagai protokol dan cara untuk mengambil file  konfigurasi seperti stdin dan http. contohnya
# uwsgi [option] [option 2] .. -w [wsgi.py with application callable]

# Simple server running *wsgi*
uwsgi --socket 127.0.0.1:8080 -w wsgi

# Running Pyramid (Paster) applications
uwsgi --ini-paste production.ini

# Running web2py applications
uwsgi --pythonpath /path/to/app --module wsgihandler

# Running WSGI application with specific module / callable names
uwsgi --module wsgi_module_name --callable application_callable_name
uwsgi -w wsgi_module_name:application_callable_name 
Meskipun mudah membingungkan dan sulit untuk dikelola, cara paling dasar untuk menjalankan uWSGI sama seperti skrip shell lainnya - dengan menyediakan konfigurasi yang diperlukan sebagai argumen\cite{cencini2017data}.

