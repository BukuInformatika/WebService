%Resume uWSGI D4TI2A Kelompok 3
%\begin{enumerate}
%\item Aldi Maulana Yusuf                		1164001
%\item Dieni Hanifah               			1164008
%\item Ikrima Ningrumsari Mulyana    		1164013
%\item Muhammad Farid Alwan Assyifa         	1164017
%\item Rojasqi Fadilla              			1164026
%\end{enumerate}

\section{definisi uwsgi}
UWSGI merupakan antarmuka Gateway Server Web yang ditulis dalam C, Nginx merupakan sebuah web server untuk
menjalankan skrip python dan Uwsgi merupakan antarmuka yang didukung oleh web server nginx yang biasanya di implementasikan dengan kerangka kerja flask. Nginx terhubung ke protokol uWSGI untuk mengaktifkan frontend web berbasis kinerja tinggi WSGI. uWSGI melengkapi NGiNX dengan menangani dinamika konten \cite{balaji2013sentinel}.

\subsection{definisi uwsgi}
uWSGI merupakan protokol komunikasi untuk jaringan, maka uWSGI yang akan menanggapi Nginx melalui protokol WSGI. Lalu data dinamis dari interaksi antara program komputasi pada Python dan uWSGI yang akan behubungan dengan Nginx, maka Nginx dapat berkomunikasi dengan kode Python backend menggunakan uWSGI lalu uWSGI akan menjalankan kode Python pada program tersebut\cite{hirschbergreal}.

\subsection{definisi uswgi}
server aplikasi uWSGI dapat menangani permintaan dinamis. testbed dapat dikontrol dengan menentukan sinyal Simulink dan parameter (misalnya variabel referensi, gain factor, parameter fungsi transfer) atau fungsi (misalnya start / stop, reset, dan perpindahan aliran sinyal). Karena penggunaan UDP blok Simulink, para RCL dapat disalurkan ke testbeds Simulink dan dikontrol dengan mudah \cite{zumsandedesign}.

\subsection{definisi uwsgi}
Meskipun yang sangat konvensi penamaan membingungkan, uWSGI itu sendiri adalah sebuah proyek besar dengan banyak komponen, yang bertujuan untuk memberikan full [software] stackuntuk building hosting services. Salah satu komponen ini, server uWSGI, menjalankan aplikasi Python WSGI. Ini mampu menggunakan berbagai protokol, termasuk protokol kawat uwsgi sendiri , yang quasi-identik dengan SCGI. Dalam rangka memenuhi (dimengerti) permintaan untuk menggunakan server HTTP yang berdiri sendiri di depan server aplikasi, nginx dan Cherokee server web Modular untuk mendukung uWSGI (berperforma terbaik) 's uwsgi protokol untuk memiliki kontrol langsung atas proses nya\cite{dong2015chemdes}.

\subsection{definisi uwsgi}
UWSGI adalah aplikasi yang memproses kode Python dan menyediakan WSGI (Layanan Web
Gateway Interface) yang memungkinkan server web untuk berkomunikasi
Dengan Django, yaitu, dalam prakteknya Nginx menawarkan traffic dan aplikasi Djangos ke uWSGI
menangani lalu lintas yang diterima. WSGI adalah antarmuka standar untuk server
dan komunikasi aplikasi.
Untuk setiap aplikasi Django, konfigurasikan pengaturan uWSGI yang dapat digunakan
Tentukan berbagai pengaturan terkait server. Gambar 16 menunjukkan
definisi yang dibuat dalam file konfigurasi uWSGI dalam aplikasi perizinan\cite{wood2018ccbuilder}.

\section{Cara Instalasi UWSGI}
Cara install uWSGI
Resep ini menjelaskan cara menjalankan web2py di belakang server web Cherokee menggunakan uWSGI.
Cherokee adalah webserver yang ditulis dalam bahasa C, serupa dengan maksud untuk Lighttpd: cepat, kompak, dan modular.
Cherokee hadir dengan antarmuka administratif yang memungkinkan seseorang untuk mengelola konfigurasinya,
yang sulit dibaca dan dimodifikasi sebaliknya. uWSGI digambarkan di situs webnya sebagai cepat
(murni C), penyembuhan diri sendiri, server aplikasi kontainer pengembang / sysadmin-friendly. Cherokee punya
modul yang disertakan untuk berbicara dengan server uWSGI.

Instal paket atau unduh, kompilasi, dan instal komponen yang diperlukan.
Buat file berikut di root instalasi web2py, dan beri nama uwsgi.xml:
<uwsgi>
<pythonpath> / home / web2py </ python>
<modul> wsgihandler </ modul>
<socket> 127.0.0.1:37719 </ socket>
<master />
<processes> 8 </ processes>
<memory-report />
</ uwsgi>

Konfigurasi ini memunculkan delapan proses untuk mengelola beberapa permintaan dari
Server HTTP. Ubah sesuai kebutuhan, lalu konfigurasikan <pythonpath> ke instalasi
akar web2py.
1.	Sebagai pengguna yang memiliki instalasi web2py, mulai server uWSGI: $ uWSGI -d uwsgi.xml
2.	Sekarang luncurkan antarmuka administratif Cherokee untuk membuat konfigurasi baru:
$ cherokee-admin
3.	Hubungkan ke antarmuka admin dengan browser di tautan berikut:
http: // localhost: 9090 /
4.	Pergi ke bagian Sumber - (A), lalu klik tombol + - (B).
5.	Pilih Remote Host on (C), lalu isi kolom teks di (D) dengan alamat IP, dan port untuk mencocokkan konfigurasi dalam file uswgi.xml sebelumnya. Setelah mengkonfigurasi sumber uWGI, sekarang mungkin untuk mengkonfigurasi Virtual Host, dan mengalihkan permintaan melalui itu. Dalam resep ini, kita memilih Virtual Host default yang ada digunakan ketika tidak ada Virtual Host lain yang memiliki kecocokan yang lebih baik untuk permintaan yang masuk.

\cite{mulone2012web2py}.

\subsection{cara instalasi USWGI}
Pada prosedur pemasangan berikut ini menguraikan tindakan yang diperlukan untuk menginstal Teaplot pada  FreeBSD berbasis sistem. Perintah yang diinputkan pertama kali  dengan menggunakan karakter  (#) hastag  untuk menunjukkan bahwa eksekusi harus dilakukan di dalam shell dengan root untuk hak aksesnya , sedangkan $ menunjukkan hak-hak atas tingkatan penggunanya dalam mengakses

1.	Jika TEACUP belum terintall maka bias didownload maelalui laman http://downloads.sourceforge.net/
	project/teacup/teacup-1.0.tar.gz. Informasi tambahan dapat diperoleh dari laporan CAIA nya.
2.	kemudian kita extract arsip instalasi Teaplotnya kedalam  TEACUP  dengan direktorinya ($TEACUP_DIR)dengan cara :
	$ cd $TEACUP_DIR
	$ tar xvf teaplot-0.1.txz
3.	Setelah itu untuk menerapkan patch Teaplot kedalam file fabfile.py didalam direktori yang sama menggunakan peintah :
	$ patch -p1 < teaplot.patch
4.	Menginstal system dependesi di FreeBSD dengan cara sebagai berikut :
 	# pkg install py27-django py27-scipy py27-numpy uwsgi spp
5.	Setelah menginputkan data diatas, maka Instalasi telah selesai di FreeBSD. Teaplot sekarang bisa dijalankan melalui fabfile di
	direktori ($FABFILE_DIR) atau memanggil tugas Fabric “animate” dengan cara :
	$ cd $FABFILE_DIR
	$ fab animate

\cite{true150828teaplot}.

\subsection{instalasi uWSGI}
Instalasi uWSI menggunakan web server Nginx, Untuk uWSGI kita perlu menginstallnya seperti berikut :
Ketik, “ pip install uwsgi “, (tanpa tanda “ ”) untuk awal menginstall uWSGI,
Sekarang, di dalam folder bin virtualenv kamu, akan ada perintah uWSGI. cari
di mana letak perintah uWSGI itu karena kita akan membutuhkannya. Setelah itu
Buat file wsgi.py di dalam folder proyek Anda dengan konten berikut:
# coding:utf-8
from main import app_factory
app = app_factory(name="myproject")
Sebuah uWSGI menggunakan instance aplikasi dari file yang telah kita buat di atas untuk terhubung ke aplikasi kita. Sebuah app_factory adalah fungsi dari bawaan pabrik yang menciptakan aplikasi kita. Pastikan saja aplikasi yang dibuat sudah dikonfigurasi dengan benar. Selanjutnya, kita melanjutkan untuk menghubungkan uWSGI ke aplikasi kita.

kita dapat memanggil binary uWSGI kita dengan semua parameter yang diperlukan untuk memuat wsgi kita. file ‘py’ langsung dari baris perintah atau kita dapat membuat file ‘ini’ , dengan semua konfigurasi yang diperlukan, dan hanya memberikannya ke binary. Seperti yang Anda lihat, pendekatan kedua biasanya lebih baik, jadi buatlah file ‘ini’ seperti berikut:

[uwsgi]
user-home = /home/your-system-username
project-name = myproject
project-path = %(user-home)/%(myproject)
# make sure paths exist
socket = %(user-home)/%(project-name).sock
pidfile = %(user-home)/%(project-name).pid
logto = /var/tmp/uwsgi.%(prj).log
touch-reload = /tmp/reload
chdir = %(project-path)
wsgi-file = %(project-path)/wsgi.py
callable = app
chmod-socket = 664
master = true
processes = 5
vacuum = true
die-on-term = true
optimize = 2

The user-home, project-name, dan project-path adalah alias yang kita gunakan untuk membuat pekerjaan kita lebih mudah. Opsi socket menunjuk ke file socket yang akan digunakan server HTTP kita untuk berkomunikasi dengan aplikasi kita.
Kita sekarang dapat mengatur server HTTP kita, ini merupakan langkah yang mudah. Cukup instal Nginx sebagai berikut:
“ sudo apt-get install nginx-full “ tanpa tanda (“”)
Sekarang, server http kita sudah aktif dan berjalan di port 80. Mari kita pastikan Nginx tahu tentang aplikasi kita. Tuliskan kode berikut ke sebuah file di proyek kita di dalam / etc / nginx / sites-available:
server {
listen 80;
server_name PROJECT_DOMAIN;
location /media {
alias /path/to/media;
}
location /static {
alias /path/to/static;
}
location / {
include /etc/nginx/uwsgi_params;
uwsgi_pass unix:/path/to/socket/file.sock;
}
}
File konfigurasi sebelumnya menciptakan server virtual yang berjalan pada port 80. Kita perlu membuat satu file terakhir di dalam / etc / init yang akan mendaftarkan proses uWSGI kita sebagai layanan. Bagian ini sangat mudah, buat saja file di project kamu dengan nama file conf dengan kode sebagai berikut:
description "uWSGI application my project"
start on runlevel [2345]
stop on runlevel [!2345]
setuid your-user
setgid www-data
exec /path/to/uwsgi --ini /path/to/ini/file.ini
Jika semuanya berjalan dengan baik, berarti instalasi berhasil \cite{maia2015building}.


\subsection {cara instalasi uWSGI}
Setelah sertifikat SSL diperoleh, maka sertifikat perlu direferensikan pada konfigurasi Nginx, yang merupakan server web yang Tethys, lalu file konfigurasi Nginx atau uWSGI akan ditampilkan. Lalu buat salinan blok server yang tidak aman dan tempelkan di bawah yang asli. Lalu masuk ke Proxy Base URL terakhir restart Nginx untuk melihat hasil install\cite{swain2018tethys}.

\subsection{cara instalasi uWSGI}
Berikut merupakan Instal paket atau men-download, mengkompilasi dan menginstal komponen-komponen yang diperlukan.
Buat file berikut di akar instalasi web2py, dan menyebutnya uwsgi.xml:
<uwsgi>
 <pythonpath>/home/web2py</pythonpath>
 <module>wsgihandler</module>
 <socket>127.0.0.1:37719</socket>
 <master/>
 <processes>8</processes>
 <memory-report/>
</uwsgi>
- Konfigurasi ini menghasilkan delapan proses untuk mengelola beberapa permintaan dari
HTTP server. Mengubah yang diperlukan, dan mengkonfigurasi <pythonpath>ke instalasi
akar web2py.</pythonpath> 
- kemudian sebagai user yang memiliki instalasi web2py, mulai server uWSGI :
$ uWSGI -d uwsgi.xml
- Sekarang luncurkan antarmuka administratif Cherokee untuk membuat konfigurasi baru:
$cherokee-admin
- lalu konekkan kedalam browser admin dengan link : http://localhost:9090/.
- kemudian kelik tombol + -
- Pilih Remote Host on (C), lalu isi kolom teks di (D) dengan alamat IP, dan port
untuk mencocokkan konfigurasi dalam file uswgi.xml sebelumnya.
-Setelah mengkonfigurasi sumber uWGI, sekarang mungkin untuk mengkonfigurasi Virtual Host, dan
mengalihkan permintaan melalui itu. Dalam resep ini, kita memilih Virtual Host default
- kemudian klik button rule management
- Hapus semua aturan yang tercantum di sebelah kiri. Hanya aturan default yang akan tetap ada
- Konfigurasikan aturan default dengan uWSGI Handler. Biarkan nilai-nilai lain tidak berubah
- Jika Anda ingin Cherokee untuk melayani file statis langsung dari folder web2py, Anda dapat menambahkan
Aturan Ekspresi Reguler. Klik tombol (A), dan pilih Ekspresi Reguler dari
menu drop-down di (B). Ketahuilah bahwa konfigurasi ini hanya berfungsi jika web2py
direktori berada pada sistem file yang sama, dan dapat diakses oleh Cherokee.
- konfigurasi ke dalam regular expressions
- Sekarang Anda dapat mengkonfigurasi Penangan Statis yang menunjuk ke subdirektori aplikasi
instalasi web2py Anda:
- Ingat untuk menyimpan konfigurasi, dan kembali atau mulai ulang Cherokee dari
antarmuka administratif; maka Anda siap untuk memulai server uWSGI
- Ubah ke ID pengguna yang benar yang digunakan untuk menginstal web2py; Sadarilah bahwa menggunakan
root tidak disarankan.
-  Pergilah ke direktori root dari instalasi web2py, di mana Anda menyimpan konfigurasi
file uwsgi.xml.
- Jalankan uWSGI dengan opsi -d <logfile>, sehingga berjalan di latar belakang:
$ su - <web2py user>
$ cd <web2py root>
$ uwsgi -x uwsgi.xml -d /tmp/uwsgi.log \cite{reingart2012web2py}.

\subsection {cara instalasi uWSGI}
Pada penginstallan uwsgi sertifikat ssl tersebut direferensikan pada konfigurasi Nginx yang merupakan server web yang Tethys, kemudian file konfigurasi pada Nginx atau uWSGI akan menampillkan halaman Nginx atau UWSGI lalu salin blok server Kemudian masuk ke Proxy Base URL setelah selasai, restart Nginx untuk melihat hasil install Nginx atau UWSGI tersebut\cite{pellicer2016desarrollo}.

\section{Contoh uWSGI}
Flask micro-framework merupakan pustaka Python yang dapat membantu untuk pembuatan Web. uWSGI merupakan implementasi protokol WSGI, uWSGI dapat menangani transmisi data
antara kode Python . Perbedaan diantara uWSGI dan Flask adalah bahwa uWSGI berfungsi sebagai aplikasi namun tidak hanya aplikasi sekarang UWSGI dapat dijalankan dimana saja sedangkan Flask berfungsi untuk membantu pembuatan web \cite{mulerolinked}.

\subsection{Contoh uWSGi}
uWSGI mendukung berbagai protokol dan cara untuk mengambil file  konfigurasi seperti stdin dan http. contohnya
# uwsgi [option] [option 2] .. -w [wsgi.py with application callable]

# Simple server running *wsgi*
uwsgi --socket 127.0.0.1:8080 -w wsgi

# Running Pyramid (Paster) applications
uwsgi --ini-paste production.ini

# Running web2py applications
uwsgi --pythonpath /path/to/app --module wsgihandler

# Running WSGI application with specific module / callable names
uwsgi --module wsgi_module_name --callable application_callable_name
uwsgi -w wsgi_module_name:application_callable_name
Meskipun mudah membingungkan dan sulit untuk dikelola, cara paling dasar untuk menjalankan uWSGI sama seperti skrip shell lainnya - dengan menyediakan konfigurasi yang diperlukan sebagai argumen\cite{cencini2017data}.

\subsection{contoh uWSGI}
Sebagai contoh uWSGI sendiri merespon server 	menuju Nginx melalui protocol wsgi, maka Data dinamis dari interaksi program komputasi Python kemudian uWSGI akan berinteraksi dengan Nginx, dan yang terakhir akan menyajikan hasil kepada klien dalam bentuk konten statis. Selain itu, untuk memenuhi kebutuhan waktu untuk operasi data, penyedia  telah mengoptimalkan beberapa parameter terkait dari Nginx dan uWSGI seperti max-requests, harakiri and keepalive_timeout\cite{dong2015chemdes}.

\subsection {Contoh uWSGI}
uWSGI merupakan antarmuka yang menjembatani web server yang ditulis dalam bahasa C, dan biasanya juga digunakan untuk mengakses REST API, uWSGI juga yang menjadikan layanan tersedia sebagai aplikasi, dan antarmuka web Nginx yang membuatnya dapat diakses ke Internet dan memungkinkan manajemen koneksi. Berikut contoh pengimplementasian uWSGI terlihat
sebagai berikut :
[ uwsgi ]
plugins = python
chdir = /home/cuckoo/cuckoo
f i l e = u t i l s /api . py
uid = cuckoo
gid = cuckoo
c a l l a b l e = app
\cite{beran2017analyza}.

\subsection{Contoh uWSGI}
Layanan web yang dapat memproses data struktural dan memiliki keamanan yang tinggi maka diperlukan aplikasi yang dapat logging data. Aplikasi dengan layanan web ini dibuat dengan pemrograman bahasa Python dengan kerangka labu digunakan untuk mengatur URL dengan protokol uWSGI dengan sistem lengkap aplikasi dapat diakses oleh internet dengan membuka alamat web aplikasi\cite{rudiana2015perancangan}. 





