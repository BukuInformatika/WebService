\documentclass{wileySix}
\usepackage{w-bookps}


\usepackage{graphicx}
\usepackage{enumitem}

\setcounter{secnumdepth}{3}

\setcounter{tocdepth}{2}

\begin{document}

\booktitle{Web Service}
\subtitle{Semua Tentang Komunikasi antar Aplikasi Berbasis Protokol internet}

\author{Rolly Maulana Awangga}

\halftitlepage
\titlepage

\offprintinfo{Web Service, pre-release}{Rolly Maulana Awangga}


\begin{copyrightpage}{2018}
Web Service / Rolly Maulana Awangga
\end{copyrightpage}


\dedication{For my family}

\contentsinbrief %optional
\tableofcontents
\listoffigures %optional
\listoftables  %optional

%%%%%%%%%
%%Content 
%%%%%%%%%

\part[Pengenalan Web Service]
{Pengenalan\\ Web Service}

\chapter[Contoh]
{Contoh\\ Latex}
\prologue{The sheer volumne of answers can often stifle insight...The purpose
of computing\index{computing!the purpose} is insight, not numbers.}
{Hamming}

\section{Definisi}
Sistem Informasi Geografis merupakan penggalan kata dan Sistem Informasi dan Geografis. Geografis dipandang sebagai bentukan dari geospasial.
Geospasial memiliki arti geo yang berarti bumi dan spasial yang berarti ruang atau keruangan. Jadi geospasial merupakan ilmu yang mempelajari 
tata ruang dari bumi. Tata ruang melingkupi letak suatu titik di bumi baik itu letak kota, provinsi atau negara. Tata ruang juga menyajikan gambaran dari ruang tersebut yang disebut dengan ilmu kartografi atau sering disebut sebagai ilmu pembuatan peta\cite{awangga2017colenak}.

\section{Sejarah Peta}
Perkembangan peta dunia tidak luput dari para ahli geografi dan kartografi. Peta dunia yang populer pada saat ini merupkan kontribusi dari para 
pembuat peta sebelumnya

\subsection{Ptolemy's}
Ptolemy's diduga membuat peta pada abad ke 2


\subsection{Muhammad al-Idrisi}
Seorang ahli geografi dan kartografi Muhammad al-Idrisi membuat peta dunia pada abad ke 11

\begin{figure}[ht]

\centerline{\includegraphics[width=1\textwidth]{figures/petaduniaalid.JPG}}
\caption{Gambaran pengantar peta dunia karya al-Idrisi tahun 1154.}
\end{figure}

\begin{figure}[ht]
	\centerline{\includegraphics[width=1\textwidth]{figures/TabulaRogeriana.jpg}}
\vskip2pt
\caption{Tabula Rogeriana digambar oleh Al-Idrisi pada tahun 1154 untuk Raja Normandia Roger II dari Sisilia, setelah delapan menetap di istananya, di mana dia bekerja untuk penjelasan dan ilustrasi peta.}
\end{figure}

\section{Penentuan Kordinat}
Kordinat digunakan untuk mengacu sebuah titik lokasi di muka bumi, adapun beberapa jenis standar kordinat yang digunakan adalah.

\subsection{Kordinat Internasional}
Kordinat internasional dikenal dengan long dan lat.


\subsection{Kordinat Indonesia}
Masih ingatkah pelajaran geografi tentang letak Indonesia? maka kita bisa melihat jawaban tersebut dalam kordinat berbahasa indonesia.


\chapter[RESTful]
{Definisi\\ RESTful}
\begin{itemize}
\item Ahmad Syafrizal Huda (1164062)
\item Annisa Fathoroni (1164067)
\item Puad Hamdani (1164084)
\item Rahmi Roza (1164085)
\item Tasya Wiendhyra (1164086)
\end{itemize}

\section{Definisi RESTful Web Service}
REST merupakan salah satu macam web service yang memasukkan konsep perpindahan antar state. State disini bisa dibayangkan seperti jika browser meminta suatu halaman web, maka server akan mengirimkan state halaman web yang sekarang ke browser. Menurut salah satu perkembangan Tidwell, D., 2001 bernavigasi melalui link-link yang disediakan sama halnya dengan mengganti state dari halaman web. Begitu pula REST bekerja, dengan bernavigasi melalui link-link HTTP untuk melakukan aktivitas tertentu, seakan-akan terjadi perpindahan state satu sama lain \cite{indrawan2017implementasi}.
Pada gambar \ref{labelgambar} menerangkan cara Rest Web Service melakukan request kepada server kemudian server membalasnya dengan result berupa json. Metode tersebut telah dikembangkan oleh Roy Thomas Fielding dalam disertasinya tentang Architectural Style.  Dalam disertasinya tersebut REST (Representational state transfer) didefinisikan sebagai suatu gaya arsitektur perangkat lunak untuk pendistribusian sistem hypermedia seperti WWW \cite{rofiq2017implementasi}.
\begin{figure}[ht]
\centerline{\includegraphics[width=1\textwidth]{figures/1restful.JPG}}
\caption{RESTful}
\label{labelgambar}
\end{figure}

\section{Prinsip atau Karakter Pada RESTful}
RESTful adalah salah satu teknologi web service untuk membuat suatu sistem yang terdistribusi dimana cara kerjanya berdasarkan resource. RESTful sendiri merupakan software yang didesain untuk penekanan pada skalabilitas,kesederhanaan dan kegunaan. Metode dalam REST terdiri dari empat prinsip utama teknologi, yaitu \cite{aji2016penerapan}:
\begin{enumerate}
\item Resource identifier melalui Uniform Resource Identifier (URI), REST Web service mencari sekumpulan sumber daya yang mengidentifikasi interaksi antar klien.
\item Uniform interface, sumber daya yang dimanipulasi CRUD (Create, Read, Update, Delete) menggunakan operasi PUT, GET, POST, dan DELETE.
\item Self-descriptive messages, sumberdaya informasi tidak terikat, sehingga dapat mengakses berbagai format konten (HTML, XML, PDF, JPEG, Plain Text dan lainnya). Metadata pun dapat digunakan.
\item Stateful interactions melalui hyperlinks, setiap interaksi dengan suatu sumber daya bersifat stateless, yaitu request messages tergantung jenis kontennya.
\end{enumerate}

\section{Sejarah RESTful}
REST tidak menarik perhatian banyak ketika pertama kali diperkenalkan pada tahun 2000 oleh Roy Fielding di Universitas California, Irvine, dalam disertasinya akademik, Architectural gaya dan arsitektur perangkat lunak berbasis desain jaringan, yang menganalisa set dari prinsip-prinsip arsitektur perangkat lunak yang menggunakan Web sebagai platform untuk komputasi terdistribusi (Lihat sumber daya untuk link ke disertasi ini). Sekarang, tahun setelah diperkenalkan, utama kerangka untuk REST telah mulai muncul dan masih sedang dikembangkan karena itu dijadwalkan, misalnya, akan menjadi bagian integral dari Java 6 melalui JSR-311 \cite{rodriguez2008restful}.

\section{Contoh-Contoh Penerapan  RESTful}
\subsection{Implementasi RESTful Web Service untuk Sistem Penghitungan Suara Secara Cepat pada Pilkada}
Metode ini yang digunakan oleh penyelenggara pemilihan umum untuk menentukan hasil pilkada. Dengan memanfaatkan teknologi yang ada, proses pengumpulan data hasil perolehan suara bisa dilakukan dengan lebih cepat. Salah satu metode baru yang bisa digunakan untuk melakukan proses tersebut adalah metode perhitungan cepat riil. Metode ini memanfaatkan teknologi informasi dan komunikasi untuk melakukan proses penghitungan suara. Real-quick count mengambil hasil perhitungan dari semua tempat pemungutan suara (TPS). Tetapi hasil tersebut dikirim langsung dari TPS ke lembaga penyedia informasi hasil perhitungan cepat, tidak melalui prosedur seperti pada real count yang mengharuskan pengumpulan data berjenjang, oleh karena itu waktu yang dibutuhkan untuk memperoleh semua hasil suara bisa dioptimalkan. Pada jurnal ini dilakukan perbandingan antara SOAP dan REST pada aplikasi mobile dan multimedia conference. Hasil penelitian yang dilakukan pada aplikasi mobile computing menunjukkan bahwa ukuran pesan pada RESTful web service mencapai 9 sampai 10 kali lebih kecil dibandingkan ukuran pesan dari web service berbasis SOAP \cite{rofiq2017implementasi}.

\subsection{Implementasi RESTful untuk Sales Order dan Sales Tracking Berbasis Mobile}
Bagian penjualan merupakan bagian yang paling penting dalam penjualan produk. Perusahaan membutuhkan sistem yang dapat membantu aktivitas dan pemesanan produk. dengan membuat sebuah Controller terlebih dahulu,yang berperan untuk menentukan informasi apa yang akan disampaikan pada saat client mengakses web service. Dibuat dengan arsitektur REST dengan menggunakan method yang di dukung protokol HTTP seperi method DELETE, UPDATE, CREATE,dll. Aplikasi mobile ini akan menggunakan data dari GPS untuk memastikan lokasi penjual juga dilengkapi barcode untuk mempercepat input data barang \cite{kurniawan2015implementasi}.

\subsection{Implementasi REST Web Service Pada Aplikasi Pengolah Pesan Yahoo Messenger Pada CV. Meliana Pratama}
Mengimplementasikan REST Web Service pada aplikasi pengolah pesan Yahoo Messenger (YM). Aplikasi REST Web Service dapat dijadikan sebagai miidleware antara aplikasi pengolahan pesan Yahoo Messenger (YM) dengan database, sehingga proses transaksi ke database menjadil lebih efisien. Hal ini dikarenakan aplikasi client tidak perlu mengetahui database apa yang digunakan oleh server \cite{ikrom2015implementasi}.

\subsection{Implementasi RESTful Web Service Pada Aplikasi Iklan Baris Online}
Pada implementasi aplikasi ini menerapkan restful web service yang dimana server akan berinteraksi dengan client pada interface yang sama atau seragam. Server akan meng-host resource sedangakn client akan menjadi konsumen dari resource yang disediakan server. Pada saat server meminta atau request data script request akan dikirim dari client ke server berbentuk alamat url yang kemudian memanggil file PHP yang mengakses data dari databse server. saat pengambilan data, client akan memanfaatkan API yang terdapat dalam server. Setelah mendapat data dari client, server kemudian akan menyebar informasi yang dibutuhkan berupa kebutuhan barang/jasa yang bersangkutan kepada member atau client \cite{fauziah2014aplikasi}.

\subsection{Implementasi Protokol OAuth 1.0 Sebagai Autentikasi pada Aplikasi SMS Blast Berbasis Android}
Sebuah aplikasi SMS Blast berbasis Android dan sebuah web service yang digunakan oleh aplikasi untuk melakukan request terhadap data nomor telepon terhadap data yang sudah ada. OAuth menggunakan token pada setiap request. Web service akan membangkitkan token yang berbeda pada setiap request dari consumer. Penggunaan token ini dapat meminimalkan kemungkinan terjadinya serangan Man in the Middle Attack dan Hijacking Attack
\cite{saputra2017implementasi}

\subsection{Implementasi RESTful Web Service One Chip Multi-Client Untuk Mengoptimalkan Penjualan Pulsa All Operator}
One  Chip  All  Operator adalah sebuah chip untuk pengisian pulsa kesemua operator selluler GSM dan CDMA bahkan juga dapat digunakan untuk pengisian pulsa listrik atau listrik prabayar. Chip atau kartu perdana  yang  digunakan bukan kartu Khusus atau tidak harus dipesan kedealer penyedia pelayanan pengisian pulsa,Chip yang   digunakan cukup perdana biasa,jadi nomor  yang dipakai sehari-hari dapat dijadikan sebagai chip untuk pengIsian pulsa ke semua operator. Proses awal yang dilakuakanya itu peses deployment restful  web  service. Deployment  restful web service  merupakan proses menjalankan web service pada server seperti apache tomcat agar aplikasi client dapat mengakses service database\cite{indrawan2017implementasi}.

\subsection{Implementasi Protocol Buffers pada Aplikasi Weblog Client dan Server}
Web service yang digunakan untuk mengirimkan dan menerima   protobuf  messages adalah  RESTful  web service  yang  merupakan  teknologi  web  service  yang ringan dan mudah diimplementasikan. Client mengirimkan data yang telah diserialisasikan dalam bentuk protobuf message melalui  HTTP  request kepada RESTful  service pada server. Protobuf  message kemudian diubah menjadi data semula dengan program deserialisasi yang telah ada di server \cite{wibowo2011implementasi}.

\subsection{Implementasi Restful Web Service Menggunakan AsyncTask pada Aplikasi Library Automation Berbasis Android}
Dengan menggunakan aplikasi Library automation berbasis android ini diharapkan dapat mempermudah untuk mengakses informasi terkait referensi yang terdapat pada perpustakaan fisik penggunaan RESTful web service dengam menggunakan AsyncTask sebagai prosesnya juga dinilai cukup baik dari segi penggunaan. Diharapkan untuk pengembangan selanjutnya meningkatkan akurasi pencarian supaya end user tidak merasa bingung saat mencari informasi \cite{yudhistiraimplementasi}.

\subsection{Penerepan Restful pada Aplikasi Ayo Piknik Indonesis Berbasis Android}
Aplikasi Ayo Piknik Indonesia berbasis android yang berbasis dengan Web-server menggunakan metode Restful Webservice yang bisa menampilkan informasi wisata dengan cepat dan tepat serta pengguna juga dapat memberikan usulan tempat wisata yang baru. Kemudian akan dilakukan verifikasi agar bisa ditampilkan. Selain itu aplikasi ini juga  dapat menambahkan data wisata dengan google maps untuk memudahkan wisatawan ataupun penduduk lokal \cite{aji2016penerapan}.

\section{Pengembangan Sistem Informasi RESTful Web Service}
Pengembangan sistem informasi kependudukan berbasis mobile dan restful pada web service yaitu\cite{kurniawati2016pengembangan}:
\begin{enumerate}
\item REST Web Service pada tahap ini akan dibuat web service yang diletakkan pada server pusat untuk mengolah data JSON. Web service memiliki 3 method yaitu json decode yaitu untuk parsing data masukan, StoreData dan json encode parsing untuk data keluaran. Parameter masukan dari database SQLite ke MySQL tampak pada gambar 5, akan diparsing ke dalam format Array. StoreData yang berhubungan langsung dengan database dalam proses input, status gagal atau sukses akan disimpan dalam Array dan diolah lagi menjadi format JSON sebagai keluaran dari web service.
\item Aplikasi Android Antarmuka aplikasi android saat dijalankan akan muncul form login. Pengguna aplikasi yaitu kepala lingkungan memasukkan username dan password kemudian tekan tombol login
\end{enumerate}

\subsection{Pengembangan Sistem Informasi Kependudukan Berbasis Moblie Dan RESTFful Web Service}
Sensus biasa dilakukan secara manual, yaitu door-to-door ke setiap rumah warga namun hal tersebut membutuhkan waktu yang lama dan tidak cukup efektif, lalu dibuat solusi dari permasalahan tersebut dengan mengintegrasikan RESTful web service pada perangkat Android. Diimplementasikan pada Android karena Android memiliki kelebihan dapat mengakses database secara offline yaitu SQLite sehingga lokasi yang berada di pedalaman tetap dapat terinputkan ke database meski tidak ada jaringan internet.
Cara kerjanya yaitu petugas sensus akan memasukan data di perangkat android, yang kemudian datanya akan dimasukkan kedalam database. Webservice RESTful ini berfungsi sebagai komunikator antara android dengan database pusat. Web service ini diletakkan di server pusat untuk mengolah data JSON. Parameter masukkan dari SQLite ke MySQL akan di parsing ke format array yang diubah lagi menjadi JSON sebagai hasil dari web servicenya \cite{kurniawati2016pengembangan}.

\section{Kelebihan dan Kekurangan RESTful Web Service}
Pada tabel \ref{table:contoh} merupakan kelebihan dan kekurangan daripada RESTful Web Service dimana RESTful Web Service ini sangat berguna dalam implementasinya \cite{nugroho2012perbandingan}.
\begin{table}[h]
\begin{tabular}{|c|c|c|}
\hline
Jenis Web Service&Kelebihan&Kekurangan\\
\hline
RESTful Webs Service&-Implementasi RESTful Web Service relatif& -Struktur data yang sangat kompleks\\
&sederhana dalam hal pemrogramannya karena& sukar diadaptasi ke dalam URL.\\
&menggunakan standar-standar yang telah&\\
&diterima secara luas (HTTP, XML, dan URL).&\\
&-Server dan klien HTTP dikenali&-Implementasi dan kinerjanya sangat bergantung\\
&oleh sebagian besar bahasa pemrograman&pada kapasitas jaringan yang digunakan\\
&dan hampir semua platform perangkat&\\
&keras/perangkat lunak yang saat ini populer.&\\
\hline
\end{tabular}
\label{table:contoh}
\end{table}

\section{Implementasi PHP Web Service Sebagai Penyedia Data Aplikasi Mobile}
Dapat disimpulkan bahwa PHP Web Service bisa diimplementasikan dalam aplikasi mobile yang membutuhkan data dinamis. Pengujian atas web service bisa dilakukan dengan membuat file PHP secara manual ataupun menggunakan SOAP web service. Untuk memudahkan pemanggilan data bisa dilakukan modifikasi dengan memberikan layer tambahan berupa PHP File yang memanggil pada SOAP web service
\cite{surendra2014implementasi}.

\section{RESTful Web Service Untuk Sistem Pencatatan Transaksi Studi Kasus PT.XYZ}
PT.XYZ merupakan sebuah PT yang bergerak dalam bidang pemasaran perhiasan yang terletak di daerah Semarang.Dalam kesehariannya,terdapat transaksi-transaksi yang mempengaruhi jumlah stok barang, pencatatan dan laporan transaksi yang dilakukan. Proses tersbut berdampak pada penyesuaian data antara data pergudangan dan data pencatatan transaksi,dikarenakan terdapat beberapa database yang digunakan secara terpisah dan aplikasi-aplikasi yang berbeda.
Pemanfaatan Web Service RESTful pada sistem transaksi PT.XYZ memfokuskan pada integrasi setiap sistem yang berbeda. Request sumberdaya yang dilakukan client memanfaatkan URL yang datanya akan disertakan di HTTP header dan kemudian dikirimkan ke web service menggunakan tipe konten application/x-www-form-URLencoded untuk melakukan validasi pada middleware RESTful. dan setiap respon elalu menggunakan format standar yaitu JSON.
\cite{tanaem2016restful}

\section{MONITORING ABSENSI HARIAN KEPEGAWAIAN PADA INSTANSI PEMERINTAHAN KOTA MAKASSAR BERBASIS RESTFUL API}
Adanya monitoring absensi kepegawaian pada instansi pemerintah kota makassar menggunakan RESTful Api maka dapat menjembatani beragamnya perangkat akses informasi dengan sistemnya masing-masing dalam hal mengakses data absensi kepegawaian dan lebih terkontrol, serta dengan pemanfaatan teknologi RESTful Api maka dapat lebih mengoptimalkan perangkat akses informasi dari yang membutuhkan sehingga membuat pengaksesan infomasi absensi dapat menjadi lebih mudah dan praktis \cite{sy2017monitoring}.



%\chapter[Web]
%{Definisi\\ Web}
%%\begin{itemize}
%\item Imron Sumadireja (1164076)
%\item Jesron Marudut (1164077)
%\item Lusia Violita Aprilian (1164080)
%\item Mhd. Zulfikar Akram Nst. (1164081)
%\end{itemize}

\section{Pengertian Website}
World wide web (www atau web) merupakan halaman situs informasi yang dapat diakses secara cepat atau sarana
antar muka informasi di internet. Web dapat menggabungkan teks, grafik, dan multimedia. Web memudahkan
penggunanya untuk mengakses informasi melalui konsep hypertext sehingga memungkinkan  suatu text untuk
menjadi acuan membuka dokumen laindo. Informasi dapat mudah disebar dan diakses.

\subsection{Sejarah Website}
Sementara itu World wide web (www) dikembangkan pertama kali oleh Tim Berners-Lee pada tahun 1989. Pada
awalnya, Tim mengusulkan WWW sebagai suatu cara berbagai dokumen diantara para peneliti. Dokumen online dapat
diakses melalui alamat unik yang disebut Universal Resource Locator atau URL. Selain itu WWW tidak hanya
dikembangkan untuk keperluan para peneliti, namun juga dikembangkan untuk kalangan pendidikan, bisnis dan
perorangan. Berdasarkan penjelasan singkat diatas dapat disimpulkan bahwa antara web dan internet memiliki
hubungan yang sangat erat walaupun keduangnay tidak bisa dikatakan sama. Web merupakan bagian dari layanan
yang dapat berjala di atas teknologi internet.

\subsection{Jenis-jenis website}
Website dikelompokan dalam beberapa jenis-jenis Website agar dapat memudahkan dalam menentukan jenis website
yang akan ditentukan. Dan berikut jenis-jenis website yang dikelompokan atas beberapa dasar:
\begin{enumerate}
\item Jenis Website berdasarkan sifat;
\begin{itemize}
\item Website Statis, merupakan web yang kontenya hampir jarang diubah
\item Website Dinamis, Web yang konten atau isinya dapat berubah-ubah setiap saat
\end{itemize}
\item Jenis Website yang dikelompokkan berdasarkan Bahasa Pemrogramannya;
\begin{itemize}
\item Server side, Website yang memakai bahasa pemrograman yang tergantung dengan servernya
\item Client side, adalah web yang tidak perluu server untuk menjalankannya. Cukup diakses dengan browser.
\end{itemize}
\item Jenis-jenis Web menurut tujuannya;
\begin{itemize}
\item Web personal, biasanya web ini merupakan web yang berisi informasi seorang
\item Corporate Web, website yang dimiliki sebuah institusi atau perusahaan.
\item Web Portal, Web ini berisi banyak layanan, seperti berita, email dan jasa
\item Web Forum, sebuah web yang dibuat sebagai sarana diskusi.
\end{itemize}
\end{enumerate}
	
\subsection{Keuntungan Web}
Keuntungan penggunaan web diantaranya yaitu :
\begin{itemize}
\item Informasi dapat diberikan segera(tepat waktu) dan diperbarui secara berkala.
\item Presentasi fleksible dan visibilitas dapat menyediakan ragam isyarat untuk diseminasi informasi.
\item Informasi dapat diorganisir melalui tautan dan menu, berbagai tingkatan informasi dapat disediakan format file yang berbeda dapat digunakan untukj informasi yang dapat diunduh. Integrasi informasi dapat dilakukann melalui tautan dan seksi lain, halaman lain, atau web lain.
\item Tauta  dan menu dapat menyediakan informasi bagi pemangku kepentingan yang berbeda, informasi dapat pula diberikan melalui daftar email kepada pemangku kepentingan.
\item Setiap orang yang dapat mengakses web dapat memperoleh informasi karena keterjangkauan global dan potensi komunikasi masl dari web.
\end{itemize}
	   
\section{tentang web scraping}
Web scraping atau scraping web (dapat disebut juga panen web atau web ekstraksi data) merupakan sebuah
perangkat lunak komputer teknik penggalian informasi dari situs web seperti mengambil mengambil data
berbentuk teks yang umumnya bertipe HTML atau XHTML. contohnya seperti Internet Explorer (IE) dan Mozilla Web
Browser. web scraping berkaitan erat dengan pengindekan web.

\subsection{manfaat dari web scraping}
Web scraping sering dikenal dengan screen scraping. Web scraping tidak dapat dimasukkan kedalam bidang data
mining karena dalam data mining menyiratkan upaya untuk memahami pola semantik dari sejumlah data besar yang
telah diperoleh. Aplikasi Web scraping hanya fokus pada cara memperoleh data melalui pengambilan dan ekstrasi
dengan ukuran data yang bervariasi. Manfaat dari web scraping adalah agar informasi yang diambil lebih
terfokus sehingga dapat memudahkan dalam melakukan pencarian sesuatu, adapun cara untuk mengembangkan teknik
web scraping yaitu dengan cara sebagai berikut:
\begin{enumerate}
\item Pengembang/pembuat program mempelajari dokumen HTML dari website yang akan diambil informasinya untuk
di tag HTML tujuannya yakni untuk mengapit informasi yang akan diambil (Create Scraping Template)
\item Pengembang/pembuat program mempelajari teknik navigasi pada website yang akan diambil informasinya
untuk ditiru pada aplikasi web scraping yang akan dibuat (Explore Site Navigation)
\item Selanjutnya aplikasi web scraping akan mengotomisasi informasi yang didapatkan dari website yang telah
ditentukan (Automate Navigation and Extraction), informasi yang didapat tersebut akan disimpan dalam 
tabel basis data (Extracted Data and Package History).
\end{enumerate}

\subsection{Perbandingan Metode Web Scraping}
Berikut perbandingan antara metode Web Scraping menggunakan CSS Selector dan Xpath Selector
\begin{enumerate}
\item Penggunaan metode XPATH Selector untuk web scraping menghasilkan artikel yang lebih lengkap
dibandingkan dengan menggunakan metode CSS Selector, Ditunjukkan dengan jumlah item dan ukuran file
yang didapatkan lebih besar. Namun juga menyisakan proses lain untuk menghilangkan kode HTML yang tidak
diinginkan dari artikel yang dihasilkan menggunakan metode XPATH Selector.
\item Dalam penggunaan memori baik metode XPATH Selector dan CSS Selector tidak memiliki perbedaan yang
signifikan(cenderung sama). Disebabkan karena engine scrapy yang baik dalam penggunaan resource-nya. 
\item Metode XPATH Selector memiliki waktu proses yang lebih cepat daripada menggunakan metode CSS Selector.
\item Pada metode XPATH, selector cukup mengikuti node pada halaman web, sehingga waktu yang dibutuhkan
relatif lebih singkat.
\end{enumerate}

\section{Tentang Web Hosting}
Web hosting merupakan jasa penyewaaan tempat penyimpanan data di internet atau biasa disebut dengan cloud
yang diperlukan oleh sebuah website. Web hosting ialah salah satu syarat agar website bisa diakses secara 
online dan dapat diakses dari seluruh dunia. Ukuran yang digunakan dalam suatu web hosting adalah kapasitas
dan bandwidth. Kapasistas merupakan ukuran besarnya kemampuan sebuah web hosting untuk menyimpan data-data di internet.

Bandwidth merupakan ukuran maksimal dari jumlah volume data yang diperbolehkan untuk diakses dari web hosting
setiap bulannya. Sebagai contoh, sebuah halaman website yang mempunyai ukuran 2 MB dan bandwidth web hosting
2000 MB, maka setiap bulannya website tersebut dapat diakses sebanyak 2000 kali.

\subsection{Tentang Domain}
Domain merupakan sebuah alamat di dunia internet atau sebuah identitas dari sebuah website. Domain digunakan untuk mempermudah dalam mengakses situs yang ada di internet. Domain terbagi menjadi 2 jenis domain yang dibagi berdasarkan pemisahaan titiknya, yaitu; Top Level Domain (TLD) dan Second Level Domain (SLD). Top Level Domain merupakan bagian terakhir dalam sebuah domain website. Contohnya "facebook.com" dan disitu yang jadi domainnya adalah ".com". Selanjutnya Second Level Domain atau SLD merupakan bagian dari domain yang terdapat sebelum Top Level Domain. Contohnya "Facebook.com" yang menjadi SLDnya adalah Facebook. Jadi SLD adalah unsur domain yang didaftarkan terdahulu pada jasa Web Hosting. Dan ada juga yang disebut Country Code Second Level Domain (ccSLD). Berguna sebagai penunjuk organisasi apa yang mendaftar pada suatu domain. Setiap negara juga mempunyai ccSLD yang berbeda-beda tiap negaranya.

\subsection{Tentang Hubungan Domain dan Web Hosting}
Hubungan Domain dan Web Hosting merupakan satu kesatuan yang saling membutuhkan. Pada sebuah Website, domain dan web hosting saling ketergantungan. Apabila yang tersedia hanya web hosting, maka website tidak akan dapat diakses. Begitu juga dengan domain, apabila yang tersedia hanya domain, maka tidak akan ada website yang akan ditampilkan, karena halaman website tersimpan didalam web hosting.

\subsection{Macam-macam Web Hosting}
Saat ini banyak jasa penyedia hosting dengan harga relatif murah bahkan gratis. Berikut adalah macam-macam web hosting :
\begin{enumerate}
\item Free Hosting / Web Hosting Gratis
Dengan free hosting, kita dengan mudah mencari layanan web hosting dan domain gratis di internet dengan menggunakan fasilitas search engine seperti google atau yang lainnya. Biasanya penyedia web hosting tidak mengenakan biaya. Namun, memiliki banyak keterbatasan beberapa fitur.
\item Web Hosting Berbagi atau Shared Hosting
Jenis hosting ini paling sering digunakan karena bukan hanya murah, namun juga memiliki layanan yang  dapat mencukupi segala kebutuhan. Biasanya, untuk menggunakan layanan web hosting ini anda hanya perlu untuk mengeluarkan biaya sebesar 100 hingga 200 ribu untuk mendapatkan ruang sebesar 2 GB – 7,5 GB dengan bandwith unlimited.
\item VPS Web Hosting
VPS merupakan singkatan dari Virtual Private Server. Jadi disini anda dapat seperti memiliki server sendiri untuk situs anda. dengan server ini, anda akan memiliki control yang lebih dalam seperti Dedicated Server.
\item Dedicated Web Hosting
Dedicated Web Hosting merupakan sebuah layanan hosting dengan server yang memiliki kemampuan untuk melakukan handle terhadap traffic dengan jumlah sangat banyak. Serta memiliki banyak fitur premium di dalamnya. Selain itu,juga memiliki control penuh terhadap server walaupun  hanya menyewanya.
\item Managed Web Hosting
Managed Hosting / Web Hosting Terkelola ini adalah web hosting yang dikhususkan untuk situs dengan. Platform yang sama. Managed web hosting biasanya lebih aman dan kinerjanya lebih optimal. Selain itu, memudahkan untuk melakukan beragam pengaturan, mulai dari installasi, sampai setting macam-macamnya.
  \end{enumerate}

\subsection{Cara Mendapatkan Web Hosting dan Domain}
Web hosting dan domain lebih sering ditemukan oleh perusahaan-perusahaan penyedia web hosting atau domain. Untuk menemukannya, cukup cari di google. Maka akan banyak perusahaan yang menyediakan jasa web hosting. Kinerja web hosting berbeda-beda dari setiap perusahaan Web hosting. Karna apabila web hosting yang dibuat oleh jasa tersebut buruk maka website tersebut akan mudah bermasalah. Oleh karena itu dalam pembelian jasa domain atau web hosting perlu diperhatikan hal-hal seperti profil dari penjual web hosting, fitur untuk website dan harga dari web hosting tersebut. Dalam memilih penyedia web hosting, pastikan penyedia mempunyai reputasi yang bagus dan terpercaya. Dan sebaiknya memilih perusahaan web hosting yang sudah dalam bentuk perusahaan CV atau PT agar pertanggung jawabannya jelas ketika terjadi gangguan web hosting.
	
\section{Apa itu cPanel?}
Apa itu cPanel?
cPanel adalah perangkat lunak control panel online untuk melakukan pengaturan website dalam sebuah web histing. cPanel adalah tool utama untuk pengguna web hosting. 
Beberapa fungsi utama cPanel antara lain yaitu upload data website, pengaturan data website, instalasi content management system, dan lain-lain. Melelui cPanel juga kita bisa mengatur data-data website. Kita bisa menambah, menghapus, atau memodifikasi data website.

\subsection{Backup date dengan cPanel}
Kemudahan yang diberikan cPanel sebagai control panel yakni untuk mempermudah proses hosting di suatu situs web menggunakan 3 tingkatan struktur untuk memberikan fungsi administrator, agen, dan yang memiliki situs web tersebut untuk mengatur berbagai macam aspek dari situs web dan administrasi server melalui sebuah web standar. Dalam menu cPanel untuk backup disediakan 3 pilihan layanan backup yang ada yakni:
\begin{enumerate}
\item System Backup, merupakan menu backup yang dapat digunakan untuk mendownload backup otomatis yang telah dibuat oleh administrator server
\item Full Backup, akan melakukan backup file untuk mengembalikan file yang corrupt, terhapus atau pindah pada server yang lain.
\item Backup Home Directory, berfungsi untuk memberikan hak akses untuk mengambil file yang berada pada directory home.
\end{enumerate}


\chapter[Backend]
{Definisi\\ Backend}
\input{section/1backend.tex}


%\chapter[Frontend]
%{Frontend}
%\input{section/1Frontend.tex}

\chapter[Frontend]
{Frontend}
%KELOMPOK 4 Blank-On1
%\begin{enumerate}
%\item Andri Fajar Sunandhar
%\item Cokro Edi Prawiro
%\item Fadila
%\item Sandro Samuel Sinaga
%\end{enumerate}


\section{Definisi Frontend}
frontend bisa disebut tampilan utama dari sebuah website pada frontend biasanya ditampilkan beberapa konten-konten yang bisa diakses oleh pengguna atau user yang menggunakan website tersebut. frontend juga berfungsi untuk user interace dari setiap web site. Biasanya frontend hanya menampilkan fungsi fungsi dari kontent sebuah web site seperti fungsi sebuah tombol untuk mengirim berkas atau untuk menampilkan konten konten yang lainnya dalam website tersebut.

Front-end adalah  segala sesuatu yang menghubungkan antara user dengan sistem back-end. Biasanya merupakan sebuah user interface 
dimana user akan berinteraksi dengan sistem. Pekerjaan yang sering muncul sebagai seorang front-end developer adalah desainer user interface
dan desainer user experience. Seorang front-end developer tidak akan membuat program atau aplikasinya yang berjalan di logic bisnis 
tapi fokusnya akan lebih banyak ke antarmuka, desain grafis (user interface designer) dan bagaimana membuat desain yang nyaman
digunakan oleh user (user experience designer). Bahasa pemrograman yang biasanya digunakan dalam pengembangan front-end adalah HTML.
\subsection{Fungsi Front-end}
Fungsi ini berhubungan langsung dengan pengguna dan berperan penting dalam keseluruhan proses bisnis dalam hal menghubungkan 
back-end dengan pengguna. layanan depan (front-end) bertugas mempresentasikan apa yang sudah dikerjakan oleh back-end
dan menjadi sarana bagi pengguna untuk mendapatkan segala sesuatu yang disediakan dibagian fungsi back-end. Peningkatan fungsi layanan depan yang baik akan mampu meningkatkan kepuasan pengguna\cite{razaq2014sistem}.
\subsection{Programming language on Front-end: Javascript }
Frontend programming language there are various. Such as HTML, CSS, and Javascript. One of them is Javascript. Javascript is language in the form of a script that in its function can run on an HTML document, where throughout history this language is the first scripting language in development / for the web. Javascript is a programming language that provides additional capabilities against the HTML language by allowing the execution of commands on the user side, which is interpreted on the browser side rather than on the server side of the web\cite{alamsyah2003pengantar}.
\subsection{Programming language on Front-end: CSS }
Frontend programming languages other than HTML, Javascript, and others, there is called CSS. CSS stands for Cascading Style Sheet. Cascading Style Sheet itself is a technology used to beautify 
the look of the website pages (sites) that you want. Using the CSS method you can easily change the overall color and appearance of the site you create, as well as to format or change the order of your site quickly\cite{poetra2003tutorial}.
\subsection{Front-end di Android}
Di Android terdapat 2 bagian, yaitu aplikasi front-end dan back-end. Front-end adalah aplikasi yang sudah terinstal dalam perangkat mobile yang digunakan.
Back-end adalah aplikasi pendukung yang berfungsi sebagai penyuplai atau sumber data pada aplikasi front-end. Front-end merupakan suatu penghubung
antara user dengan basisdata yang digunakan untuk melakukan pemrosesan data yang disimpan. Front-end dapat diciptakan menggunakan 
beberapa bahasa program seperti Visual Basic, Visual C++, Visual Foxpro, Java, dan sebagainya. Sedangkan back-end merupakan basisdata itu sendiri.
 Secara garis besar aplikasi Front-end dibagi menjadi 2 kategori, yaitu :
\begin{enumerate}
\item Decision Support Front-end yaitu aplikasi yang hanya menampilkan  dan mencetak informasi yang diambil dari basisdata baik melalui predefined atau user defined Query.
\item  Transaction Processing front-end yaitu aplikasi yang mencakup kemampuan untuk mengedit, menambah, dan menghapus record dari basisdata\cite{nuari2014perancangan}.
\end{enumerate}
\section{Konsep Membangun Aplikasi Frontend Berbasis Web APPML(Application Modeling Language) }
Diperlukan sebuah metode penghubung antara sistem dengan dukungan JSON, XML. Dengan teknik APPML (Application Modeling Language) yang diterapkan pada sebuah aplikasi front-end berbasis HTML 5 tanpa melakukan koneksi database secara langsung, tetapi cukup memanggil service berbasis JSON maka akan diperoleh data atau informasi yang dibutuhkan tanpa harus mengunjungi sistem informasi yang ada secara langsung\cite{triyono2017konsep}.

\section{definisi web serfice }
Webservice terdiri dari 2 kata yaitu Web yang berati websit atau online 
sedangangkan service berarti layanan atau melayani aplikasi berbasis web 
Website adalah suatu sistem perangkat lunak yang dirancang untuk mendukung interaksi antara sisitem pada suatu jaringan.
web service digunakan sebagai suatu fasilitas yang di sediakan oleh suatu website untuk menyediakan layanan berupa informasi kepada 
sistem lain, sehingga sisitem lain dapat berinteraksi dengan sistem tersebut melalui service yang telah disediakan oleh sistem web service.
Webservice menyimpan data informasi dalam bentuk XML, sehingga data tersebut dapat diakses oleh sitemlain miskipun berbentuk platfrom. 
\subsection{keterkaitan web service dan front end }
sebagian besar orang sering berpikir bahwa suatu website dimiliki oleh suatu pihak 
itu merupakan suatu yang disebut dengan website. banyak yang berpikir bahwa aplikasi yamg berbasiskan 
web merupakan suatu aplikasi yang menitik beratkan tampilan front endnya pada suatu web browser 
padahal nyatanya aplikasi berbasis web tidak sepenuhnya menggunakan web browser sebagai tampilan 
frontendnya. menurut Gani pengertian website di sini atalah suatu jaringan yang luas atau keterhubungan 
antara beberapa aplikasi dan atau komponen suatu aplikasi menjadi suatu aplikasi yang baru.

\subsection{Arsitektur Web Service}
\begin{figure}[ht]
\centerline{\includegraphics[width=1\textwidth]{figures/1arsitektur.JPG}}
\caption{Arsitektur web service.} 
\label{1arsitektur}
\end{figure}

Gambar \ref{1arsitektur} mendefinisikan arsitektur dari Web Services dimana pada Web Services sendiri terdiri dari Layanan Untuk Requestor, Registery dan juga Provider. Dimana kegiatan yang dilakukan untuk setiap layanan pada arsitektur tersebut ialah, `publish' untuk Service Registery dan Service Provider. `Bind' untuk Service Provider dan Serviice Requestor , dan juga `Find' untuk Service Registery dan Service Requestor. Pada Tabel \ref{1table} dipaparkan bahasa pemerograman yang sering digunakan dalam pembuatan frontend. Terdapat tiga komponen utama dari web service, komponen komponen tersebut antara lain :
service Provider, Service Requestor, Service Registry .

\begin{enumerate}
\item service Provider adalah penyedia web service yang berfungsi menyediakan kumpulan webService yang dapat diakses oleh USER atau pengguna.
\item Service Requestor Adalah aplikasi yang bertindak sebagai pengguan yang melakukan permintaan layanan berupa WebService kepada Service Provider.
\item Service Registry Adalah tempat dimana service provider mempublikasikan layanannya. pada arsitektur Webservice, service registry bersipat opsional\cite{kurniawan2015implementasi}.
\end{enumerate}




\begin{table}[h]
\caption{Bahasa pemerograman yang sering digunakan di Frontend}
\centering
\begin{tabular}{ccccc}
\hline
one&two&three&four&five\\
\hline
Java&Phyton&PHP&Javascript&C++\\
\hline
\end{tabular}
\label{1table}
\end{table}





\chapter[Pengertian Web Service]
{pengertianwebservice}
%Resume tentang Pengertian Web Service

%Kelompok 2 D4 TI / 2B

%Alwan Suryansah				1164033 
%Dinda Ayu Pratiwi				1164034
%Kurnia Sandi					1164042
%Teduh Sanubari					1164054
%Wildan Khaustara Wijaksana		1164058


\section{Definisi}

\subsection{Hartati Deviana}

	Web service adalah suatu komponen perangkat lunak self-containing dan aplikasi modular self-describing yang dapat disiarkan, dialokasikan, dan dijalankan di dalam web. Web service adalah teknologi yang mentransformasikan kemampuan internet dengan cara menambahkan beberapa kemampuan seperti kemampuan transactional web. Apa itu Transactional Web? Transactional Web yaitu kemampuan web dalam hal saling berinteraksi dengan pola program-to-program (P2P). Fokus web selama ini didominasi oleh komunikasi program-to-user dengan interaksi business-to-consumer (B2C), sedangkan transactional web akan didominasi oleh P2P dengan interaksi business-to-business (B2B). \cite{deviana2013penerapan}.

	Web service adalah suatu sistem perangkat lunak yang dirancang untuk mendukung interoperabilitas dan interaksi antar sistem pada suatu jaringan. Web service digunakan sebagai suatu fasilitas yang disediakan oleh suatu web site untuk menyediakan layanan (dalam bentuk informasi) kepada sistem lain, sehingga sistem lain dapat berinteraksi dengan sistem tersebut melalui layanan-layanan (service) yang disediakan oleh suatu sistem yang menyediakan web service. Web service menyimpan data informasi dalam format XML, sehingga data ini dapat diakses oleh sistem lain walaupun berbeda platform, sistem operasi, maupun bahasa compiler. 

\subsection{Richards Robert}

	Web service merupakan salah satu implementasi dari teknologi XML (Extensible Markup Language) pada proses pertukaran antara (data exchange) platform yang berbeda sercara berbeda.

\textit{"A Web service is a software system designed to support interoperable machine-to-machine interaction over a network. It has an interface described in a machine-processable format(specifically WSDL).Other systems interact with the Web service in a manner prescribed by its description using SOAP messages, typically conveyed using HTTP with an XML seriali zation in conjunction with other Web-related standards"}.

Menurut Richards, web service dapat digunakan untuk berkomunikasi antara mesin satu dengan mesin yang lain melalui interface perantara yang umumnya berupa WSDL(Web Service Definition Language), layanan ini biasa bekerja pada protokol HTTP dengan bentuk response dan request berupa SOAP messange. SOAP (Simple Object Access Protocol) adalah standar untuk bertukar pesan-pesan berbasis XML melalui jaringan komputer atau sebuah jalan untuk program yang berjalan pada suatu sistem operasi (OS) untuk berkomunikasi dengan program pada OS yang sama maupun berbeda dengan menggunakan HTTP dan XML sebagai mekanisme untuk pertukaran data. Format SOAP message adalah mengikuti frame XML yang terstandarisasi \cite{ihya2011pembuatan}. 

\subsection{Chen, Xi dan Zheng, Zibin dan Yu, Qi dan Lyu, Michael R}

	Web Service adalah komponen perangkat lunak yang terintegrasi untuk mendukung interaksi antar mesin dengan mesin yang lainya ( komputer ) antar jaringan , layanan web service telah banyak digunakan untuk membangun suatu aplikasi yang berorientasi dengan layanan industri dan akademisi dalam beberapa tahun trakhir , jumlah layanan web yang tersedia untuk umum terus meningkat di internet , Namun  ini menyulitkan pengguna untuk memilih layanan yang tepat di antara banyaknya layanan web services\cite{chen2014web}.

\subsection{Witono, Timotius and Susanto, Raphael}

	Pengertian sederhana web service adalah aplikasi yang dibuat agar dapat dipanggil atau diakses oleh aplikasi lain melalui internet atau intranet dengan menggunakan XML sebagai format pengiriman pesan. Web service digunakan saat pengguna akan mentransformasi sebuah logik atau sebuah class dan objek yang terpisah dalam satu ruang lingkup yang menjadi satu, sehingga tingkat keamanan dapat ditangani dengan baik\cite{witono201511}.

\subsection{Kurniawan, Erick}

	Web Service adalah layanan yang tersedia di Internet. Web Service menggunakan format standar XML untuk pengiriman pesannya. Web Services juga tidak terikat kepada bahasa pemrograman atau sistem operasi tertentu (Ethan Cerami, 2002). Web Services adalah antar muka yang mendeskripsikan koleksi yang dapat diakses dalam jaringan menggunakan format standar XML untuk pertukaran pesan. Web Services mengerjakan tugas yang spesifik. Web Services dideskripsikan menggunakan format standar notasi XML yang disebut services description (Gottschalk, 2002)\cite{chen2014web}.

\subsection{Sarbini, Riska Nurtantyo}

	Web service merupakan satuan diskrit dari fungsionalitas programatis yang diekspos 
kepada client via protokol komunikasi, dan format data standar bernama HTTP dan 
XML. Protokol ini mengatasi masalah komunikasi lintas internet dan lintas 
firewall tanpa beralih ke solusi superior yang memerlukan port-port komunikasi 
tambahan yang harus dibuka untuk akses eksternal. Dikarenakan web service mamiliki fungsi untuk menformat dan menguraikan pesan XML\cite{sarbini2015pengembangan}. 

\subsection{M. Shalahuddin dan Rosa A.S.}

	Web Service merupakan suatu sistem yang menyediakan pelayanan yang dibutuhkan oleh klien. Klien dari web service tidak hanya berupa aplikasi web, tetapi juga bisa sebuah aplikasi enterprise. Jadi web service tidak sama dengan web server, bahkan sebuah aplikasi web pada web server dapat menjadi klien dari web service\cite{inayah2014aplikasi}.

\subsection{Gottschalk (2002)}

	Web Service adalah teknologi yang mengubah kemampuan internet dengan menambahkan kemampuan transactional web, yaitu kemampuan web untuk saling komunikasi dengan pola program to program (P2P). Fokus web selama ini didominasi oleh komunikasi program to user dengan interaksi business to costumer (B2C), sedangkan stransactional web akan didominasi oleh P2P dengan interaksi business to business\cite{fauziah2014aplikasi}.


\subsection{Slameto, Andika Agus}

	Web service adalah suatu sistem perangkat lunak yang dirancang untuk mendukung interoperabilitas dan interaksi antar sistem pada suatu jaringan. Web service digunakan sebagai suatu fasilitas yang disediakan oleh suatu web site untuk menyediakan layanan (dalam bentuk informasi) kepada sistem lain, sehingga sistem lain dapat berinteraksi dengan sistem tersebut melalui layanan-layanan (service)yang disediakan oleh suatu sistem yang menyediakan web service. Web service menyimpan data informasi dalam format XML, sehingga data ini dapat diakses oleh sistem lain walaupun berbeda platform, sistem operasi, maupun bahasa compiler\cite{slameto2015penerapan}.

\subsection{Jurnal Masyarakat Informatika}

	Web service adalah antarmuka yang mendeskripsikan sekumpulan operasi yang dapat diakses dalam sebuah jaringan melalui pesan XML yang telah distandartkan.xml iyalah bahasa markup yang sudah terintregrasi dengan web service. W3C mendefinisikan web service sebagai sebuah sistem perangkat lunak yang dirancang untuk mendukung inter operasi mesin ke mesin di sebuah jaringan.  Web service merupakan komponen perangkat lunak loosely coupled, dapat diguna ulang, membungkus fungsionalitas diskret, didistribusikan, dan diakses secara programatik melalui protokol internet standart . dan sangat di di perhatikan di bidang informatika \cite{saputra2integrasi}.

\subsection{Jurnal Sistem dan Teknologi Informasi}

	Web service menurut World Wide Web Consortium (W3C) (2004), organisasi yang mengembangkan standar-standar dalam dunia web, mendefinisikan web service sebagai "\textit{“a software system designed to support interoperable machine-to-machine interaction over a network. It has an interface described in a machine-processable format (specifically WSDL). Other systems interact with the Web service in a manner prescribed by its description using SOAP messages, typically conveyed using HTTP with an XML serialization in conjunction with other Web-related standards.” }"(Lucky,2008).

Berdasarkan definisi dari W3C dapat disimpulkan bahwa web service merupakan aplikasi yang dibuat agar dapat dipanggil atau diakses oleh aplikasi lain melalui internet maupun intranet dengan menggunakan XML sebagai format pengiriman pesan\cite{prasetya2013perancangan}.

\subsection{Pengertian Web Service menurut Hartono, Fajar Fani and Hendry, H and Somya, Ramos}

	Web Service dapat diartikan sebuah antar muka atau dalam bahas inggris yaitu interface  yang berarti menggambarkan sebuah sekumpulan operasi-operasi yang kemudian dapat diakses melalui jaringan, misalnya internet dalam bentuk pesan “Extensible Markup Language (XML)”. Web Service juga menyediakan standar komunikasi dalam berbagai software yang berbeda-beda, dan dapat berjalan di berbagai platform maupun framework\cite{hartono2013aplikasi}.

\subsection{Pengertian Web Service Menurut Kasaedja, Bramwell A and Sengkey, Rizal and Lantang, Oktavian A}

	O’Reilly menerbitkan sebuah buku, David A Chappel dan Tyler Jewell sebagai penulis mengartikan bahwa web service adalah suatu kumpulan logika bisnis dalam internet yang dapat di akses melalui protocol internet. Dalam buku tersebut juga dijelaskan bahwa terdapat tiga komponen teknologi dalam Web service yaitu, Simple Object Acces Protocol (SOAP), Web Service Description Language (WSDL), dan Universal Description, Discoveri, Integration (UDDI)\cite{kasaedja2014rancang}.

\subsection{Hamdani, Hamdani and Haviluddin, Haviluddin and Darmawangsa, Ngurah Satria}

	Web service diartikan sebagai sebuah antar muka (interface) yang menggambarkan sekumpulan operasi-operasi yang dapat diakses melalui jaringan, misalnya internet, dalam bentuk pesan XML. Web service diartikan sebagai sepotong atau sebagian informasi atau proses yang dapat diakses oleh siapa saja, kapan saja dengan menggunakan piranti apa saja, tidak terikat dengan sistem operasi atau bahasa pemrograman yang digunakan.

\subsection{Novi Nuari}

	Webservice ialah suatu sistem perangkat lunak yang dibangun guna mendukung interaksi antar mesin dalam suatu jaringan. Webservice digunakan sebagai suatu fasilitas yang disediakan oleh suatu website untuk menyediakan layanan (dalam bentuk informasi) kepada mesin lain, sehingga mesin lain dapat berinteraksi dengan mesin tersebut melalui layanan-layanan (service) yang disediakan oleh provider\cite{nuari2014perancangan}.

\subsection{Wellem, Theophilus}

	Web service merupakan suatu software sistem yang mendukung interaksi yang interoperable dari machine to machine melalui jaringan (World World Wide Consortium).  (Stencil Group). Dengan suksesnya Web service sebagai suatu standar teknologi software, memberikan peluang yang besar untuk pengembangan aplikasi terdistribusi melalui Internet.
Web service sebagai suatu standar teknologi software, memberikan peluang yang besar untuk pengembangan aplikasi terdistribusi melalui Internet. Saat ini Web service tidak hanya dapat diakses melalui komputer saja, tetapi juga dapat diakses melalui mobile device, seperti telepon seluler dan PDA, sehingga memungkinkan diciptakannya layanan mobile menggunakan Web service dan aplikasi mobile yang menggunakan Web service ini\cite{wellem2015perancangan}.

\subsection{Jurnal Informatika Kenali, Eko Win }

	Menurut Gerami (2002) web services adalah suatu layanan-layanan yang disediakan oleh internet, dengan menggunakan pengiriman pesan format Extensible Markup Language (XML), dan tidak saling bergantung pada satu sistem operasi atau Bahasa pemrograman. Komponen dalam web service memiliki 3 arsitektur, dan masing-masing komponen tersebut adalah Service provider, Service requestor, dan Service registry\cite{kenali2015desain}. 

\subsection{Sigit, Haris Triono and Sulistiyono, Sulistiyono}

	Web Service adalah bagian dari perangkat lunak yang membuat dirinya tersedia melalui internet dan menggunakan sistem pesan XML standar. XML digunakan untuk mengkodekan semua komunikasi ke Web Service. Misalnya, klien memanggil Web Service dengan mengirim pesan XML, kemudian menunggu tanggapan XML yang sesuai. Karena semua komunikasi ada dalam XML, Web Service tidak terkait dengan sistem operasi atau bahasa pemrograman manapun. Web Service adalah kumpulan protokol dan standar terbuka yang digunakan untuk pertukaran data antara aplikasi atau sistem\cite{sigit2017desain}.  

\section{Manfaat}

Layanan web memungkinkan penyedia layanan dan vendor untuk menjual layanan mereka dengan memublikasikannya
Yang di akses melalui World Wide Web.
Manfaat dari layanan web kita dapat berbagi data walaupun memiliki jarak yang jauh dan dapat mempermudah membagi suatu data dalam sebuah pekerjaan
interoperabilitas. Manfaat ini berasal dari antarmuka XML standar dan deskripsi akses
diberikan oleh WSDL (Web Services Description Language). Deskripsi WSDL sangat membantu dalam perusahaan
integrasi aplikasi, integrasi B2B (menyelesaikan tantangan antara bisnis dan bisnis partner, seperti customer, supplier, bank, dan jasa transportasi ) \cite{ferris2003web}.

\section{Arsitektur Web service}

\subsection{\textit{Service Oriented Architecure (SOA)} }

	Konsep arsitektur yang mendasari teknologi Web service adalah Service Oriented Architecure (SOA), SOA mendefinisikan 3 peran berbeda yang menunjukkan peran dari masing-masing komponen dalam system, yaitu (W3C, 2004) :
\begin{itemize}
\item \textit{Service provider}, yaitu suatu entitas yang menyediakan interface terhadap sistem yang menjalankan suatu sekumpulan tugas tertentu.
\item \textit{Service requestor}, yaitu suatu entitas yang meminta/memperoleh (dan menemukan) \textit{software service} dalam rangka meyelesai kan suatu tugas tertentu atau menyediakan solusi bisnis tertentu.
\item \textit{Service registry}, yaitu entitas yang bertindak sebagai penyimpan (\textit{repository}) suatu \textit{software service} yang dipublikasikan oleh \textit{service provider}\cite{hidayat2014penerapan}.
\end{itemize}

\subsection{Jurnal Masyarakat Informatika}

	Web service dibangun dari tiga komponen unsur utama, yaitu service provider, service registry, dan service requestor. Komponen-komponen tersebut saling berinteraksi melalui komponen web service itu sendiri, yang berupa deskripsi dan implementasi layanan dan prasarana. Dan juga terdapat tiga macam operasi yang memungkinkan komponen komponen tersebut untuk dapat saling berinteraksi, yaitu publish, find, dan bind. Keterkaitan antara peran, operasi, dan komponen web service \cite{saputra2integrasi}.

\subsection{Arsitektur RESTful Web services}

	Berikut merupakan langkah-langkah yang dilakukan dalam model dasar RESTful Web services (HostBridge, 2009):
\begin{enumerate}
\item Query Request Provider melalui HTTP dengan menggunakan URI (Uniform Resource Identifier). Request menggunakan methods (metode) HTTP untuk menentukan apakah request tersebut dimaksudkan untuk Create (menciptakan), Read (membaca), Update (memperbarui), atau Delete (menghapus) data.
\item HostBridge mengembalikan sebuah dokumen dalam bentuk XML untuk Requester (pemohon) dengan CICS data enclosed\cite{arsana2014rancang}.
\end{enumerate}



\section{Kesimpulan}

	Dari berbagai definisi tersebut dapat disimpulkan bahwa web service merupakan middleware sebuah internet yang memungkinkan berbagai sistem untuk saling berkomunikasi tanpa terpengaruh pada platform. Web service membungkus operasi-operasi ke dalam sebuah antarmuka yang ditulis dalam notasi XML. Antarmuka ini menyembunyikan detil implementasi dari layanan. Pertukaran informasi yang terjadi dalam web service juga menggunakan pesan dalam format XML \cite{saputra2integrasi}.




\chapter[Port]
{Port}
\input{section/1Port.tex}


\chapter[Aplikasi Web Service]
{Aplikasi Web Service}
%Nama Kelompok 1
%\begin{enumerate}
%\item Farid Ariyanto Saputra
%\item Nurgivani Syarifatul Husna
%\item Velariza Alvioletta
%\item Yogi Aditya Saputra
%\end{enumerate}


\section{REST}
\subsection{Pengertian REST}
REST atau singkatan dari Representational State Transfer adalah salah satu model arsitektur web yang memiliki aturan berupa interface yang seragam sehingga jika diterapkan dalam web service akan meningkatkan dan memaksimalkan kinerja dalam web sevice, terutama dalam performa dan kemudahan dalam memodifikasi. Dalam arsitektur REST, data-data serta fungsinya dianggap sebagai sumber daya yang biasa di akses melalui URI, yang merupakan singkatan dari Uniform Resource Identifier yang biasanya berupa link pada web.
\subsection{Konsep Kerja REST}
REST merupakan salah satu jenis web service yang menerapkan konsep perpindahan antar state. Jika digambarkan state bisa dikatakan seperti, saat browser meminta suatu halaman web, lalu server akan mengirimkan state halaman web yang diminta kepada browser. (Tidwell, D., 2001) Begitu juga REST mempunyai konsep kerja, dengan bernavigasi dengan link-link HTTP untuk melakukan suatu aktivitas tertentu, seakan telah terjadi perpindahan state, dari state satu ke state lainnya. Perintah  HTTP yang biasanya digunakan adalah fungsi GET, POST, PUT dan DELETE. Respon yang akan dikirim berupa hasil dalam bentuk XML yang sederhana tanpa ada protokol pembagian paket data supaya informasi yang diterima lebih mudah dibaca dan tidak ada pemecahan data pada pengguna atau client.
\subsection{Arsitektur REST}
Representational State Transfer (REST)
\begin{itemize}
\item Aritektur REST \\
REST adalah penyederhanaan dari HTTP. HTTP sering menjadi sesuatu yang tidak diperlukan atau sesuatu yang menyulitkan. Tetapi, dalam beberapa tahun sekarang, dengan kembalinya prinsip REST telah mengindikasikan bahwa HTTP telah cukup baik di atas segalanya.
\item API Flexibility dan Simplicity 
\item Keuntungan Arsitektur REST 
\item Representasi JSON (JavaScript Object Notation) 
\end{itemize}
\subsection{Perintah Dalam REST}
Perintah HTTP dalam REST antara lain :
\subsubsection{GET}
Dalam layanan RESTful webservices terdapat pemetaan metode HTTP yang salah staunya terdiri dari GET. GET merupakan salah satu fungsi CRUD dalam RESTful webservices. Fungsi GET adalah untuk menyediakan layanan read-only pada resource. Untuk kode respons sukses dan error yang digunakan untuk perintah GET adalah 200 untuk kode sukses dan 404 untuk kode error. 

POST, salah satu perintah HTTP yang berfungsi untuk membuat resource baru.
PUT, salah satu perintah HTTP yang berfungsi untuk memperbaharui resource yang sebelumnya sudah dibuat.
DELETE, salah satu perintah HTTP yang berfungsi menghapus resource yang telah di buat.
\subsubsection{PUT}
PUT merupakan metode http request yang biasanya untuk melakukan update data sumber daya. Put digunakan untuk mengganti sumber daya asli pada interface dengan prinsip rest dengan definisi metode http. Permintaan yang dihasilkan berfungsi untuk memperbarui dengan cara  mengidentifikasi permintaan uri atau dalam artian transfer representasi baru dari sumber daya dari klien ke server akan meminta alih-alih mentransfer atribut sumber daya sebagai seperangkat nama parameter dan nilai pada permintaan uri.
\subsubsection{DELETE}
Dalam layanan web RESTful, ada pemetaan antara metode HTTP  yang salah satunya ada perintah DELETE. DELETE. DELETE biasa digunakan untuk removes a resource atau menghapus sumber daya. Untuk kode respons sukses dan error yang digunakan untuk perintah DELETE adalah 200 untuk kode sukses dan 400 atau 404 untuk kode error atau gagal. 
\subsubsection{POST}
HTTP POST adalah salah satu layanan request dalam RESTful. Penggunaan HTTP POST merupakan suatu layanan yang digunakan ketika ingin membuat resource baru atau sumber daya baru. Pada sisi client, request di proses dapat dengan menambahkan resource yang teridentifikasi melalui body sebagai sub body dengan resource dalam request URL(Uniform Resource Language).

\section{HTTP Status Code}
\subsection{200 (OK)}
HTTP Status Code 200 memiliki arti permintaan yang dilakukan oleh pengguna telah sukses dieksekusi. Didalam RESTful, HTTP Status Code 200 sering digunakan untuk indikasi sukses dalam melakukan permintaan dari sisi client. Berbeda dengan 204, Status 200 mengembalikan informasi tergantung metode yang digunakan.
Sedangkan pada Status 204, selalu mengirimkan informasi dengan metode PUT, POST, maupun DELETE.
\subsection{201 (Accepted)}
Kode 202 (Accepted) artinya request diterima tapi server tidak melakukan apapun. Kode 202 juga biasa digunakan untuk tindakan yang prosesnya lama. Maksudnya , permintaan telah diterima untuk di proses, tetapi proses belum selesai. 
Maksud dari itu semua juga, server menerima permintaan dari beberapa proses lainnya (misalkan batch-oriented proses yang hanya berjalan sehari sekali) tanpa membutuhkan agen user penghubung server hingga proses selesai.
\subsection{202 (Created)}
Sebuah rest api akan merespon dengan kode status 201 setiap sebuah koleksi diciptakan, atau menambahkan toko, sumber daya baru atas permintaan klien. Mungkin juga ada waktu ketika sumber daya baru dibuat sebagai hasil dari beberapa tindakan pengontrol, yang dalam hal ini 201 juga akan menjadi respons yang tepat dan akurat.
\subsection{204 (No Content)}
Kode status 204 (No Content) biasanya dikirim sebagai respons atas permintaan PUT, POST, atau DELETE, ketika API REST menolak untuk mengirim sebuah pesan status kembali atau perwakilan apa pun di response message’s body.\\
Respons 204 TIDAK HARUS menyertakan message’s body, karena selalu diakhiri oleh baris yang kosong pertama setelah bidang header.

\chapter[Protokol]
{Protokol}
\input{section/1Protokol_kelompok3.tex}

\chapter[Arsitektur Client Server]
{Arsitektur Client Server}
\input{section/2arsitekturclientserver.tex}

\chapter[Common Gateway Interface]
{Common Gateway Interface}
%KELOMPOK 4 Blank-On1
%\begin{enumerate}
%\item Andri Fajar Sunandhar
%\item Cokro Edi Prawiro
%\item Fadila
%\item Sandro Samuel Sinaga
%\end{enumerate}



\section{Common Gateway Interface}
CGI merupakan metode yang dipakai untuk mempertukarkan data di antara server dan klien (browser). CGI merupakan sebuah standar dimana program atau script bisa mengirim data kembali ke web server dimana ia diproses, yaitu dengan menggunakan tag HTML standar untuk mendapatkan data dari seseorang, kemudian meneruskannya ke CGI. Selanjutnya CGI melakukan serangkaian aksi terkait data tersebut\cite{prihatmoko2013pengembangan}.



\par CGI adalah interface untuk menjalankan program-program eksternal,dibawah informasi server, biasanya server HTTP (walaupun CGI standar dirancang untuk lintasan-platform yang
menangani semua jenis hardware dan software yang berbeda, windows CGI 1.3 khusus untuk platform microsoft Windows 95/98 dan windows NT). Dengan CGI server bisa
melayani informasi yang tidak ada dalam format yang mudah dibaca oleh client,seperti data yang ada dalam database SQL, dan melakukan gateway antara dua sesuatu yang 
dihasilkan oleh browser client. Seringkali program gateway ini disebut script.

\par Sebuah server web memproses permintaan klien CGI menggunakan skrip atau aplikasi CGI. Sebagai contoh, ketika sebuah database ditanyakan oleh klien, 
server web bertindak sebagai gateway antara database dan klien. Server web mentransmisikan permintaan klien ke aplikasi CGI yang melakukan kueri basis data,
 memformat hasil dan mengembalikan data berformat HTML ke server web. Server web kemudian mentransmisikan data berformat HTML ke klien untuk ditampilkan kepada pengguna.

\par Di server, protokol yang diperluas lebih didukung oleh antarmuka gerbang umum (CGI) yang mengubah komunikasi dari perangkat I / O non-standar ke format yang kompatibel 
dengan transaksi atau program aplikasi data yang dapat dijalankan pada server atau komputer yang dipasangkan ke server. 
Dengan cara ini, CGI memungkinkan pemrosesan perintah kemampuan yang diperluas untuk dipisahkan dari fungsi komunikasi yang dilakukan oleh server.

\par Adapun pengertian lain dari Common Gateway Interface yaitu sekumpulan aturan untuk mengarahkan sebuah server web berkomunikasi dengan software dalam mesin yang sama begitu pula sebaliknya antara software CGI programs dengan web server. Setiap perangkat lunak dapat menjadi perogram CGI dengan syarat software tersebut dapat melakukan input dan output sesuai setandar CGI. CGI menjadi setandar menghubungkan untuk menghubungkan data informasi yang terjadi antara server dan aplikasi, seperti HTTP. Script CGI dapat mengirtimkan data kembali ke web server  dimana CGI diperoses. CGI merupakan interface antara halaman website dengan web server yang menjalankan perogram\cite{aditya2015analisis}.

\begin{figure}[ht]
\centerline{\includegraphics[width=1\textwidth]{figures/1arsitekturCGI.JPG}}

\caption{CGI Arsitektur Diagram} 
\label{1arsitekturCGI}
\end{figure}

Gambar\ref{1arsitekturCGI} menjelaskan bahwa antara HTTP server di pelantarai oleh CGI program dalam mengakses data 
dari database. Jadi jika data yang diminta di batasi atau tidak memiliki hak akses oleh CGI data tersebut tidak dapat di munculkan 
oleh web browser.  Cara untuk memahami perinsip dari (Common Gateway Interface ) CGI, dapat dicoba dengan melakukan click pada suatu URL suatu website.
setelah melakukan hal tersebut browser akan menghubungi HTTP web server dan meminta URL dari website tersebut. Kemudian web server tersebut akan
mengurai (prasing) URL dan akan mencari berkas dari link tersebut, bila ditemukan maka akan diteruskan ke browse, begitu juga sebaliknya jika tidak ada
maka akan diberikan pesan error. lalu web browser akan menampilkan hasilnya, baik url yang tadi diminta maupun pesan error karena URL yang dituju tidak ada. Meski begitu, ada kemungkinan untuk mengatur suatu HTTP server untuk membatasi akses terhadap suatu berkas.
jadi halaman URL yang dituju tidak bisa diakses hal tersebut merupakan fungsi dari CGI script. supaya lebih paham dapat dilihat 
pada arsitektir program CGI.

\par  CGI (Common Gateway Interface) memungkinkan server web memanggil suatu program, lalu mengirimkan data-data spesifik dari pengguna ke program tersebut.
 Hasil proses tadi diterima oleh CGI yang selanjutnya menyerahkannya kepada server web untuk kemudian, yang pada gilirannya akan mengirimkan
informasi tersebut kembali dalam bentuk HTML ke browser web pengguna, Server web kemudian mentransmisikan data berformat HTML ke klien untuk ditampilkan kepada pengguna. \cite{ibrahim2011sistem}.

\section{PHP and Common Gateway Interface interconnections }
\textit {Common Gateway Interface is a standard that is used to connect various application programs to web pages. One example of the programming language is PHP. PHP is a software that is open source and can pass across the various platform. Php can be run in 2 ways ie as apache module in web server and also as binary in Common Gateway Interface.This language was created in 1994 by Ramus Lerdoff.  Initially, PHP is a CGI program that is devoted to receiving input through forms displayed in web pages or browser. The PHP code is usually processed by a PHP interpreter which is usually executed as a native web server module or Common Gateway Interface}\cite{nahado2015bumbu}.


\section{Security in Common Gateway Interface }
\textit {Common Gateway Interface is used to connect WWW (World Wide Web) systems with software or other software on the web server. The presence of the Common Gateway Interface allows connection interactive between the user and also the web server. Common Gateway Interface itself is often used as a mechanism to get information from users through "fill out a form", access the database, or generate dynamic pages. Although in principle the mechanisms in the Common Gateway Interface do not have security holes, programs or scripts created as CGI can have security flaws either created intentionally or unintentionally. That is because CGI program itself is run on the web server to use the web server resources}\cite{afrianto2015materi}.


\section{The Application of Common Gateway Interface }
\textit {CGI is applied to the making of applications involving python language with PHP language. CGI itself is implemented or modified as CGI Fast CGI protocol, where its function in the application is as an interface in other applications to the web server which is an alternative facility to improve its own performance for CGI which is intended for the web server application process which is Apache web server to dynamic language another. Processes and handling from CGI to FastCGI can be demonstrated from the use of this working support facility such as python as the dynamic language used and the fastCGI module for the server to be used on RFC2109 proxy caching}\cite{kridoyono2017optimasi}.


\section{ Web Database and Common Gateway Interface Interconnections }
\textit {The Internet database development platform is adapted for approaches by connecting with or from CGI (Common Gateway Interface). The technical description includes a discussion of the Common Gateway Interface in which CGI functions as an interface for executing information on external programs, under server information, usually HTTP servers (although the Common Gateway Interface standard is designed for path-platforms that handle all different hardware and software.) Using a CGI server can serve information that does not exist in a format that is easy to read by the client, such as in existing data in SQL databases, and performs a gateway between two things generated by the client browser, which is usually the gateway program called scripts}.


\section{Honeypot}
Menurut Lance Spitzner Honeypot adalah sumber daya keamanan yang mempunyai nilai jika sistem disusupi atau diserang. Pada dasarnya Honeypot merupakan suatu alat untuk mendapatkan informasi dari penyerang. Honeypot merupakan sistem yang dirancang untuk diperiksa dan diserang.
Honeypot Dionaea merupakan salah satu Honeypot interaksi rendah yang bertujuan menangkap salinan malware berbahaya yang masuk ke dalam sistem. Malware tersebut biasanya ada pada layanan yang ditawarkan dalam jaringan. Dionaea menggunakan Python sebagai bahasa script dan libemu sebagai pemecah kode. Dionaea mendukung Internet Protocol v6 dan Transport Layer Security (TLS)\cite{andros2015implementasi}.

\section{Web Server Gateway Interface (WSGI) }
Salah satu keunggulan yang dijelaskan sebelumnya adalah karena Google App Engine dan Django dirancang untuk menggunakan standar WSGI untuk menjalankan aplikasi.
Django dapat berjalan dengan lingkungan server yang berbeda. Misalnya yang populer Server Apache didukung menggunakan mod python atau mod wsgi.
Juga untuk python maprelational Objectperational mendukung PostgreSQL, MySQL, SQLite dan Oracle.

\par Sebuah server web diatur di atas sistem operasi untuk mengirim permintaan HTTP, tetapi juga bisa melayani file statis seperti gambar, file JavaScript, halaman HTML, dll.
 Itu memproses pesan JSON dengan Flask, yang merupakan kerangka mikro untuk Python yang difokuskan pada kode aplikasi web, Karena server web tidak dapat berkomunikasi
 secara langsung dengan Flask, kami mengimplementasikan Web Server Gateway Interface (WSGI) untuk bertindak sebagai proxy antara server dan Python / Flask.

\section{ Python Program Langguage and Common Gateway Interface Interconnections }
\textit {In an application development with python programming language can be seen the relationship between Common Gateway Interface with python itself. Python programming language that is intepreter so that supports access in realtime (right at the point) in the data retrieval or the results of monitoring data. Another reason that is taken and considered is because the Python programming language using the OOP approach of Object Oriented Pprogramming so ideal and suitable for dingunakan on web programming in the Common Gateway Interface (CGI). To run the monitoring system as in the application to be built is very possible once to use or utilize the program interface that can bridge and help users through the web browser on the remote terminal. The interface of this program is called CGI or Common Gateway Interface which can usually be found by users and available on linux}\cite{ohara2005aplikasi}.

\section{Penyokong Aplikasi Web berbasis Python}
\par Perlu anda ketahui bahwa web berbasis Python memiliki beberapa penyokong untuk membuat website yangbaik.
Sebelum mengetahui apa saja penyokong webbsite Python alangkah lebih baiknya mengetahui apa Python itu.
Python adalah bahasa pemerograman yang dinamik, yang banyak digunakan secara luas dari banyak domain aplikasi, 
seperti pengembangan website dan internet, akses basisdat, Dekstop Graphical User Interface, ilmiah dan numerik, pendidikan,
pemerograman jaringan, permainan, dan Grapik 3D. dapat berjalan pada semua sistem oprasi, seperti linux, Windows, Mac, dan lainnnya.
bahasa pemerograman ini memiliki lisensi open-source yang dapat dengan gratis digunakan atau didistribusikan bahkan untuk penggunaan komersial\cite{andros2015implementasi}.

\subsection{Supervisord}
\par Perlu diketahui bahwa sebuah website memiliki tenggang waktu untuk tetap hidup, slah satu fungsi dari Supervisord adalah untuk mengaktifkan 
kembali sebuah website. Jika terjadi banyak permintaan (request) dan web server mati, maka untuk mengaktifkannya kembali harus masuk masuk ke server secara manual.
Dengan menggunakan supervisord hal trsebut dapat ditangani. Selain itu Supervisord merupakan alat untuk memonitoring sejumlah proses yang ada di bawahnya.  

\subsection{Gunicorn}
\par Green Unicorn yang kemudian disingkat menjadi `Gunicorn'merupakan sebuah HTTP Server yang digunakan untuk python, berbasis webserver gateway interface `WSGI' dan dikhususkan 
untuk lingkungan Unix-Like. Sebenarnya Gunicorn merupakan proyek yang diambil dari proyek Unicron untuk Ruby. Gunicorn memiliki penyesuaian yang sangat tinggi terhadap web berbasis WSGI 
seperti Django, Falcon dan lainnya. adapun beberapa fitur unggul dari Gunicorn antara lain :


\begin{enumerate}
\item Dukungan terhadap WSGI, Django, dan Paster
\item Manajemen prose worker secaraotomatis
\item Konfigurasi yang mudah
\item Konfigurasi pada banyak worker
\item Berbagai macam server hooks untuk extensibilitas
\item Kopatibel dengan Python 2.x atau 3.x
\end{enumerate}


\chapter{Instalasi PIP dan Contoh Penggunaan}
{Instalasi PIP dan Contoh Penggunaan}
\documentclass[12pt,a4paper]{article} 
\linespread{1.5}
\begin{document}
\title{Instalasi PIP dan Contoh Penggunaan}
\maketitle

\begin{itemize}
\item
Nama Kelompok 1\\
Farid Ariyanto Saputra 1164036\\
Nurgivani Syarifatul Husna 1164050\\
Velariza Alvioletta 1164056\\
Yogi Aditya Saputra 1164060 \\
\end{itemize}

\section{Python}
\subsection{Pengertian Python}
Python merupakan salah satu Bahasa pemrograman yang bersifat open source yang tertafsir oleh typing yang dinamis dan kuat. Python juga memiliki banyak library, seperti struktur data, files, dan jaringan. Bahasa pemrograman python juga banyak digunakan untuk berbagai keperluan, contohnya komputasi ilmiah, system administrasi, dan pengembangan web. Selain itu pula, keuntungan Bahasa pemrograman python yakni memiliki alat simulasi python gratis.
\subsection{Pengertian Python}
Python adalah suatu bahasa pemrograman yang bisa dikatakan bahasa pemrograman jaman sekarang, karena usianya sangat muda namun sudah banyak digunakan oleh programmer. Phyton dapat mendukung dalam membangun aplikasi berbasis desktop, web, mobile maupun lainnya. Untuk membangun sebuah aplikasi, bahasa pemrograman ini juga bisa digunakan menggunakan framework maupun tanpa framework. Namun, apabila tidak menggunakan framework akan membutuhkan waktu yang lama dalam tahap membangun aplikasi, begitu juga sebaliknya apabila menggunakan framework pembangunan aplikasi akan menjadi lebih cepat dan terstruktur, biasanya framework yang digunakan adalah Django, dimana disana terlah tersedia komponen seperti models, templates, views, forms, dan admin interface.
\subsection{Pengertian Python}
Python merupakan sebuah bahasa dalam pemrograman yang dibuat oleh Guido Van Rossum dan populer sebagai sebuah bahasa pemrograman berbasis Web. Python dikenal sebagai sebuah bahasa yang menggabungkan kapabilitas, kemahiran, dengan sintaksis kode yang jelas. Mengambil dari pengertian wikipedia, Python merupakan sebuah bahasa pemrograman interpretatif yang bisa digunakan dalam berbagai macam program web dengan filosofi perancangan yang berfokus ada tingkat keterbacaan kode.
\subsection{Pengertian Python}
Phyton merupakan salah satu Bahasa pemrograman kelas atas serta memiliki sifat intrepeter, object oriented, serta interaktif serta dapat berjalan pada sistem operasi seperti UNIX, MAC, Windows maupun platfrom lain. Karena Python merupakan bahasa pemrograman kelas tinggi, python dapat di kombinasikan dalam penggunaan tata kalimat dengan modul-modul yang telah siap pakai serta struktur data yang lebih efisien.

\section{PIP}
\subsection{Pengertian PIP}
PIP yang memiliki kepanjangan dari Pyhton Index Packaging. PIP itu sendiri adalah sebuah app store atau biasa disebut package manager yang biasa digunakan untuk mencari, mengunduh, menginstal serta mengelola package atau modules yang biasa ditemukan di PyPI ( Pyhton Package Index ). Dimana PyPI adalah sebuah library perangkat lunak untuk Bahasa pemrograman Pyhton.
\subsection{Pengertian PIP}
PIP merupakan singkatan dari python index packaging. PIP adalah sebuah aplikasi manajemen package yang biasa digunakan untuk menginstall dan mengelola package yang telah ditulis oleh python. Untuk menemukan packagenya, kita bisa mencari di situs Python Package Index (PyPI). Ada kurang lebih 134443 package dalam python yang bisa diinstall melalui PyPI.
\subsection{Pengertian PIP}
PIP (python index packaging) merupakan Package Management System yang biasanya digunakan untuk mengunduh dan mengelola package Python. Banyak sekali package yang bisa di temukan di PyPI. 
PIP bisa langsung digunakan di Python versi 2.7.9 dan versi 3 namun apabila menggunakan versi dibawahnya harus melakukan instalasi terlebih dahulu. 
Pip juga sebuah sistem untuk memeriksa perilaku sistem terdistribusi secara otomatis terhadap harapan programmer tentang sistem. Pip mengklasifikasikan perilaku sistem valid atau tidak valid, mengelompokkan perilaku ke dalam set yang dapat dipikirkan, dan menyajikan perilaku keseluruhan dalam beberapa bentuk yang sesuai untuk menemukan atau memverifikasi kebenaran perilaku sistem.
\subsection{Pengertian PIP}
PIP merupakan singkatan dari python index packaging yang merupakan package management sistem yang sering kali di gunakan untuk mengelola package python. Packages python dipasang dengan manajer paket pip, yang termasuk dalam semua lingkungan virtual. seperti sesi prompt perintah python akan memanggil versi alat ini daripada milik virtual enviroment yang diaktifkan.
\end{document}

%\chapter[Penggunaan Aplikasi Testing WebService]
%{Common Gateway Interface}
%\input{section/1penggunaanaplikasitestingwebservice.tex}

\chapter[Pemanggilan Modul]
{Common Gateway Interface}
\input{section/2pemanggilanmodul.tex}

\chapter[Instalasi PIP]
{Common Gateway Interface}
\documentclass[12pt,a4paper]{article} 
\linespread{1.5}
\begin{document}
\title{Instalasi PIP dan Contoh Penggunaan}
\maketitle

\begin{itemize}
\item
Nama Kelompok 1\\
Farid Ariyanto Saputra 1164036\\
Nurgivani Syarifatul Husna 1164050\\
Velariza Alvioletta 1164056\\
Yogi Aditya Saputra 1164060 \\
\end{itemize}

\section{Python}
\subsection{Pengertian Python}
Python merupakan salah satu Bahasa pemrograman yang bersifat open source yang tertafsir oleh typing yang dinamis dan kuat. Python juga memiliki banyak library, seperti struktur data, files, dan jaringan. Bahasa pemrograman python juga banyak digunakan untuk berbagai keperluan, contohnya komputasi ilmiah, system administrasi, dan pengembangan web. Selain itu pula, keuntungan Bahasa pemrograman python yakni memiliki alat simulasi python gratis.
\subsection{Pengertian Python}
Python adalah suatu bahasa pemrograman yang bisa dikatakan bahasa pemrograman jaman sekarang, karena usianya sangat muda namun sudah banyak digunakan oleh programmer. Phyton dapat mendukung dalam membangun aplikasi berbasis desktop, web, mobile maupun lainnya. Untuk membangun sebuah aplikasi, bahasa pemrograman ini juga bisa digunakan menggunakan framework maupun tanpa framework. Namun, apabila tidak menggunakan framework akan membutuhkan waktu yang lama dalam tahap membangun aplikasi, begitu juga sebaliknya apabila menggunakan framework pembangunan aplikasi akan menjadi lebih cepat dan terstruktur, biasanya framework yang digunakan adalah Django, dimana disana terlah tersedia komponen seperti models, templates, views, forms, dan admin interface.
\subsection{Pengertian Python}
Python merupakan sebuah bahasa dalam pemrograman yang dibuat oleh Guido Van Rossum dan populer sebagai sebuah bahasa pemrograman berbasis Web. Python dikenal sebagai sebuah bahasa yang menggabungkan kapabilitas, kemahiran, dengan sintaksis kode yang jelas. Mengambil dari pengertian wikipedia, Python merupakan sebuah bahasa pemrograman interpretatif yang bisa digunakan dalam berbagai macam program web dengan filosofi perancangan yang berfokus ada tingkat keterbacaan kode.
\subsection{Pengertian Python}
Phyton merupakan salah satu Bahasa pemrograman kelas atas serta memiliki sifat intrepeter, object oriented, serta interaktif serta dapat berjalan pada sistem operasi seperti UNIX, MAC, Windows maupun platfrom lain. Karena Python merupakan bahasa pemrograman kelas tinggi, python dapat di kombinasikan dalam penggunaan tata kalimat dengan modul-modul yang telah siap pakai serta struktur data yang lebih efisien.

\section{PIP}
\subsection{Pengertian PIP}
PIP yang memiliki kepanjangan dari Pyhton Index Packaging. PIP itu sendiri adalah sebuah app store atau biasa disebut package manager yang biasa digunakan untuk mencari, mengunduh, menginstal serta mengelola package atau modules yang biasa ditemukan di PyPI ( Pyhton Package Index ). Dimana PyPI adalah sebuah library perangkat lunak untuk Bahasa pemrograman Pyhton.
\subsection{Pengertian PIP}
PIP merupakan singkatan dari python index packaging. PIP adalah sebuah aplikasi manajemen package yang biasa digunakan untuk menginstall dan mengelola package yang telah ditulis oleh python. Untuk menemukan packagenya, kita bisa mencari di situs Python Package Index (PyPI). Ada kurang lebih 134443 package dalam python yang bisa diinstall melalui PyPI.
\subsection{Pengertian PIP}
PIP (python index packaging) merupakan Package Management System yang biasanya digunakan untuk mengunduh dan mengelola package Python. Banyak sekali package yang bisa di temukan di PyPI. 
PIP bisa langsung digunakan di Python versi 2.7.9 dan versi 3 namun apabila menggunakan versi dibawahnya harus melakukan instalasi terlebih dahulu. 
Pip juga sebuah sistem untuk memeriksa perilaku sistem terdistribusi secara otomatis terhadap harapan programmer tentang sistem. Pip mengklasifikasikan perilaku sistem valid atau tidak valid, mengelompokkan perilaku ke dalam set yang dapat dipikirkan, dan menyajikan perilaku keseluruhan dalam beberapa bentuk yang sesuai untuk menemukan atau memverifikasi kebenaran perilaku sistem.
\subsection{Pengertian PIP}
PIP merupakan singkatan dari python index packaging yang merupakan package management sistem yang sering kali di gunakan untuk mengelola package python. Packages python dipasang dengan manajer paket pip, yang termasuk dalam semua lingkungan virtual. seperti sesi prompt perintah python akan memanggil versi alat ini daripada milik virtual enviroment yang diaktifkan.
\end{document}

\chapter[Variabel]
{Variabel}

%Resume Variabel Kelompok 3 D4TI2B
%\begin{enumerate}
%\Fikri aldi nugraha                  1164038
%\Nur Arkhamia Batubara               1164049 
%\Miftahul Hasanah                    1164046 
%\Si Made Angga Dwitya P              1164053 
%\Widary Anggraini Mindo V Siahaan    1164057
%\end{enumerate}

\section{Pengenalan Variabel}
variabel merupakan konsep yang telah ada ukurannya atau telah diberi nilai. Variabel umur seseorang dapat diteliti dengan 
menanyakan kepada yang bersangkutan, kemudian hasilnya dicatat. Variabel pengetahuan orang tentang media televisi dapat diukur,  yaitu dengan memberi pertanyaan tentang jenis-jenis acara yang sering disaksikan, ketepatan jam tayang acara televisi tersebut, 
dan nama stasiun televisi yang menyiarkan acara. Variabel kedisiplinan pegawai bekerja dapat dilihat dari ketepatan waktu masuk  dan pulang kerja. Sesuai namanya variabel maka di dalamnya ada variasi (ukuran) atau nilai, misalnya tinggi-sedang-rendah, 
sering-jarang, disiplin-tidak disiplin, dan sangat memahami- memahami-tidak memahami.

Variabel memberikan konsep yang telah diberi ukuran tertentu. Ukuran inilah yang membedakan variabel dengan yang bukan variabel. 
Contoh jenis kelamin merupakan variabel dan dibedakan menjadi 2, yaitu perempuan dan laki-laki. Pendapatan seseorang atau 
kelompok masyarakat merupakan variabel yang dapat dibedakan tingkatannya menjadi kategori setuju, tidak setuju, dan tidak tahu. 
Umur seseorang dapat disebut variabel dan dibedakan (dikategorikan) menjadi empat, yaitu kanak-kanak, remaja, dewasa, dan 
manula/tua. Kekompakan kelompok disebut variabel dan setelah diukur dengan kriteria tertentu dapat dibedakan/dikelompokkan 
menjadi tinggi (kompak sekali), sedang (cukup kompak), rendah (kompak saja), dan sangat rendah (tidak kompak). Kemajuan negara 
dapat disebut variabel dan ditandai dengan ukuran (indikator) berupa tingkat pendidikan atau melek huruf (literasi), pendapatan 
nasional (product domestic brutto), dan tingkat ekspor produksi barang dan jasa. Jadi, variabel adalah konsep (keadaan, 
kegiatan) yang telah diberi ukuran tertentu dan dapat dijadikan objek atau unsur dalam penelitian ilmiah.

Data yang diperoleh dari responden dengan acuan hubungan antarvariabel yang umumnya bersifat dugaan (hipotetis). Jenis atau 
macam penelitian kausal adalah dua variabel (bivariat) atau lebih dari dua variabel (multivariate). Hal yang membedakan unsur 
kausal dengan penelitian exploratory dan descriptive adalah ada masalah/permasalahan, ada hubungan antarvariabel, ada hipotesis, 
ada metode pengumpulan data, ada sumber/narasumber/responden yang diteliti, dan metode analisis data yang jelas dan terstruktur, 
termasuk analisis kuantitatif yang umumnya menggunakan alat bantu statistik inferensial yang meneliti dan menghitung besarnya 
hubungan (korelasi) antarvariabel.

Dua variabel dapat dihubungkan bersama dalam satu penelitian komunikasi berjudul Hubungan antara Pengetahuan tentang Media 
dengan Terpaan terhadap Media. Dapat diduga (dirumuskan hipotesis), melalui penelitian ini bahwa ada sekelompok orang yang 
pengetahuan tentang media tinggi (variabel pendahulu/sebab) dan terpaan terhadap media juga tinggi (variabel pengikut/akibat). 
Ada pula pengetahuannya tentang media sedang, tetapi terpaan terhadap media tinggi, dan seterusnya. 

\section{Pengertian Variabel}
Variable adalah suatu objek penelitian yang bervariasi dan memiliki gejala yang bervariasi, atau sebagai suatu pusat penelitian yang 
dapat diukur. Variable juga sebagai konsep yang mempunyai  lebih dari satu nilai , suatu keadaan dan suatu kondisi. Pembahasan tentang 
variable sangat  penting  untuk suatu keperluan pada penetapan system penelitian, menstrukturkannya ke dalam teori penelitian sebagai 
landasan pengembangan hipotesis.

Variabel sebagai suatu tanda pengenal/identifier yang digunakan untuk mewakili suatu nilai tertentu di dalam satu proses program. 
Sebuah nilai dari suatu variable bisa diubah sesuai kebutuha, Nama dari suatu variable  yang dapat ditentukan sendiri oleh program. 
Sebuah operator dalam Bahasa pemograman merupakan sebuah symbol yang dapat dikenakan atau dapat mempengaruhi nilai dari satu atau 
beberapa variable .

Dalam suatu program tidak hanya variable biasa yang meiliki nilai yang berubah-ubah , tetapi suatu bahasa pemograman juga menggunakan 
jenis variable konstanta.  Nilai variable konstanta yang sudah didefenisikan tidak dapat diubah selama proses program. Kemudian aturan 
penamaan konstanta pada dasarnya sama dengan penamaan variable biasa, hanya saja ada beberapa perbedaan dari sisi cara mengisi 
konstanta serta tidak menggunakan tanda $ di depan nama konstantaya.

Variable menyimpan data dan nilai nya dapat berubah-ubah, variable memiliki hubungan yang sangat erat dengan tipe data. Karena setiap 
adanya suatu data perlu di tentukan apa tipe datanya, yang berfungsi sebagai processor dalam mengolah data tersebut. Tipe data adalah 
suatu kelompok yang memiliki jenis-jenis tertentu, dengan kata lain jenis dari data tersebut. 

\subsection{Contoh program yang mengandung variable}
\begin{verbatim}
<?php
$gaji = 100000;
$pajak = 0.2;
$thp = $gaji – ($gaji*pajak;
echo “Gaji sebelum ada pajak = Rp. $gaji <br>”;
echo “Gaji yang dibawa pulang = Rp. $thp”;
?>
\end{verbatim}

\subsection{Variabel pada PHP}

Pada Bahasa pemograman PHP pada dasarnya tipe data variable tidak didefenisikan oleh sang programmer, melainkan tipe data akan secara 
otomatis ditentukan oleh interpreter PHP. Akan tetapi untuk beberapa kebutuhan, programmer dapet mendefenisikan tipe data variable. 
PHP mendukung 8 (delapan) buah tipe data yang memiliki sifat primitive dalam programnya. yaitu :
\begin{enumerate}
1.Boolean
2. Integer
3. Float
4.string 
5. Array
6. Object
7. resource
8. NULL
\end{enumerate}

\section{Aturan Penamaan Variabel}
Dalam bahasa pemrograman, penamaan variabel menjadi salahsatu hal yang sangat penting dan perlu diperhatikan. 
Ada beberapa bahasa pemrograman yang memiliki sifat case-sensitive sehingga penggunaan huruf besar dan kecil dibedakan seperti pada saat penamaan variabel. Variabel dengan nama PERUSAHAAN akan berbeda dengan perusahaan. 
Beberapa aturan yang  digunakan dalam penamaan variabel adalah sebagai berikut:
\begin{enumerate}
\item Harus unik, tidak boleh ada variabel dengan nama yang sama pada satu ruang lingkup yang sama.
\item Harus dimulai dengan huruf(alfabet).
\item Maksimum 255 karakter, tetapi hanya 40 karakter pertama yang dianggap sebagai nama variabel. Karakter sisanya akan diabaikan.
\item Tidak boleh ada spasi. Sebagai ganti spasi dapat menggunakan karakter underscore “_”. Misalnya nama_dosen.
\item Tidak boleh menggunakan karakter-karakter khusus yang digunakan unutk operator, seperti +,-,*/,<,>,;,=,#,koma, dan lain-lain.
\end{enumerate}

\section{Deklarasi Variabel dan Variabel Dinamis}
Variabel di PHP bisa digunakan meskipun belum dideklarasikan. Penamaan variabel harus diawali dengan tanda ($) dan diikuti oleh nama ringkas. Nama variabel tidak boleh diawali dengan angka, tetapi bisa berisi angka dan karakter underscore (_). 
Untuk menghindari kesalahan dalam menggunakan variabel, perlu diketahui, nama variabel bersifat case sensitive.
\begin{verbatim}
//deklarasi dan inisialisasi variabel
$result   = 1 + 5;	 // valid
$Result   = 2 + 5;   // valid, beda dengan $result
$_result  = 3 + 5;   // valid
//deklarasi dan inisialisasi variabel
$1   = 1 + 5;	// tidak valid
$1res  = 2 + 5;	// tidak valid
$#res  = 3 + 5;	// tidak valid
$my-res  = 4 + 5;	// tidak valid
\end{verbatim}
Tanda dollar berfungsi untuk mempermudah membantu anda dalam membedakan variabel dengan fungsi dan keyword PHP.
Tanda dollar merupakan bagian dari nama variabel atau bisa juga dikatakan sebagai suatu operator yang mengacu ke memori.
Pada  bahasa pemrograman PHP,  anda dapat menciptakan nama variabel secara dinamis yaitu dengan menggunakan sintaks variabel-variabel (double variabel).
Variabel-variabel akan mengambil nilai dari suatu variabel dan memperlakukannya seperti nama variabel pada umumnya. Contohnya sebagai berikut:
\begin{verbatim}
$a = ‘Hello’;
$$a = ‘World’;
\end{verbatim}
Tanda double dollar atau ($$) adalah sintaks yang digunakan untuk menuliskan suatu variabel-variabel. Variabel yang juga disebut variabel dinamis ini bisa dipanggil melalui dua cara.

\section{Jenis – Jenis Pengukuran Variabel}
Di dalam variable juga terdapat alat alat yang digunakan dalam pengukuran. Sehingga varibel dikelompokan menjadi empat menurut bentuk   
pengukuran, itu dilakukan supaya banyak cara yang dilakukan dalam menentukan variable yang diinginkan. Variabel Nominal, yaitu variabel
yang dapat ditetapkan berdasarkan penggolongan. Variabel ini juga bersifat diskrit (bijaksana) dan saling pilih memillih antara satu 
kategori dan kategori lain. Beberapa alat pengukuran:

\subsection{Variabel Ordinal}
Variabel Ordinal adalah variabel yang dapat dibangun dan dibentuk berdasarkan jenjang atau tingakatan atribut tertentu. Jenjang 
tertinggi dan terendah juga sesungguhnya dan terendah sesungguhnya ditetapkan menurut kesepakatan. Dan oleh karena itu, angka 1 atau 
juga kelompok 10 dapat berada pada tingkatan jenjang yang tinggi sampai paling rendah. Variabel ini sangat banyak digunakan oleh 
beberapa pihak karena kemudahan dan ke fleksibelannya.

\subsection{Variabel Interval}
Variabel Interval adalah variabel yang dapat dibentuk dan dibangun dari pengukuran. Dalam pengukuran tersebut kita dapat mengasumsikan 
bahwa terdapat satuan pengukuran yang sama. Misalnya,variabel pendapatan artis dalam setahun.Variabel ratio juga merupakan suatu 
variabel yang memiliki permulaan angka nol yang mutlak. Suatu contoh juga seperti variabel umur yaitu: ada yang berumur 0, 1, 2, 3, 
4tahun dan sebagainya. Itu adalah beberapa pengukuran yang ada sehingga lebih mudah melakukan pengukuran variabel.

\section{Bentuk dan Ragam Variabel}
Selain pengukuran variabel seperti penjelasan diatas, variabel juga dibedakan dalam ragamnnya yang berbentuk berbeda-beda seperti 
variabel bebas, variabel tergantung, dan variabel penyela. Variabel bebas adalah variabel yang menentukan arah dan perubahan tertentu 
pada variabel tergantung. Sementara itu variabel bebas berada dalam posisi yang lepas dari pengaruh variabel tergantung. Dengan begitu 
variabel tergantung merupakan variabel yang dipengaruhi oleh variabel bebas.

Namun dalam penelitiann banyak dapat dibuktikan, tidak selamanya variabel bebas dapat mempengaruhi variabel tergantung.
Dengan demikian perubahan pada variabel tergantung tidak semata-mata disebabkan oleh variabel bebas, akan tetapi karena ada factor lain. 
Factor ini juga disebut variabel penyela. Variabel penyela ini berada di antara variabel bebas dan variabel tergantung yang di dalam 
suatu hubungan terdapat sebab akibat. Variabel ini dapat mempengaruhu variabel tergantung  dan pengaruh dari variabel bebas.


\section{Hubungan variable pada hipotesis dapat kita bagi 3 yaitu}
1.Model kontingensi dapat kita nyatakan dalam bentuk table silang. Contohnya saja yaitu hubungan antara variable antar agama dan antara 
variable parpol pada pemilu pada tahun 1997. 
2.Model asosiatif yang mana model ini terdapat diantara 2 buah variable yang mana keduanya sama sama ordinal atau keduanya sama sama 
interval.variabel ini mempunyai pola monoton linier.
3.Hubungan fungsional yaitu merupakan antara suatu variable yang berfungsi di dalam sebuah variable lain.

\section{Macam macam variable}
Berdasarkan cara pengukuran maka variable(Ferdinand,2006:12) dapat dibagi atas:
1.Yang pertama variable laten , merupakan suatu variabel bentukan dari hasil bentukan melalui indikator-indikator yang diamati dalam 
kehidupan duni nyata. Nama lain dari variabel ini yaitu factor atau konstruk maupun unobserved variable
2.Yang kedua yaitu variabel terukut merupakan sebuah variabel yang mana datanya harus kita cari melalui penelitian langsung kelapangan 
contohnya saja yaitu seperti survei secara langsung. Nama lain dari variabel ini diantaranya observed variable, maupun indicator 
variable.

\section{variabel dalam penelitian}
variabel dalam sebuah penelitian dapat kita bedakan menjadi:
a.Variabel independen merupakan variabel yang menjelaskan atau dapat mempengaruhi variabel lainnya.
b.Variabel dependen merupakan variabel yang dijelaskan atau dipengaruhi oleh variabel independen.
c.Variabel moderation  merupakan sebuah variabel yang akan menjadikan kuat  atau melemahkan hubungan langsung antara variabel independen 
dan variabel dependen.
d.Variabel intervening merupakan variabel yang mempengaruhi hubungan antara variabel variabel dependen dan variabel variabel independen 
menjadi hubungan yang tidak langsung.

\subsection{Bentuk hubungan dasar antara variabel yaitu}
1.Hubungan antara variabel independen dengan variabel dependen dapat dikatakan yaitu hubungan korelasional dan juga hubungan antara 
sebab akibat. Hubungan antar variabel ini dapat berupa positif maupun neatif.
2.Hubungan antara variabel independen dan dependen yang dimoderasi oleh  variabel moderating. Dimana variabel moderating tersebut 
mempengaruhi hubungan langsung antara variabel dependen dan independen.
3.Hubungan antara variabel dependen dan independenyang dimediasi oleh variabel intervening. Dimana mempengaruhi hubungan antara kedua 
variabel tersebut menjadi tidak memiliki hubungan tidak langsung.

\section{Array}
Array adalah variabel yang mampu menampung koleksi data terurut dengan tipe sama. 
Nilai-nilai data dalam array disebut dengan elemen-elemen array. Letak urutan elemen array ditunjukkan oleh indeks elemen array. 
Ada dua jenis variabel array, yaitu array yang ukurannya tetap (Fixed Array) dan array yang ukurannya dapat berubah (Dynamic Array).



\chapter{Fungsi Python}
{Fungsi Python}
\documentclass[12pt,a4paper]{article}
\usepackage[left=3.00cm, right=2.00cm, bottom=2.00cm, top=3.00cm]{geometry}
\linespread{1.5}
\begin{document}
\title{FUNGSI PYTHON}
\maketitle

\begin{itemize}
\item
NAMA KELOMPOK 4\\
Ajis Trigunawan			1164031\\
Alimu Dzul Ikroom		1164032\\
Muhammad Hanafi			1164092\\
Riki Karnovi			1164052\\
Yoga Sakti Hadi P		1164059\\
\end{itemize}

\section{Fungsi Phyton}
Python adalah bahasa pemrograman yang dibuat oleh Guido van Rossum dan popular sebagai bahasa skripting dan pemrograman Web. Merujuk pengertian dari wikipedia, Python adalah bahasa pemrograman interpretatif multiguna dengan filosofi perancangan yang berfokus pada tingkat keterbacaan kode. Python diketahui sebagai bahasa yang kemampuan dengan sintaksis kode yang sangat jelas. Salah satu fitur yang tersedia pada python adalah sebagai bahasa pemrograman dinamis yang dilengkapi dengan manajemen memori otomatis.

Python bisa digunakan dalam bermacam-macam pengembangan perangkat lunak dan juga bisa berjalan di banyak platform sistem operasi. Sebuah Komputer hanya bisa mengeksekusi program yang penulisannya dalam bahasa mesin atau bahasa tingkat rendah. Python adalah salah satu bahasa pemrograman tingkat tinggi, Sehingga agar bisa di eksekusi maka program harus diproses dulu sebelum dapat dijalankan. Keuntungan Python dengan bahasa tingkat tingginya yaitu lebih manusiawi.

Bahasa tingkat tinggi bisa dengan mudah dirubah portabel untuk disesuaikan dengan mesin yang menjalankannya. Hal ini beraneka ragam dengan bahasa mesin yang hanya dapat digunakan untuk mesin tersebut. Dengan berbagai macam kelebihan ini, maka tidak sedikit aplikasi ditulis menggunakan bahasa tinflrat tinggi. Proses mengubah dari bentuk bahasa tingkat tinggi ke tingkat lebih kecil dalam bahasa pemrograman ada rkra tipe.

Yakni interpreter dan compiler. interpreter membaca program berbahasa tingkat tinggi lau memproses program tersebut. Hal ini berarti interpreter melakukan perintah apa yang dikatakan dalam program tersebut. Dapat dikatakan. interpreter membaca per baris kemudian mengeksekusinya. 

Python merupakan Bahasa pemograman yang hampir tidak bisa dibedakan dengan C/C++, Setiap python mempunyai fungsi yang dapat mengembalikan sebuah Nilai, Tetapi di Python  bisa juga  tidak mengembalikan sebuah Nilai dan biasaya dikenal dengan nama subroutines yang terdapat pada pemograman VB. Python merupakan Bahasa yang tidak menggunakan compiler dan Bahasa ini juga bisa mengembangkan perangkat lunak, Membangun GUI desktop dan lain-lain.

Python dalam pengambangan web sering digunakan dibackend untuk pengelolaan database server, selain itu Python juga banyak digunakan untuk mengembangkan AI oleh google dan developer lainnya. Python juga di gunakan untuk pengembangan jaringan syaraf tiruan salah satunya project tensorflow. Python juga banyak digunakan dalam IOT karna python adalah salah satu bahasa mesin yang mudah dipelajari dan di pahami.

\end{document}


\chapter[Fungsi Python]
{Fungsi Python}
\documentclass[12pt,a4paper]{article}
\usepackage[left=3.00cm, right=2.00cm, bottom=2.00cm, top=3.00cm]{geometry}
\linespread{1.5}
\begin{document}
\title{FUNGSI PYTHON}
\maketitle

\begin{itemize}
\item
NAMA KELOMPOK 4\\
Ajis Trigunawan			1164031\\
Alimu Dzul Ikroom		1164032\\
Muhammad Hanafi			1164092\\
Riki Karnovi			1164052\\
Yoga Sakti Hadi P		1164059\\
\end{itemize}

\section{Fungsi Phyton}
Python adalah bahasa pemrograman yang dibuat oleh Guido van Rossum dan popular sebagai bahasa skripting dan pemrograman Web. Merujuk pengertian dari wikipedia, Python adalah bahasa pemrograman interpretatif multiguna dengan filosofi perancangan yang berfokus pada tingkat keterbacaan kode. Python diketahui sebagai bahasa yang kemampuan dengan sintaksis kode yang sangat jelas. Salah satu fitur yang tersedia pada python adalah sebagai bahasa pemrograman dinamis yang dilengkapi dengan manajemen memori otomatis.

Python bisa digunakan dalam bermacam-macam pengembangan perangkat lunak dan juga bisa berjalan di banyak platform sistem operasi. Sebuah Komputer hanya bisa mengeksekusi program yang penulisannya dalam bahasa mesin atau bahasa tingkat rendah. Python adalah salah satu bahasa pemrograman tingkat tinggi, Sehingga agar bisa di eksekusi maka program harus diproses dulu sebelum dapat dijalankan. Keuntungan Python dengan bahasa tingkat tingginya yaitu lebih manusiawi.

Bahasa tingkat tinggi bisa dengan mudah dirubah portabel untuk disesuaikan dengan mesin yang menjalankannya. Hal ini beraneka ragam dengan bahasa mesin yang hanya dapat digunakan untuk mesin tersebut. Dengan berbagai macam kelebihan ini, maka tidak sedikit aplikasi ditulis menggunakan bahasa tinflrat tinggi. Proses mengubah dari bentuk bahasa tingkat tinggi ke tingkat lebih kecil dalam bahasa pemrograman ada rkra tipe.

Yakni interpreter dan compiler. interpreter membaca program berbahasa tingkat tinggi lau memproses program tersebut. Hal ini berarti interpreter melakukan perintah apa yang dikatakan dalam program tersebut. Dapat dikatakan. interpreter membaca per baris kemudian mengeksekusinya. 

Python merupakan Bahasa pemograman yang hampir tidak bisa dibedakan dengan C/C++, Setiap python mempunyai fungsi yang dapat mengembalikan sebuah Nilai, Tetapi di Python  bisa juga  tidak mengembalikan sebuah Nilai dan biasaya dikenal dengan nama subroutines yang terdapat pada pemograman VB. Python merupakan Bahasa yang tidak menggunakan compiler dan Bahasa ini juga bisa mengembangkan perangkat lunak, Membangun GUI desktop dan lain-lain.

Python dalam pengambangan web sering digunakan dibackend untuk pengelolaan database server, selain itu Python juga banyak digunakan untuk mengembangkan AI oleh google dan developer lainnya. Python juga di gunakan untuk pengembangan jaringan syaraf tiruan salah satunya project tensorflow. Python juga banyak digunakan dalam IOT karna python adalah salah satu bahasa mesin yang mudah dipelajari dan di pahami.

\end{document}


\chapter[Hello World Python dan Identation]
{Hello World dan Identation}
%Resume Hello word python dan identation

%Kelompok 2 D4 TI / 2B

%Alwan Suryansah				1164033 
%Dinda Ayu Pratiwi				1164034
%Kurnia Sandi					1164042
%Teduh Sanubari					1164054
%Wildan Khaustara Wijaksana		1164058

\documentclass[12pt]{article}

\begin{document}
\section{Apa itu Pyton?}

\subsection{Journal JCONES}
	Tulisan ini mendiskusikan pengertian python , python adalah sebuah bahasa pemrograman model skripsi atau ( scripting language) terorientasi objek. atau bisa juga di artikan sebagai bahasa pemrograman yang freeware atau perangkat bebas , tidak ada batasan dalam penyalinan atau mendistribusikan . yang di dalamnya terdapat source code , debugger dan profiler \cite{perkasa2014rancang}.
\section{Membuat "Hello World" dengan Pyton}


\section{Pengertian Identation pada Pyton}


\section{Error yang Muncul serta Solusinya}


\section{Kesimpulan}

\end{document}




\chapter[Instalasi Flask]
{Instalasi Flask}
\documentclass[12pt,a4paper]{article} 
\usepackage{graphicx}
\begin{document}

\section{Flask}
\subsection{Definisi Flask}
Flask merupakan micro web framework yang ditulis dalam bahasa pemrograman Python. Flask disebut micro framework karena tidak membutuhkan tools tertentu. Framework ini juga tidak mempunyai database abstraction layer, validasi form, atau komponen lainnya yang dimana sudah menyediakan library pihak ketiga yang menyediakan fungsi lain. Akan tetapi, framework flask mendukung extension yang dapat menambah fitur aplikasi seakan-akan mereka diimplementasikan dalam flask itu sendiri.
\subsection{Instlasi Flask}
\subsubsection{On Linux}
Untuk melakukan instalasi flask, ada beberapa tahap yang harus dilakukan yaitu :
\begin{enumerate}
\item Pastikan anda sudah menginstal pyhton baik pyhton versi 2.x maupun 3.x. Jika belum instal, instal terlebih dahulu.

\item Untuk melakukan instalasi flask, kita harus membuat sebuah virtual environment lalu instal flask di dalamnya. Berikut adalah cara instal virtual enviroment / virtualenv.

\$ sudo apt-get install python-virtualenv

\item Lalu, buat satu direktori di hard disk kita untuk tempat pengerjaan project. 

\$ mkdir myproject

\$ cd myproject

\item Setelah itu, kita masuk ke direktori yang sudah di buat sebelumnya dan jalankan terminal untuk menjalankan virtualenv untuk membuat virtualenv di dalam direktori tersebut. Berikut adalah cara menjalankan virtualenv.

\$ virtualenv venv

New python executable in venv/bin/python

Installing setuptools, pip............done.

\item Lalu, berikut adalah cara untuk mengaktifkan virtualenv.

\$ . venv/bin/activate 

dan jika ingin menonaktifkannya, berikut caranya 

\$ deactivate

\item Selanjutnya, kita melakukan penginstalan flask

\$ pip install Flask
\end{enumerate}
\subsubsection{On Mac OS}
1. Instal pip terlebih dahulu menggunakan perintah :

	python get-pip.py

2. Selanjutnya instal flask dengan menggunakan perintah berikut :

	sudo pip install Flask

3. Sekarang, jalankan Simple.py untuk memastikannya berfungsi dengan menggunakan perintah berikut :

	python simple.py

Lalu, jalankan di web browser dengan perintah :

http://127.0.0.1:5000/

ketika sudah selesai, hentikan server dengan menekan kontrol-C.
\subsubsection{On Windows}

1.	Pertama, bukalah command prompt lalu ketikan perintah: pip install virtualenv

2.	Lalu, buatlah folder untuk aplikasi dengan cara mengetikan perintah: mkdir myproject.
Myproject bisa diganti dengan nama lain sesuai keinginan.

3.	Setelah itu, ketikan perintah untuk masuk ke dalam folder myproject: cd myproject. Lalu ketikan perintah: virtualenv flask

4.	Kemudian, ketikan perintah untuk masuk ke dalam folder myproject: cd myproject. Dan ketikan perintah: virtualenv flask

5.	Kemudian, ketikan perintah untuk menginstal flask sebagai berikut:
\begin{figure}[ht]
\centerline{\includegraphics[scale=1]{../figures/3instal.png} }

\caption{Instalasi Flask} 
\label{Flask}
\end{figure}

Pada gambar \ref{Flask} dijelaskan bahwa perintah instal flask

6.	Lalu, buat direktori app di folder myproject dengan perintah: mkdir app

7.	Kemudian masuk ke folder app dan buat folder baru bernama: static, templates dengan perintah mkdir. Sehingga dalam folder app terdapat folder static dan template.

8.	Dan instalasi selesai.

\subsection{Perbedaan Flask}
\hspace{1cm} Flask adalah salah satu microframework yang dapat digunakan di pyhton yang di buat dengan toolkit wsgi dan jinja2. Tentunya flask itu sendiri memiliki banyak perkembangan dari versi pertama saat ia di publikasikan hingga yang terbaru. Disini saya akan menjelaskan tentang sedikit perbedaan antara flask versi 1,0 dan flask versi 0.12. Berikut adalah perbedaannya :
Berikut adalah tabel \ref{table:perbedaan} perbedaan flask.
\begin{table}[h]
\caption{Perbedaan Flask}

\centering
\begin{tabular}{ccc}
\hline
&Flask 1.0&Flask 0.12\\
\hline
Python version&tidak mendukung python versi 2.6 dan 3.3&Support Python 2.6 dan 3.3\\
\hline
Release Date&26 April 2018&21 Desember 2016\\
\hline
\end{tabular}
\label{table:perbedaan}
\end{table}

\subsection{Perbedaan Flask dengan Pyramid}
Python adalah Bahasa pemrograman yang banyak memiliki kelebihan salah satunya dapat digunakan untuk membangun aplikasi web. Disini akan menjelaskan perbedaan Antara web framework Python flask dan pyramid.
1.	Flask pertama kali dirilih pada April 2010 yang dibuat dan dimaintain oleh Armin Ronacher. Flask merupakan web framework yang sederhana namun dapat memperluas dengan berbagai pustaka tambahan yang sesuai dengan kebutuhan.
2.	Pyramid adalah web framework Python yang open source dan pertama kali dikembangkan pada Desember 2010, sebelumnya sudah ada sejak 2008 bernama repose.bfg namun dikembangkan kembali. Pyramid merupakan web framework yang penuh kesederhanaan, minimalis, dokumentasi yang terkini,cepat,reliable, dan terbuka.
\subsection{Perbedaan Flask dengan CherryPy}
Adapun perbedaan antara framework Flask dan CherryPy, adalah CherryPy dibuat dengan cara mengadaptasi cara pembuatan aplikasi Python yang berbasis objek. Dengan framework CgerryPy, programmer dapat terbiasa membuat aplikasi pyhon berbasis objek tanpa menemui kesulitan yang berarti. Sedangkan Flask, merupakan framework python yang sederhana yang disimpan dalam satu berkas .py. Library dari flask dapat diperluas sesuai kebutuhan penggunanya.
\subsection{Perbedaan Flask dengan Django}
1.	Flask memberikan kesederhanaan , fleksibilitas , dan kontrol yang halus . Ini tidak unopinionated (yang memungkinkan untuk memutuskan bagaimana kita ingin menerapkan banyak hal).\\
2.  Django menyediakan pengalaman menyeluruh : pengguna mendapatkan panel admin , database interface , ORM , dan struktur direktori untuk aplikasi dan proyek out of the box.
\subsection{Perbedaan Flask dengan Bottle}
1.	Bottle adalah framework web mikro WSGI yang cepat, sederhana dan ringan untuk Python . Ini didistribusikan sebagai modul file tunggal dan tidak memiliki dependensi selain dari Python Standard Library .\\
2.	Flask adalah salah satu microframework yang dapat digunakan di dalam bahasa pemrograman pyhton yang di buat dengan toolkit wsgi dan jinja2.
\subsection{Perbedaan Flask dengan Web.py}
Python adalah Bahasa pemrograman yang banyak memiliki kelebihan salah satunya dapat digunakan untuk membangun aplikasi web. Disini akan menjelaskan perbedaan antara web framework Python flask dan web.py.
1.	Flask pertama kali dirilih pada April 2010 yang dibuat dan dimaintain oleh Armin Ronacher. Flask merupakan web framework yang sederhana namun dapat memperluas dengan berbagai pustaka tambahan yang sesuai dengan kebutuhan.
2.	Web.py dipublikasikan oleh Aaron Swartz dan dibuat untuk menjadi sebuah web framework yang sederhana tetapi powerful. Dokumentasinya juga cukup lengkap dengan paduan dasar, paduan tingkat lanjut, session, template, testing, basis data, refensi api, dan contoh-contoh kode yang bisa digunakan bebas.

\section{Virtual Environtment}
\subsection{Definisi}
\hspace{1cm} Virtual Environment adalah sebuah salinan interpreter python yang dapat kita gunakan untuk proses instalasi paket secara pribadi tanpa mempengaruhi python global yang ada di sistem kita.

Apabila Kita menginstal Virtual Environment atau virtualenv dapat mencegah konflik paket dan konflik pada penerjemahan python system. Virtual Environment merupakan salah satu komponen sebelum melakukan instalasi Flask.
\subsubsection{Pada Python 3}
Dalam Python terdapat dua versi yakni python 2 dan python 3. begitupun instalasi sebuah virtual environment pada python 2 dan python 3 berbeda. Karena dengan python 3, virtual environment atau virtualenv mensupport native dengan paket venv pustaka standar python. Untuk menambahkan atau menginstal pustaka tersebut pada python 3 adalah sebagai berikut :

\$ sudo apt-get install python3-venv

\section{Fitur Pada Flask}
\subsection{Fitur Pada Framewor Flask}
Framework flask memiliki bebrapa fitur yang dapat mendukung pembuatan web. Fitur-fitur tersebut diantaranya : 
1.	Berisi developing server dan debugger.
2.	Dukungan terintegrasi untuk pengujian.
3.	RESTful request dispatching.
4.	Menggunakan Jinja2 template engine.
5.	Dukungan untuk secure cookies (sisi klien sesi).
6.	100% WGI 1.0 compliant.
7.	Berbasis Unicode
8.	Dokumentasi yang ekstensif.
9.	Kompatibilitas Google App Engine
10.	Ekstensi yang tersedia untuk meningkatkan fitur-fitur yang dibutuhkan.

\end{document}


\chapter[Definisi Dekorator, Contoh Kode dan Fungsi]
{Definisi Dekorator, Contoh Kode dan Fungsi}
\documentclass[12pt,a4paper]{article}
\usepackage[left=3.00cm, right=2.00cm, bottom=2.00cm, top=3.00cm]{geometry}
\linespread{1.5}
\begin{document}
\title{definisi Dekorator, Contoh Kode dan Fungsi}
\maketitle

\begin{itemize}

\item
NAMA KELOMPOK 4\\
Ajis Trigunawan			1164031\\
Alimu Dzul Ikroom		1164032\\
Muhammad Hanafi			1164092\\
Riki Karnovi			1164052\\
Yoga Sakti Hadi P		1164059\\

\end{itemize}

\section{Definisi Dekorator, Kode dan Fungsi}

\subsection{Definisi Dekorator}
Python merupakan Bahasa pemrograman dengan fitur canggih dan ekspresif., salah satunya adalah dekorator. Dalam konteks desain, dekorator secara dinamis mengubah fungsi, metode atau pun kelas tanpa harus menggunakan subclass secara langsung. Ini merupakan hal yang ideal ketika kita perlu memngembangkan fungsi yang tidak ingin kita ubah. Kita dapat mengimplementasikan pola dekorator di mana saja dan di Python tentunya  dan Python memfasilitasi penerapannya dengan menyediakan lebih banyak fitur dan sintaksis yang ekspresif untuk itu. 

Python menawarkan fitur dekorator sejak versi 2.4. Secara sederhana, dekorator adalah pabrik fungsi. Mereka memungkinkan kita untuk mengubah fungsi Python biasa menjadi fungsi yang berfungsi seperti MATLAB®. Dekorator Python menerima fungsi tepat sebelum dimuat ke dalam ruang kerja saat ini. Dekorator dapat memanipulasi fungsi dengan cara yang sewenang-wenang. Dekorator fungsi memodifikasi setiap fungsi yang diterjemahkan oleh OMPC. Kami menggunakan dekorator untuk meniru keberadaan variabel nargin / nargout, untuk memungkinkan penugasan ke variabel baru, dan untuk menerapkan pengembalian tersirat.
Decorators sesuai dengan namanya secara bahasa memiliki arti yaitu pendekorasi atau penghias. Jika di kaitkan dengan python maka decorators memiliki arti yaitu adalah sebuah method yang ‘mengambil’ method lain dan menambahkan beberapa fungsi kepada method tersebut tanpa harus melakukan modifikasi. Bahasa simplenya adalah method yang melakukan ‘pendekorasian/penghiasan’ kepada method lainnya, bisa disebut dekorator hanyalah fungsi python dan pada dasarnya dekorator adalah pembentukan fungsi. Dekorator bekerja sebagai tempat. memodifikasi kode seebelum dan sesudah di eksekusinya fungsi itu, menambah fungsionalitas aslinya sehingga mendekorasinya lagi.

Sebelumnya kita harus memahami bahwa semua yang ada di Python adalah objek (termasuk kelas). Nama pengenal, seperti variabel yang kita deklarasikan merujuk kepada objek tersebut. Begitu juga dengan fungsi. Fungsi adalah termasuk objek juga. Satu objek bisa memiliki banyak pengenal (identifier) yang merujuk kepadanya dengan contoh kodenya:

\begin{verbatim}
def first(msg):
    print(msg)

first("Hello")
second = first
second("Hello")
\end{verbatim}

Pada saat kode di atas dijalankan, kedua fungsi first dan second menampilkan output yang sama. Di sini, variabel first dan second merujuk pada objek fungsi yang sama.
Sekarang mari kita tinjau hal yang lain. Sebuah fungsi bisa dijadikan sebagai argumen dari fungsi yang lain.
Bila Anda sudah pernah menggunakan fungsi seperti map, filter, dan reduce di Python, maka Anda sudah tahu tentang hal ini.
Fungsi yang menjadikan fungsi lain sebagai argumen disebut juga fungsi dengan orde yang lebih tinggi. Contohnya seperti berikut ini:

\begin{verbatim}
def inc(x):
    return x + 1
    
def dec(x):
    return x - 1
    
def operate(func, x):
    result = func(x)
    return result
\end{verbatim}


Kita bisa memanggil fungsi tersebut seperti berikut:

\begin{verbatim}
>>> operate(inc, 3)
4
>>> operate(dec, 3)
2
Lebih lanjut lagi, sebuah fungsi bisa mengembalikan fungsi lain.
def is_called():
    def is_returned():
        print("Hello")
    return is_returned
new = is_called()
#Outputs "Hello"
new()
\end{verbatim}

\begin{verbatim}
Pada contoh tersebut, is_returned() adalah fungsi bersarang yang didefinisikan dan dikembalikan tiap kali fungsi is_called() dipanggil.

Selain harus memahami tentang hal di atas, dalam mempelajari bahasa python juga sudah harus paham tentang python closure.
Kembali ke Decorator
Sebuah decorator pada python mengambil fungsi sebagai argumennya, menambahkan beberapa hal, dan kemudian mengembalikannya.
def make_pretty(func):
    def inner():
        print("I got decorated")
        func()
    return inner

def ordinary():
    print("I am ordinary")





\begin{verbatim}
Bila kode di atas kita jalankan pada mode interaktif, maka hasilnya adalah seperti berikut:

>>> ordinary()
I am ordinary

>>> # Mari kita buat decorator dari fungsi ordinary
>>> pretty = make_pretty(ordinary)
>>> pretty()
I got decorated
I am ordinary

\end{verbatim}

Pada contoh di atas, 
\begin{verbatim}
make_pretty()
\end{verbatim}  
adalah sebuah decorator. Pada baris perintah 

\begin{verbatim}
pretty = make_pretty(ordinary)
\end{verbatim}

Fungsi ordinary didekorasi dan fungsi kembaliannya diberi nama pretty.
Kita bisa lihat bahwa fungsi decorator menambahkan beberapa fungsionalitas ke fungsi asli. Hal ini mirip dengan pengemasan kado. Decorator bertindak sebagai bungkusnya. Objek yang didekorasi (isi kado) tidak berubah. Akan tetapi, ketika dibungkus, akan terlihat lebih bagus (karena didekorasi).

Dapat kita panggil fungsinya didalam fungsi lain.
Contohnya seperti berikut:

\begin{verbatim}
def greet(name):
    def get_message():
        return "Hello "

    result = get_message()+name
    return result

print greet("John")

# Outputs: Hello John
\end{verbatim}

Dekorasi Fungsi Dengan Parameter

Decorator di atas sangat simple dan hanya berlaku untuk fungsi yang tidak punya memiliki parameter. Bagaimana jika fungsi yang akan didekorasi memiliki argumen seperti berikut?
\begin{verbatim}
def divide(a, b):
    return a/b
>>> divide(2, 5)
0.4
>>> divide(2, 0)
Traceback (most recent call last):
...
ZeroDivisionError: division by zero
Sekarang kita akan membuat decorator untuk mengecek penyebab error ini.
def smart_divide(func):
    def inner(a,b):
        print("Saya akan membagi",a,"dan",b)
        if b == 0:
            print("Whoops! tidak bisa membagi dengan 0")
            return

        return func(a,b)
    return inner

@smart_divide
def divide(a,b):
    return a/b
\end{verbatim}

Kode yang menggunakan decorator ini akan mengembalikan None jika terjadi error.
\begin{verbatim}
>>> divide(2, 5)
Saya akan membagi 2 dan 5
0.4
>>> divide(2, 0)
Saya akan membagi 2 dan 0
Whoops! tidak bisa membagi dengan 0
\end{verbatim}

Dengan cara tersebut kita bisa mendekorasi fungsi yang memiliki parameter.
Bila kita perhatikan dengan baik, kita akan melihat kalau semua parameter dari fungsi yang di dalam decorator akan menjadi parameter dari fungsi yang didekorasi. Dengan itu, kita bisa membuat decorator yang lebih umum yang dapat bekerja dengan berapapun jumlah parameternya.


Fungsi sebagai parameter
Karena setiap parameter fungsi adalah referensi ke objek dan fungsi sendiri adalah objek juga, kita dapat meneruskan fungsi referensi ke dalam fungsinya - untuk parameter ke dalam fungsi.
Berikut contohnya: 
\begin{verbatim}
def g():
    print("dan ini aku g")
def f(func):
    print("yo ini aku f")
    func() 
f(g)

# Outputs: 	yo ini aku f
			dan ini aku g
\end{verbatim}


Dekorator adalah fungsi yang mengambil fungsi lain dan memperluas perilaku fungsi yang terakhir tanpa secara eksplisit memodifikasinya, "Dekorator" yang kita dimaksud adalah kepedulian terhadap Python tidak persis sama dengan DecoratorPattern yang dijelaskan. Dekorator Python adalah perubahan spesifik pada sintaks Python yang memungkinkan kita mengubah fungsi dan metode dengan lebih mudah (dan mungkin kelas dalam versi yang akan datang). Ini mendukung aplikasi yang lebih mudah dibaca dari Decorator Pattern tetapi juga kegunaan lain juga.

fungsi dari decorators juga adalah menambahkan atau merubah beberapa fungsionalitas ke fungsi asli. Hal ini mirip dengan pengemasan bungkus kado. Decorators bertindak sebagai bungkusnya. Objek yang didekorasi atau isi kado tidak berubah. Akan tetapi, ketika dibungkus, akan terlihat lebih menarik (karena didekorasi). Umumnya, kita mendekorasi fungsi dan menyimpannya ke variable.

Cara paling mudah untuk menentukan rute dalam aplikasi flask adalah melalui decorator app.route oleh aplikasi instance. Dibawah ini adalah contoh kodenya:\\
\begin{verbatim}
@app.route(‘/’)\\
Def index():\\
      Return ‘ <h1> Hello World!</h1> ’\\
\end{verbatim}

Dalam posting berikutnya dalam seri ini, dan penggunaan terakhir di app.route Flask () kita akan melihat bagaimana pola URL dinamis bekerja, dengan mendekonstruksi contoh sebagai berikut:

\begin{verbatim}
app = Flask(__name__)
@app.route("/hello/<username>")
def hello_user(username):
    return "Hello {} !".format(username)
    
# Untuk Outputnya adalah hello dan username yang di isikan 
di belakang parameter /hello/ usernamenya.
\end{verbatim}

Saat kita mendesain API untuk situs web, seringkali kita menggunakan metode DELETE untuk menghapus. Sayangnya, dan HTML hanya memungkinkan GET dan POST untuk metodenya, berikut contoh contoh pengunaannya:

\begin{enumerate}
\item contoh untuk methods get:
\begin{verbatim} 
@app.route('/resc/<int:id>', methods=['GET'])
def get_resc(id):
 ...
\end{verbatim}

\item contoh untuk methods post:
\begin{verbatim} 
@app.route('/resc/<int:id>', methods=['POST'])
def post_resc(id):
 ...
\end{verbatim}

\item contoh untuk methods delete:
\begin{verbatim} 
@app.route('/resc/<int:id>', methods=['DELETE'])
def del_resc():
 ...
\end{verbatim}
\end{enumerate}
Dalam pemrograman berorientasi objek, pola dekorator adalah pola desain yang memungkinkan perilaku untuk ditambahkan ke objek individu, baik secara statis atau dinamis, tanpa mempengaruhi perilaku objek lain dari kelas yang sama. Pola dekorator sering berguna untuk mematuhi Prinsip Tanggung Jawab Tunggal, karena memungkinkan fungsionalitas dibagi antara kelas dengan bidang perhatian yang unik.

Catatan:
Dekorator adalah fitur standar dari bahasa python. Penggunaan umum dekorator adalah untuk mendaftarkan fungsi sebagai fungsi handler yang akan dipanggil ketika peristiwa tertentu terjadi.

Dekorator milik paling mungkin untuk kemungkinan desain yang paling indah dan paling kuat di Python, tetapi pada saat yang sama konsep ini dianggap oleh banyak orang sebagai rumit untuk masuk. Tepatnya, penggunaan menghias sangat mudah, tetapi menulis dekorator dapat menjadi rumit, terutama jika Anda tidak berpengalaman dengan dekorator dan beberapa konsep pemrograman fungsional. 

Meskipun itu adalah konsep dasar yang sama, kami memiliki dua jenis dekorator dengan Python:
•	Dekorator fungsi
•	Dekorator kelas
Dekorator dalam Python adalah objek Python yang dapat dipanggil yang digunakan untuk memodifikasi fungsi atau kelas. Referensi ke fungsi "func" atau kelas "C" diteruskan ke dekorator dan dekorator mengembalikan fungsi atau kelas yang dimodifikasi. Fungsi atau kelas yang dimodifikasi biasanya berisi panggilan ke fungsi asli "func" atau kelas 
"C"


Panduan untuk dekorator fungsi Python
Python kaya dengan fitur canggih dan sintaksis ekspresif. Salah satu favorit saya adalah dekorator. Dalam konteks pola desain, dekorator mengubah fungsi suatu ke fungsi, metode atau kelas secara dinamis tanpa harus menggunakan subclass secara langsung. Ini sangat ideal ketika Anda perlu memperluas fungsi fungsi yang tidak ingin Anda ubah. Kita dapat mengimplementasikan pola dekorator di mana saja, tetapi Python memfasilitasi penerapannya dengan menyediakan lebih banyak fitur dan sintaksis yang ekspresif untuk itu.

\subsection{Jenis-Jenis Fungsi Dekorator}


Dekorator fungsi adalah sejenis deklarasi runtime tentang fungsi yang definisinya mengikuti. dekorator dikodekan pada baris tepat sebelum pernyataan def yang mendefinisikan fungsi atau metode, dan ini terdiri dari symbol @ yang di ikuti oleh referensi ke fungsi metafungsi (atau objek callable lain) yang mengelola fungsi lain.


\begin{verbatim}
@Decorator

       Class C: … C = \#Decorator(C\\
            X = C()\\
            Y = C() \#Overwrites x!\\
\end{verbatim}
            

\end{document}


\chapter[5RespondatauErorHandlingFlask]
{5RespondatauErorHandlingFlask}

\section{Pengertian Flask}
<<<<<<< HEAD
Flask merupakan sebuah microweb framework yang dicantumkan ke dalam bahasa pemrograman Python berdasar dari Werkzeug toolkit dan template engine Jinja2. Berlisensi BSD. Flask disebut micro framework karena tidak membutuhkan alat-alat tertentu atau pustaka. Flask tidak memiliki database abstraction layer, validasi form, atau komponen lain karena ada pustaka pihak ketiga yang menyediakan fungsi umum.
=======
Flask merupakan sebuah Microweb framework yang dicantumkan ke dalam bahasa pemrograman Python berdasar dari Werkzeug toolkit dan template engine Jinja2. Berlisensi BSD. Flask disebut micro framework karena tidak membutuhkan alat-alat tertentu atau pustaka. Flask tidak memiliki database abstraction layer, validasi form, atau komponen lain karena ada pustaka pihak ketiga yang menyediakan fungsi umum.

\section{Error Handling di Flask}
Error handling dalam flask akan munculnya "Internal Server Erro" pesan yang muncul di sesi terminal saat aplikasi dijalankan. Kemudian akan menampilkan  stack trace dari kesalahan yang terjadi. Stack trace sangat berguna untuk mencari error karena memberikan urutan pemanggilan mana yang menyebabkan sebuah error tersebut terjadi. Stack trace akan memperlihatkan bagian mana yang menjadi penyebabnya. Error ini muncul dari SQLAlchemy yang mencoba menulis username baru ke database, tapi database menolak karena kolom username sebelumnya sudah diatur dengan unique=True.

\subsection{Debug Mode}
Halaman error yang muncul sebelumnya cocok dipakai oleh aplikasi yang sudah diunggah ke sebuah production server. Jika ada error, user akan diberitahu dengan sebuah halaman khusus (yang nanti akan kita perbagus), dengan pesan error yang lebih detail disimpan di file log server.
Tapi saat aplikasinya sedang dibuat, kita tentu menginginkan debug mode untuk diaktifkan. Jika Flask aktif dalam mode ini, kita akan mendapatkan pesan error yang sangat membantu yang akan ditampilkan di browser.

\begin{verbatim}
(venv) $ export FLASK_DEBUG=1
\end{verbatim}

Saat aplikasi sedang dibuat, di butuhkan debug mode untuk mengaktifkan. Apabila Flask aktif dalam mode ini,  pesan error akan muncu dan sangat membantu dan akan ditampilkan di browser. Untuk mengaktifkan debug mode, stop dulu aplikasi, lalu atur environment variable. 
Setelah mengatur FLASK DEBUG, restart ulang server. Pesan yang ditampilkan saat memulai server menjadi agak berbeda dibanding sebelumnya:
\begin{verbatim}
(venv) microblog2 $ flask run
 * Serving Flask app "microblog"
 * Forcing debug mode on
 * Running on http://127.0.0.1:5000/ (Press CTRL+C to quit)
 * Restarting with stat
 * Debugger is active!
 * Debugger PIN: 177-562-960
 \end{verbatim}
 
 Membuat aplikasi crash seperti sebelumnya untuk melihat pesan interactive debugger di browser:
 
 \begin{figure}[ht]
\centerline{\includegraphics[width=1\textwidth]{figures/5eror.PNG}}
\caption{membuat aplikasi crash.}
\label{eror}
\end{figure}
\ref{eror} dijelaskan bahwa Debugger akan memungkinkan kita meng-expand setiap stack frame dan melihat source code yang terkait. Bisa juga membuka prompt Python di frame manapun dan mengeksekusi perintah Python yang valid, misalnya untuk memeriksa isi dari suatu variabel. Sangat penting untuk tidak mengaktifkan debug mode di production server. Sebagai keamanan tambahan, debugger yang berjalan di browser akan dikunci terlebih dahulu dan meminta nomor PIN yang bisa dilihat saat menjalankan perintah flask run. 


\subsection{Custom Error Pages}
Flask memberikan sebuah mekanisme bagi semua aplikasi untuk memasang halaman error khusus sehingga para user tidak perlu melihat halaman awal yang biasa-biasa saja. Untuk contoh, buat suatu halaman error untuk kode HTTP 404 dan 500, dua kesalahan yang paling sering terjadi pada aplikasi. Membuat halaman untuk halaman error lain tidak berbeda.

Untuk membuat custom error handler, dekorator @errorhandler akan dipakai. Disini akan menulis error handler di file app/errors.py.
app/errors.py: Custom error handlers berikut contohnya :
\begin{verbatim}
from flask import render_template
from app import app, db

@app.errorhandler(404)
def not_found_error(error):
    return render_template('404.html'), 404

@app.errorhandler(500)
def internal_error(error):
    db.session.rollback()
    return render_template('500.html'), 500
 \end{verbatim}
 
Fungsi untuk error handling sangat mirip dengan fungsi view. Untuk kedua jenis error tadi, kita akan menampilkan template khusus. Perhatikan bahwa kedua fungsi tersebut mengirimkan nilai kedua selain template yaitu kode nomor error-nya. Untuk semua fungsi view yang sudah dibuat, tidak perlu mengirimkan kode nomor 200 (untuk menandakan successful response) karena sudah diberikan secara otomatis.
Karena kedua fungsi di atas merupakan fungsi untuk menangani halaman error khusus, maka  perlu memberikan kode status untuk merefleksikan jenis error apa yang akan mereka tangani.

Error handler untuk kode 500 dapat dipanggil setelah sebuah database error, yang salah satu kasusnya terjdi bila ada username yang sama. Untuk memastikan semua percobaan database yang gagal tidak mempengaruhi database yang sudah ada, kita memanggil sesi rollback. Sesi ini akan membersihkan database dari percobaan mengisi data yang sebelumnya gagal (sehingga data yang terubah tidak setengah-setengah).

Berikut ini template untuk halaman error 404:
\begin{verbatim}



    <h1>File Not Found</h1>
    <p><a href="{{ url_for('index') }}">Back</a></p>

 \end{verbatim}
Template meng-extends base.html, sehingga mereka akan memiliki tampilan seperti halaman aplikasi yang normal. 
Agar error handler yang sudah kita tulis terdaftar di Flask, kita perlu mengimpor file app/errors.py sesudah menginisiasi aplikasi
Jika sudah mematikan debug mode dengan FLASK DEBUG 0 di sesi terminal lalu mencoba menganti username sekali lagi, maka kita akan mendapatkan halaman error yang sedikit lebih bersahabat.


\subsection{Mengirim Error Melalui Email}
Masalah lain dengan error handler bawaan Flask adalah tidak ada notifikasi, stack trace untuk setiap error dicetak di terminal, yang artinya output dari proses server harus dimonitor untuk melihat jika terjadi error. Saat aplikasi dijalankan saat melakukan pengembangan hal ini bisa dimaklumi, tapi jika aplikasi sudah di kirim ke server, siapa yang akan memeriksa output yang dikeluarkan? Jadi solusi yang lebih baik diperlukan disini.

Langkah pertama yang mesti dilakukan adalah memberikan detail server email ke file configuration:
Solusi yang akan dilakukan untuk mengatur Flask agar mengirim email setiap terjadi error.
\begin{verbatim}
config.py: Email configuration

class Config(object):
    # ...
    MAIL_SERVER = os.environ.get('MAIL_SERVER')
    MAIL_PORT = int(os.environ.get('MAIL_PORT') or 25)
    MAIL_USE_TLS = os.environ.get('MAIL_USE_TLS') is not None
    MAIL_USERNAME = os.environ.get('MAIL_USERNAME')
    MAIL_PASSWORD = os.environ.get('MAIL_PASSWORD')
    ADMINS = ['your-email@example.com']
    \end{verbatim} 
    
Variabel configuration untuk email diantaranya adalah server, port, penanda untuk mengaktifkan koneksi terenkripsi atau tidak, disertai dengan username dan password. Kelima variabel diambil dari environment variable. Jika server email tidak diatur di environment variable, maka itu akan menjadi pertanda bahwa pengiriman error email perlu dimatikan. Port server email juga perlu dimasukkan di environment variable, tapi jika tidak diatur, port standar nomor 25 akan dipakai. Data username dan password tidak wajib diberikan. Variabel ADMIN adalah daftar email yang akan menerima email error.

Flask menggunakan paket logging dari Python untuk menulis log dan  sudah memiliki kemampuan untuk mengirim log via email. Yang perlu dilakukan untuk mengirimkan pesan log tersebut ke email adalah menambahkan sebuah instance SMTPHandler ke objek Flask logger, yang bernama app.logger  kita hanya akan mengaktifkan email logger jika aplikasi dijalankan tanpa debug mode saat nilai app.debug berisi True juga saat server email ada di file configuration.

Kode-kode di atas akan membuat sebuah instance dari SMTPHandler, mengatur level-nya sehingga hanya membuat laporan error bukan warning, informational atau debugging message, lalu mengirim laporan error tersebut ke objek app.logger dari Flask. Ada dua cara untuk menguji fitur ini. Cara pailng mudah ialah dengan menggunakan server debugging SMTP dari Python. Server ini adalah server email fake , bukannya mengirim, ia akan mencetak email ke console (terminal). 

Untuk menguji kode yang kita buat dengan server ini, atur MAIL_SERVER=localhost dan MAIL_PORT=8025. Jika menggunakan Linux atau Mac OS, perlu menggunakan perintah sudo sehingga perintah tersebut bisa dijalankan. Jika menggunakan Windows, pastikan membuka aplikasi cmd sebagai administrator. Hak akses admin diperlukan dikarenakan port di bawah dari 1024 adalah port yang hanya bisa dijalankan oleh administrator. 

Biarkan server SMTP berjalan lalu kembali ke terminal awal, jalankan perintah export. MAIL_SERVER=localhost dan MAIL_PORT=8025. Pastikan variabel FLASK_DEBUG sudah diatur menjadi 0 atau tidak diatur sama sekali, sehingga aplikasi tidak mengirim email dalam debug mode. Jalankan aplikasi dan picu error. SQLAlchemy digunakan untuk melihat terminal yang menjalankan server email fake  yang akan menampilkan sebuah pesan email dengan kode-kode error.
                                                                                                                                          Cara pengujian yang kedua untuk fitur ini adalah dengan menggunakan server email asli. Di bawah ini akan konfigurasi untuk akun server email Gmail sebagai berikut:
\begin{verbatim}
export MAIL_SERVER=smtp.googlemail.com
export MAIL_PORT=587
export MAIL_USE_TLS=1
export MAIL_USERNAME=<your-gmail-username>
export MAIL_PASSWORD=<your-gmail-password>
\end{verbatim}

jika menggunakna Microsoft Windows, selalu gunakan set sebagai ganti export disetiap perintah di atas.

% contoh aplikasi web service
% web service
% protokol
% port

% HTTP
% URL
% POST
% GET



\bibliographystyle{IEEEtran}.
\bibliography{references}.

\printindex

\end{document}
