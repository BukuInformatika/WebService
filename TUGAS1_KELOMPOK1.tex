\documentclass[12pt, a4paper]{article}
\begin{document}

\begin{itemize}
        \item
        Ahmad Syafrizal Huda (1164062) \\
        Annisa Fathoroni (1164067) \\
        Puad Hamdani (1164084) \\
        Rahmi Roza (1164085) \\
        Tasya Wiendhyra (1164086) \\
\end{itemize}
    Secara harfiah internet adalah kependekan dari “interconnected network” yang berarti rangkaian komputer yang terhubung satu sama lain. Hubungan melalui suatu sistem antar perangkat komputer untuk lalu lintas data itulah yang dinamakan network. Sebagai contoh yaitu LAN, MAN dan WAN. Misalnya LAN. LAN merupakan singkatan dari Local Area Network yang menghubungkan komputer atau jaringan dalam area tertentu seperti kantor, sekolah atau warnet. Jadi komputer yang terhubung melalui satu jaringan dan saling berkomunikasi dengan waktu dan wilayah yang tidak terbatas, disebut internet.

    Internet di dunia bisnis untuk pergantian informasi, pencatatan produk,media yang mempromosikani, surat elektronik, bulletin boards, kuesioner elektronik, dan mailing list. Biasanya digunakan untuk berkomunikasi,berdiskusi, dan dilibatkan secara proaktif dan interaktif dalam perancangan, pengembangan,pemasaran, dan penjualan produk. Pemasaran melalui internet terdapat 2 metode, yaitu push dan pull marketing. keutamaan dari perencanaan bisnis yang didapat di internet ialah komunikasi dunia dan interaktif diantaranya, serta menyediakan informasi penting dan pelayanan yang sesuai dengan kebutuhan konsumen, juga meningkatkan kerja sama.

    Internet of Things (IOT) merupakan perkembangan keilmuan yang sangat menjanjikan untuk mengoptimalkan kehidupan berdasarkan sensor cerdas dan peralatan pintar yang bekerjasama melalui jaringan internet (Keoh, Kumar, & Tschofenig, 2014).Menurut (Burange & Misalkar, 2015) Internet of Things (IOT) adalah struktur di mana objek. orang disediakan dengan identitas eksklusif dan kemampuan untuk pindah data melalui jaringan tanpa memerlukan 2 arah antara manusia ke manusia yaitu sumber ke tujuan atau interaksi manusia ke komputer.

    Internet Service Provider (ISPs) menghubungkan pengguna dengan perusahaan/pebisnis provider ke internet yang luas atau lebih dikenal dengan internet publik. Para perusahaan ini harus bersaing satu sama lainnya dalam harga, kecepatan, kinerja, dan lainnya. Tetapi mereka pun harus bekerja sama dalam menyediakan konektivitas global dengan semua hal yang berkaitan dengan internet. Tier 1 ISP adalah ISP yang memiliki hak untuk mengakses ke dalam routing internet global tapi tidak membeli transit atau pemberhentian apapun dari orang lain atau pihak lainnya.

    Pengaruh dan dampak internet sebagai alat media komunikasi. Seperti halnya media massa yang lain, keberadaan internet ini membangkitkan berbagai pertanyaan akan efek negatif yang ditimbulkannya, selain keberadaan efek positif seperti penyampaian dan pengiriman informasi yang cepat dan update melalui fasilitas-fasilitas e-mail, sural kabar online, forum diskusi dan juga chatting serta beragam situs-situs yang ada yang memperkaya khasanah pengetahuan penggunanya. Lebih lanjut keberadaan media komunikasi ini acapkali dianggap sebagai penyebab perilaku asosial penggunanya.

    Ada beberapa macam koneksi yang dapat dilakukan agar anda dapat terkoneksi dengan internet, dan melakukan aktivitas online sepuasnya. Jenis-jenis koneksi juga menentukan kecepatan akses internet anda, dan tentu saja membutuhkan biaya. Perlu diperhatikan pada kecepatan akses, KBps kependekan dari “kilobit per second” (“kilobit per second”), yang berbeda dari KBps (“kilobyte per second/kilobita per detik”), dimana 1 bita = 8 bit.
    
    Sejarah Internet : Rangkaian pusat yang membentuk internet diawali pada tahun 1969 oleh ARPA (Advance Research Project Agency), sebuah badan yang dibentuk pada tahun 1958 oleh Amerika yang terdiri dari para peneliti dan teknisi dari universitas dan laboratorium yang ada di Amerika. Awalnya badan ini dibentuk untuk menyaingi Rusia, yang saat itu lebih maju dibidang satelit. Para peneliti bekerja, tidak harus di satu lokasi, untuk membuat penelitian dan mendedikasikan hasil penelitian tersebut untuk perkembangan teknologi Amerika Serikat.

\end{document}

