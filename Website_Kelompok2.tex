\documentclass[12pt, a4paper]{article}
\linespread{1.5}

\begin{document}
\title{World Wide Web}
\maketitle

\begin{itemize}
\item
	Imron Sumadireja (1164076) \\
	Jesron Marudut (1164077) \\
	Lusia Violita Aprilian (1164080) \\
	Mhd. Zulfikar Akram Nst. (1164081) \\
\end{itemize}

\section{Pengertian Website}
World wide web (www atau web) merupakan halaman situs informasi yang dapat diakses secara cepat atau sarana antar muka informasi di internet. Web dapat menggabungkan teks, grafik, dan multimedia. Web memudahkan penggunanya untuk mengakses informasi melalui konsep hypertext sehingga memungkinkan  suatu text untuk menjadi acuan membuka dokumen laindo. Informasi dapat mudah disebar dan diakses.

Sementara itu World wide web (www) dikembangkan pertama kali oleh Tim Berners-Lee pada tahun 1989. Pada awalnya, Tim mengusulkan WWW sebagai suatu cara berbagai dokumen diantara para peneliti. Dokumen online dapat diakses melalui alamat unik yang disebut Universal Resource Locator atau URL. Selain itu WWW tidak hanya dikembangkan untuk keperluan para peneliti, namun juga dikembangkan untuk kalangan pendidikan, bisnis dan perorangan. Berdasarkan penjelasan singkat diatas dapat disimpulkan bahwa antara web dan internet memiliki hubungan yang sangat erat walaupun keduangnay tidak bisa dikatakan sama. Web merupakan bagian dari layanan yang dapat berjala di atas teknologi internet.

\end{document}