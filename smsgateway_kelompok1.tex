\documentclass[12pt,a4paper]{article} 
\linespread{1.5}
\begin{document}
\title{SMS Gateway}
\maketitle

\begin{itemize}
\item
Nama Kelompok 1\\
Farid Ariyanto Saputra 1164034\\
Nurgivani Syarifatul Husna 1164050\\
Velariza Alvioletta 1164056\\
Yogi Aditya Saputra 1164060 \\
\end{itemize}

\section{SMS GATEWAY}
\subsection{Pengertian SMS Gateway}
Gateway dapat diartikan sebagai suatu tali penghubung antara dua sistem yang berbeda sehingga dapat terjadi suatu transaksi data antara masing-masing sistem tersebut. SMS gateway adalah suatu teknologi SMS (Short Message Service) yang mengolah SMS yang dilakukan secara komputerisasi dan memanfaatkan layanan SMS itu sendiri untuk berbagai keperluan serta tujuannya masing-masing. \\
SMS gateway adalah suatu system yang menghubungkan Antara telepon selular dengan server dan diinformasikan melalui SMS atau Short Message Service. Selain itu, SMS gateway juga bias diartikan sebagai layanan dua arah. Maksudnya selain bisa menerima pesan dari luar,  SMS gateway bisa mengirim pesan balasan secara otomatis ke nomor tujuan dalam waktu tertentu.
\subsection{Cara Kerja SMS Gateway}
Mekanisme utama yang dilakukan dalam suatu sistem SMS adalah melakukan pengiriman short message dari satu terminal customer ke terminal customer yang lain. Hal ini dilakukan untuk sebuah entitas yang bernama  Short Message Service Center (SMSC) atau Message Center (MC). Ketika pesan SMS dikirim dari handphone (mobile orginated) pesan tersebut tidak langsung dikirim ke handphone tujuan (mobile terminated) namun terlebih dahulu ke SMSC lalu pesan tersebut dikirimkan ke handphone tujuan.
\subsection{Kelebihan SMS Gateway}
Beberapa keunggulan dalam layanan SMS, antara lain : \\
\indent 1. tidak membutuhkan biaya mahal, yang penting nomor dalam keadaan \indent aktif dan pengiriman relatif cepat serta di jamin terkirim ke no tujuan. \\
\indent 2. layanan ini bersifat fleksibel, pengguna dapat mengirim pesan\\ \indent dimanapun dan kapanpun saja. \\
\indent 3. mudah digunakan oleh semua kalangan terutama kalangan non-IT. \\
\end{document}